\documentclass{article}
\input{WebMacros}

\usepackage{fullpage}
%\usepackage{times}
\def\SqlKeyword#1{\textbf{#1}}
\def\URL#1{\texttt{#1}}

\title{GGobi and Database Management Systems from GGobi\\ 
Reading data from MySQL \& Postgres
}
\author{
Duncan Temple Lang, Lucent Bell Labs
}


\begin {document}
\maketitle

\begin{abstract}
  A common source of data is Database Management Systems (DBMS), and
  specifically relational databases.  We describe the facilities
  currently provided by Ggobi for reading data from MySQL
  (\URL{http://www.mysql.org}), a particular DBMS.
\end{abstract}

The ability to read GGobi input from a single XML opens a large number
of possibilities.  Reading data from a database however is also a
powerful extension to any system.  Databases offer many conveniences
for storing values in comparison with files, URLs, etc.
These advantages include
\begin{itemize}

\item Large quantities of data can be managed by software
 designed for that purpose. 

\item Only a single instance of the data is needed, as applications
  can dynamically query the segment of that data they are interested
  \textsl{when they need it}.  These client applications can be
  written in any of numerous languages that provide a library or
  bindings for the DBMS.  Support for many of these systems is
  provided in Python, Perl, Java, C, etc.

\item Data can be updated dynamically and made available to clients.

\item Updates to the data (e.g. correction of errors) need be
  performed in a single location, simplifying administration
and providing greater data integrity.

\item Many of the common, basic computations can be performed
using SQL specified in a client process and performed locally on the
data within the DBMS.

\item Most DBMS provide a security model for limiting access
to the data, allowing great resolution for the administrators.

\end{itemize}


\section{MySQL \& Postgres}
Postgres and MySQL are database management systems ...

\section{GGobi \& MySQL}
When one starts GGobi, one can instruct it
to use a connection to a MySQL database
using the \flag{-mysql}.
\begin{verbatim}
 ggobi -mysql
\end{verbatim}
No value need be specified with this flag.  However, if a filename is given,
the values for establishing the database connection and performing the
query are read from that file.  The file is a property file consisting
of \texttt{name: value} pairs.  The entries have the same names as
those values presented in the GUI described below.  See
\ref{tab:mysqlLogin}.  Any white-space is trimmed away from the
values. An example of such a file is given
in figure \ref{fig:mysqlInputs}.
\begin{figure}[htbp]
  \begin{center}
    \leavevmode
\begin{verbatim}
Host:        tuolomne
User:        duncan
Database:    ggobi
Data query:  SELECT * from flea;
#Password: 
Color query: Hi there
\end{verbatim}    
    \caption{Default SQL Connections}
    \label{fig:mysqlInputs}
  \end{center}
\end{figure}



This approach allows one to store symbolic information about how to
obtain the data.  For basic security reasons, the password used to
access the dataset should not be stored in this file.  In the case
that it is, it is read and used directly to establish a connection.
If it is omitted from the file, a GUI dialog (described below) is
displayed that allows one to interactively set all the values
governing the SQL connection. The values from the fields specified in
the properties file will be inserted as defaults in the GUI.  The user
need only fill in the password to complete the connection.  In the
rare circumstances that no password is needed (e.g. for convenient
public access to the database), simply providing an empty value for
the Password property is adequate to indicate that a connection should
be attempted with no password.



In the case that the \flag{-mysql} is specified with no filename, a
dialog window (see \ref{fig:dbmsGUI}) is displayed that allows one to
interactively specify any and all of the details about the database
and the data that one wants.  The fields are described in table
\ref{tab:mysqlLogin}

\begin{table}[htbp]
  \begin{center}
    \leavevmode
\begin{tabular}{ll}\label{tab:mysqlLogin}
host & the name or IP address of the host 
machine on which the MySQL server (daemon)
is running.\\
user &  the login name to use when creating the
connection with the database. This is not a login
for the machine, but for the database. This
is created by the database adminstrator
using the \SqlKeyword{GRANT} command in SQL. \\
password & the password required for the
specified user on that database server.
This is also set by the administrator
when granting access to a database and/or table.
\\
database &   Identifies
which database (collection of tables) within
the database server you wish to access. \\

port & Identifies the port on the host machine on which the database
server is receiving connections.  It is unlikely that you will have to
set this. It is only used when the database server has been customized
to run on a non-standard port. \\ 

socket & ? \\
flags &  ? \\
Data query & the SQL query that is sent to
the server which generates 
a table (result set) which is  treated
as the raw data.  \\
Segments query &  an SQL query
determining which observations are connect.
The result should be a table with $2$ columns
and the entries should identify 
the observations by index (starting at $1$).
 \\
Color query & this is an SQL query that returns a table of RGB values.
A fourth column, if present, should contain the names by which color
values can be referenced.
\end{tabular}
    \caption{SQL Parameters}
    \label{tab:mysqlLogin}
  \end{center}
\end{table}
The non-data queries are not implemented yet.  It would be useful to
integrate the data sources so that appearance was specified locally,
but the actual data values were queried via SQL.
\begin{figure}[htbp]
  \begin{center}
    \leavevmode
    \pdfimage{Figures/dbmsGUI.jpg}
    \caption{DBMS Input GUI}
    \label{fig:dbmsGUI}
  \end{center}
\end{figure}

\end{document}
