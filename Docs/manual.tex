\documentclass[11pt]{article}
\usepackage{fullpage}
\bibliographystyle{plain}
%\bibliographystyle{asa}
%\bibliographystyle{harvard}
%\bibliographystyle{apalike}


% Issues:
%  I haven't worked out a consistent notation:  when to display
%  a button's contents in bf, when to use em, etc.

%%%%%%%%%%%%%%%%%%%%%%%%%%%%%%%%%%%%%%%%%%%%%%%%%%%%%%%%%%%%%%%%%%%%%%

% If it turns out that I can use pdflatex, I can use jpeg graphics
% files and go straight to pdf.

\newif\ifpdf
\ifx\pdfoutput\undefined
  \pdffalse
\else
  \pdftrue
\fi
  
\ifpdf
  \pdfoutput=1
  \usepackage[pdftex]{graphicx}  % uncomment if using graphicx
  %\usepackage[pdftex]{hyperref}  % uncomment if using hyperref
\else
  %\usepackage{graphicx}  % uncomment if using graphicx
  \usepackage[colorlinks=true,
    linkcolor=webgreen,
    filecolor=webbrown,
    citecolor=webgreen]{hyperref} % uncomment if using hyperref
  \definecolor{webgreen}{rgb}{0,.5,0}
  \definecolor{webbrown}{rgb}{.6,0,0}
  % dvips -z
\fi
%%%%%%%%%%%%%%%%%%%%%%%%%%%%%%%%%%%%%%%%%%%%%%%%%%%%%%%%%%%%%%%%%%%%%%

\hyphenation{XGobi}
\hyphenation{GGobi}
\hyphenation{ggobi}
\hyphenation{di-men-sion-al two--di-men-sion-al}
\input epsf
\def\Rcirc{\mbox{\small{\ooalign{\hfil\raise.07ex\hbox{\tiny R}\hfil\crcr\mathhexbox20D}}}}

\parindent 0in
\parskip 6pt

\begin {document}

\title {{\tt ggobi} Manual}
\author{
Deborah F. Swayne, AT\&T Labs -- Research \\
Dianne Cook, Iowa State University \\
Andreas Buja, AT\&T Labs -- Research \\
Duncan Temple Lang, Lucent Bell Labs
}

\date{September 2001}

\maketitle

\begin{abstract}

The GGobi software is a data visualization system with state-of-the-art
interactive and dynamic methods for the manipulation of views of
data.  It represents a significant improvement on its predecessor, XGobi,
with multiple plotting windows, more flexible color management, XML file
handling, and better portability to Windows.

The most significant change may be that ggobi can be embedded in other
software and controlled using an API (application programming
interface).  This design has been developed and tested in partnership
with R.  When ggobi is used with R, the result is a full marriage
between ggobi's direct manipulation graphical environment and R's
familiar extensible environment for statistical data analysis.

It has the same graphical functionality whether it is running
standalone or embedded in other software.  That functionality
includes 2-D displays of projections of points and edges in
high-dimensional spaces, as well as scatterplot matrices, parallel
coordinate and time series plots.  Projection tools include average
shifted histograms of single variables, plots of pairs of variables,
and grand tours of multiple variables.  Views of the data can be
reshaped.  Points can be labeled and brushed with glyphs and colors.
Several displays can be open simultaneously and linked for labeling
and brushing.  Missing data are accommodated and their patterns can
be examined.
\end{abstract}

\section{Introduction}

This paper gives an overview of the layout and functionality of GGobi,
interactive graphical software for exploratory data analysis.  Readers
who are familiar with XGobi will find much that is familiar in GGobi's
design, and might want to read section \ref{slbl:xgobi} first, where
key differences between the two programs are described. There are
several papers that describe parts of the functionality existing in
both packages
\cite{BCS95,SCB97,SB98,BAHM88,CBC93,CBCH95,CB95,BCAH95c}.

You will find that you can use GGobi for simple tasks with virtually
no instruction.  All that a user needs is some cursory knowledge of
the developments in interactive statistical graphics of the last 15
years, as well as a willingness to experiment with the sample data
provided and guided by the tooltips.  In parallel with the hands-on
learning process, it is probably useful to acquire a basic
understanding of the overall layout and functionality of the system.
The greatest success is obtained by users who have gained experience
with the system and combine it with creativity and data analytic
sophistication.

We begin with a tutorial, and move on to describe GGobi in detail.

\section{Tutorial}

Several sample data files are included with the GGobi distribution, in
a directory called {\bf data,} and there you will find the file {\bf
olive.xml}, a dataset on olive oil samples from Italy \cite{FALT83}.
This data set takes advantage of an XML input/output format, a feature
in GGobi that wasn't available in XGobi. It requires that GGobi has
been compiled with XML support. If Ggobi has been compiled without XML
support, the set of files olive.dat, olive.col, olive.row, similar to
the XGobi data file input, can be used. The olive oil data consists of
the percentage composition of 8 fatty acids (palmitic, palmitoleic,
stearic, oleic, linoleic, linolenic, arachidic, eicosenoic) found in
the lipid fraction of 572 Italian olive oils. There are 9 collection
areas, 4 from southern Italy (North and South Apulia, Calabria,
Sicily), two from Sardinia (Inland and Coastal) and 3 from northern
Italy (Umbria, East and West Liguria).

Start GGobi for the olive oils data by typing:

\begin{quote}
ggobi olive
\end{quote}

Two windows will appear, the GGobi console (control panel),
and a scatterplot, as shown in Figure~\ref{fig1}.

The console has a panel of controls on the left, labeled {\bf
XYPlot}, and a variable selection region on the right.  You can see
that the scatterplot contains a 2-dimensional projection of the data,
a plot of Area vs Region.  Move the mouse to one of the variable labels
in the variable selection region, and leave it in place until the
tooltip appears, explaining how to select new variables for the
plot.   Begin to get a feeling for the data by looking at several of
the 2d plots:  Area vs palmitic, Region vs oleic....

\begin{figure}[h]
%\epsfxsize 6in \centerline{\epsfbox{figures/manual1.ps}}
\hbox{\pdfimage width 3in {Figures/olive1.jpg}\pdfimage width 3in {Figures/olive2.jpg}}
%\vspace{.15in} \centerline{ \parbox{6in}
\caption{Layout of a ggobi session.  The plotting window contains a
scatterplot of the Area vs Region, from the olive oils data.
}%}
\label{fig1}
\end{figure}

Next, get acquainted with the main menubar for the console by
exploring each of its menus.  Pay particular attention to the
Display, ViewMode and Tools menus.

\begin {itemize}
\item The Display menu is the interface for opening new plotting
  windows.
\item The ViewMode menu is the interface for specifying both the
  projection (1d, 2d, 3d or higher) and mouse interactions (for scaling
  the plot, highlighting points, and so on) for the current plot.
\item The Tools menu lets you open other windows to manipulate
  characteristics of the data and the view.
\end {itemize}

Using the Display menu, open another scatterplot display. Notice that
the new window has a narrow white band drawn around the outside of
the plotting area:  that means that the new window is now the
``current display'' and its plot is the ``current plot.''  Click in the
plotting region of the other scatterplot to make it the current plot,
and notice what happens in the variable selection region of the
console when you alternate between the two:  it should always show
the variables that are plotted in the current plot.

Now set up a plot of linoleic vs eicosenoic in the first scatterplot,
and Area vs Region in the second, and make that the current plot.
Using the ViewMode menu, choose {\bf Identify}. Look at the buttons
inside the leftmost portion of the ggobi console. Notice that
they're contained by a frame labeled {\bf Identify,} which is the
view mode of the current plot.  This frame contains most of the row
labeling controls, which are described in the section
\ref{slbl:Identify}. Move the cursor around the plot window with mouse
movement, and observe the labels of the point closest to the cursor is
written. The labels for this data set show the geographic area
where the sample was taken.

Using the ViewMode menu, choose {\bf Brush.}  Look at the buttons and
menus inside the leftmost portion of the ggobi console.  Notice
that they're contained by a frame labeled {\bf Brush,} which is the
view mode of the current plot.  This frame contains most of the
brushing controls, which are described in the section
\ref{slbl:Brush}.  A few of the brushing controls, though, are in the
main menubar: in the Reset menu and at the bottom of the Options menu.

The rectangle that appears in the current plot is the ``paintbrush,''
and dragging it over points changes their color. Change the color of
the brush by opening up the {\bf Choose color \& glyph} panel (Figure
\ref{fig2}). Now change the size of the brush, holding down the
middle or right button and dragging the mouse.  Make it almost as
tall as the window but very thin, and then hold down the left button
and drag the mouse to  paint the first region, then the second and
third regions.  Since you're brushing in the {\bf Transient} style,
the points resume their original color when the paintbrush no longer
surrounds them.

While you brush, keep an eye on the plot of linoleic vs eicosenoic,
and notice where the painted points fall in that scatterplot (Figure
\ref{fig3}).  Using two linked 2d plots is one way to explore the
relationship among three variables.

\begin{figure}[htp]
\vspace{-0.5in}
%\epsfxsize 6in \centerline{\epsfbox{figures/manual1.ps}}
\pdfimage width 4in {Figures/olive-colorchoose.jpg}
%\vspace{.15in} \centerline{ \parbox{6in}
\caption{The chooser for selecting point color and glyph and line
type and thickness to paint with. It also allows setting the plot
background color, and has a full color wheel for changing the set of
colors.
}%}
\label{fig2}
\end{figure}

\begin{figure}[htp]
%\epsfxsize 6in \centerline{\epsfbox{figures/manual1.ps}}
\hbox{\pdfimage width 3in {Figures/olive-brush1.jpg}\pdfimage width 3in {Figures/olive-brush2.jpg}}
%\vspace{.15in} \centerline{ \parbox{6in}
\caption{Brushing in a GGobi session. Brushing Region 1 shows that it
corresponds to a cohesive cluster in the variables linoleic and 
eicosenoic acid.
}%}
\label{fig3}
\end{figure}

Now change to {\bf Persistent} brushing and paint the 3 Regions
different colors. You may also want to use different symbols to
represent the different Areas, as you'll see shortly in Figure 5.

Next open up the {\bf Variable transformation} tool, select all of
the fatty acid variables -- hold the control button down and select
them one at a time, or select the first one, then hold down the shift
button down while you select the last one.  Once they're all
highlighted, choose {\bf Standardize} in the stage 2 transformation
panel to standardize all the variables to have mean 0 and variance
1.  This tool has numerous transformations, which can be applied in
sequence.  Stage 0 transformations are used to adjust variables'
values so that they're within the range of some stage 1
transformations. Stage 2 transformations allow for post-processing of
stage 1 transformations, so that the variables can be logged and then
standardized, for example.

But that was just an instructive detour; click ``Reset all'' to
to turn off standardization and any other transformations you've
explored.

\begin{figure}[htp]
%\vspace{-0.5in}
%\epsfxsize 6in \centerline{\epsfbox{figures/manual1.ps}}
\pdfimage width 6in {Figures/olive-var.jpg}
%\vspace{.15in} \centerline{ \parbox{6in}
\caption{The variable manipulation panel contains basic summary 
statistics of variables. It can also be used for adding new variables
and for variable subset selection before launching multi-plot
displays.
}%}
\label{fig4}
\end{figure}

Now look at the plot of linoleic vs eicosenoic. Open up the {\bf
Color \& glyph groups ...} tool, and use the Hide and Show buttons to
make groups of points invisible and then visible again (Figure
\ref{fig5}). Finally, use these controls hide to the cases from
geographic region 1, and then {\bf Rescale} the plot.

\begin{figure}[htp]
\hbox{\pdfimage width 3in {Figures/olive-brush3.jpg}\pdfimage width 3.5in {Figures/olive-brush4.jpg}}
\caption{Toggling clusters of points on and off according to color/glyph value.}
\label{fig5}
\end{figure}

Now switch into {\bf 2D Tour} using the {\bf ViewMode} menu. The
variable checkboxes change to variable circles. Click on the circles
for the variables palmitoleic through arachidic to toggle these variables
into the tour, and click on Region and Area to toggle these variables out.
Then click {\bf Reinit}.  The tour should now include 7 variables.

The scrollbar at the top of the tour controls is used to adjust the
speed of the tour.  Drag it to the right to speed up the rotation.
The circle at the bottom left of the plot window displays the axes
for the tour. These can be removed by toggling the ``Show axes''
button on the plot window's {\bf Options} menu.  Pause the tour when
you see a separation of the two regions (as seen in Figure
\ref{fig6}).

Now you are going to use manual tour controls to sharpen up the
separation by manually changing the coefficients of the variables in
the projection.  Click on the {\bf Manip} (purple) button below the
variable selection region of the console, and then click on linoleic
acid (a purple circle will be seen now in the variable circle for
linoleic acid).  Select {\bf Oblique} in the {\bf Manual
manipulation} menu.  Now in the plot window, hold down any mouse
button while you drag the cursor. The coefficient corresponding to
linoleic acid will increase and decrease following the mouse,
inducing a rotation of the scattercloud.  Similarly rotate oleic acid
and arachidic acid in and out of the plot, with the aim of finding a
projection where the two regions are maximally separated (for
example, Figure \ref{fig6}).  There are 5 choices of manipulation
mode (unconstrained oblique, vertical, horizontal, radial and
angular) to explore.

\begin{figure}[htp]
\hbox{\pdfimage width 3in {Figures/olive-tour1.jpg}\pdfimage width 3in {Figures/olive-tour2.jpg}}
\caption{Tour plots revealing difference between regions 2 and 3 of the
olive oils.}
\label{fig6}
\end{figure}

Next use the Tools menu to open the {\bf Variable manipulation} tool,
(Figure \ref{fig4}).  It contains a table with a few basic statistics
for each variable, and the buttons below the table allow you to set
variable limits, add new variables, and a few other things.  Try
clicking the mouse in the table -- you can select one row of the
table at a time, or use the control and shift keys as modifiers to
select more than one.  Select all the fatty acid variables.

Now open a parallel coordinates display using the {\bf Display} menu.
%Using the {\bf Selection Mode:} menu on the control
%panel select {\bf Delete} mode to remove Region and Area from the
%parallel coordinates plot. Then switch to {\bf Append} mode and add
%the remaining acid variables.
Select the first plot in the parallel coordinates display by
clicking on it, and use the {\bf ViewMode} menu to switch to {\bf
Brush} mode. Choose a new color (say yellow) and large closed glyph,
and transiently paint the case with a very low value on palmitic
(Figure \ref{fig7}). This case also has a very high value for oleic
acid, and low value for linolenic acid.

\begin{figure}[htp]
\hbox{\pdfimage width 6in {Figures/olive-par.jpg}}
\caption{Brushing in a parallel coordinates display reveals an outlier in 
palmitic, oleic and linolenic acids.}
\label{fig7}
\end{figure}

Using the {\bf Color \& glyph groups} tool from the {\bf Brush}
controls, click on the appropriate `S' button to bring the Region 1
cases back into the plot.  Using the {\bf File} menu, save the data,
preserving the colors and glyphs that have been assigned during this
session.

This has been a brief introduction to the use of GGobi. The following
section contains more detailed information on the functionality.

\section{Layout and functionality}

\subsection{The major functions}

Across the top of the control panel, as seen in in Figure \ref{fig1},
stretches a row of menu buttons:  {\bf File}, {\bf Display,} {\bf
ViewMode}, {\bf Tools}, {\bf Options} and {\bf DisplayTree} are
always visible; others appear as appropriate.

As expected, the {\bf File} menu contains items for selecting
input/output functions and for exiting.

The {\bf Display} menu allows a new plotting window to be opened. 
The display types include
\begin{itemize}
\itemsep 0em
\item scatterplot,
\item scatterplot matrix, 
\item parallel coordinates plot, and
\item time series plots.
\end{itemize}

If the data has missing values, a scatterplot or scatterplot matrix
of the ``missingness'' information can be opened.  Each display type
is discussed in section ~\ref{DisplayTypes}.

The {\bf ViewMode} menu contains items to set the projection 
and to the interaction type:
\begin{itemize}
\itemsep 0em
\item 1DPlot: 1-D dotplots and average shifted histograms,
\item XYPlot: 2-D scatterplots,
\item 1D Tour: 1-D tour,
\item 2D Tour: 2-D tour,
\item 2x1D Tour: a correlation tour; that is, independent 1-D tours on
      horizontal and vertical axes,
\item Scale: axis scaling,
\item Brush: setting point glyphs, edge types, and point and edge colors,
\item Identify: labeling points
\item Move Points: direct manipulation of point positions.
\end{itemize}

When you choose a new view mode, the controls at the left of the main
window will change correspondingly:  each mode has its own
parameters, and its own rules for responding to mouse actions in the
plotting windows.  The view modes are discussed in
section~\ref{slbl:ViewModes}.

The {\bf Tools} menu gives access to 
\begin{itemize}
\itemsep 0em
\item a variable manipulation table,
\item a variable transformation pipeline,
\item an variable sphering panel,
\item jittering controls,
\item a panel for automatically brushing by mapping a selected
      variable to the color scale,
\item a panel for hiding groups of cases,
\item subsetting functions for systematic and random subsampling, 
\item a tool for managing missing values.
\end{itemize}

Each tool is discussed in section ~\ref{Tools}.

There are a couple of distinctions between view modes and tools.  The
view mode functions determine the mouse interactions in the display
windows, while none of the Tools does.  Furthermore, view mode
functions populate the control panel in the ggobi console, while
tools open their own windows, as shown in Figure~\ref{fig4}.  For
example, the {\bf variable manipulation tool} is a window containing
a table with one row of data for each variable and several buttons,
and the {\bf variable transformation tool} is also a separate window,
with a list of variable names and a set of transformation menus.

The {\bf Options} menu allows users to set some options for the main
control window:  whether tooltips are displayed and whether the
control panel is shown.  In some view modes, it contains additional
options that are specific to that mode.

Other menus are present only during certain modes:  you will
sometimes find an {\bf I/O} menu or a {\bf Reset} menu.

The {\bf DisplayTree} menu allows users to open a tree listing
all the currently open display windows, each of which may contain
several plots.

\subsection{Graphical displays}
\label{slbl:GraphicalDisplays}

\subsubsection{Current display, current plot}

Since there are multiple displays, some of which contain multiple
plots, the question arises:  Which plot in which display window
corresponds to the console?  If you select the {\bf Scale} mode,
how can you tell which plot is going to respond?

There is in GGobi a notion of the ``current display'' and the
``current plot.''  (We need both because some displays, like the
scatterplot matrix, contain multiple plots.) The current plot is the
one which is outlined with a thick white border; the current display
is the one which contains the current plot.

To reset the current plot and display, just click once (left, right
or middle) in the plot you wish to address.  To understand the
effects of this selection, open a few displays and set them in
different ViewModes, then click on different plots and see what
happens.  The white border should follow your actions, and the
console should update so that its panels correspond to the
current display type and view mode.

\subsubsection{Display types}
\label{DisplayTypes}

Each display type is briefly described here. 
As mentioned earlier, there are presently four main display types:

The {\bf scatterplot} display is a window containing a single
scatterplot.  By default, it includes axes; they can be turned off
using the Options menu in the main control window.  It has the
largest number of view modes of any display, and each of the
projection modes has its own rules for variable selection.  The
variable selection interface for the 1DPlot and XYPlot modes is a
column of checkboxes: clicking left on one of these selects a
variable to be plotted horizontally, clicking middle or right selects
a vertical variable.  The tour modes all use a set of labeled
circles for variable selection, and they provide some feedback about
current projection.  The variable selection behavior for the tour
modes will be described in the section for each mode.

The {\bf scatterplot matrix} is a window containing a symmetric matrix
of scatterplots for the chosen variables.  The plots along the diagonal
are ASHes (Average Shifted Histograms).  The matrix is required to be
symmetric, and that constraint affects its variable selection behavior.

\begin{itemize}
\item Replace:  First select one of the ASH plots along the diagonal
  to tell ggobi uniquely which variable to replace, then click on
  one of the checkboxes in the variable selection region.
\item Insert:  First select one of the plots along the
  diagonal to tell ggobi uniquely where to insert the new plot,
  then click on one of the variable checkboxes.  (GGobi will
  not add a variable that's already plotted.)
\item Append:  Click on one of the variable checkboxes to append
  a new plot after all other plots.  (GGobi will not add a variable
  that's already plotted.)
\item Delete:  No plot selection is required; just click on the
  variable you want to delete.
\end{itemize}

The {\bf parallel coordinates} display contains a single parallel
coordinates plot, which can be arranged horizontally or vertically.
(To understand this plot if you are encountering it for the first
time, imagine deconstructing a high-dimensional scatterplot and
arranging its axes in parallel instead of orthogonally.  To represent
case $i$, think of drawing a dot on each axis,  with the point on
axis $j$ being the value of $x[i][j]$, and then connecting the dots
into one set of connected line segments \cite{In85,We90}.)  The line
segments are drawn by default, but you can turn them off using the
{\bf Options} menu on the display menubar.

By default, the plots are simple dotplots, but they can also be drawn
using one of the two methods for 1D plots:  as a textured dot
strip or an ASH.

The variable selection behavior works as follows:

\begin{itemize}
\item Replace:  First select one of the plots, then click on one
  of the variable checkboxes to replace its plotted variable.
\item Insert:  First select one of the plots, then click on
  one of the variable checkboxes to insert a new plot before
  the current plot.  (GGobi will not add a variable that's
  already plotted.)
\item Append:  Click on one of the variable checkboxes to append
  a new plot after all other plots.  (GGobi will not add a variable
  that's already plotted.)
\item Delete:  No plot selection is required; just click on the
  variable whose plot you want to delete.
\end{itemize}

The {\bf time series display} contains a row or column of 2-variable
plots with a common axis, usually a time variable.  By default, the
points are connected with line segments.  

The time series display uses the checkbox variable selection interface,
and the behavior of the checkboxes depends on the state of the
{\bf Selection mode} option menu in the control panel.

\begin{itemize}
\item Replace: If you want to replace the horizontal (time) variable,
  no plot selection is required; simply click left on the variable you
  want to choose.  To replace a vertical variable, first select the
  plot you want to change, and then click right or middle on the
  variable you want.  (For now, ggobi won't let you use a variable
  that's already plotted.)
\item Insert:  Select a plot in the display, and then click
  middle or right on a variable checkbox to add a new plot before
  the current plot.  (This applies to vertically plotted
  variables only.)
\item Append:  Click on one of the variable checkboxes to append
  a new plot after all other plots.  (GGobi will not add a variable
  that's already plotted.)
\item Delete: No plot selection is required; just click middle or
  right on the variable checkbox for the variable you want to
  delete.  (This applies to vertically plotted variables only.)
\end{itemize}

\subsubsection{Missing values displays}
\label{slbl:MissingData}

When your data includes missing values, two additional display types
are available.  The missing values scatterplot and scatterplot matrix
display not the data itself but the ``missingness'' for each point, namely
a jittered scatterplot of indicator variables encoding the presence or
absence of values for each variable.  Examining a number of views in one
of these displays allows us to explore the joint distribution of missing
values across variables.  Since these displays are linked to all others,
brushing the missings in a missing values display allows us to explore
the association between missing values and the variables.

\section{Data format}
\label{slbl:DataFormat}

\subsection {ASCII}
\label{slbl:ASCII}

The ascii data format used in XGobi are still
supported, with some changes and some exclusions.

The only essential file is the one containing the data itself.  Each
line in the file contains one row of the input data matrix, and lines
must be separated by carriage returns.  Columns, or variables, can be
separated by any number of tabs or spaces.  The file needs to have the
suffix {\em .dat}.

You can supply variable and case labels in associated files.
Variable (column) labels can be in a file named {\bf datafile.col}
(or {\bf datafile.column}, {\bf datafile.collab}, or {\bf
datafile.var}).  Case (row) labels can be supplied in a file named
{\bf datafile.row} (or {\bf data\-file.row\-lab} or {\bf
data\-file.case}).  There should be one label per line, and the label
can include blanks.  If variable or case labels are not supplied,
default labels using column or row numbers are used.

The files {\bf datafile.glyphs} and {\bf datafile.colors} can be
used to set the plotting characters and colors to be used in
drawing each point.  Glyphs can be specified in two ways:

\begin{itemize}
\item with a string for glyph type and a number for glyph size, where the
string is one of ``.'' (a single-pixel point), ``plus'', ``x'', ``or''
(open rectangle), ``fr'' (filled rectangle), ``oc'' (open circle), ``fc''
(filled circle),
 and the number is between 0 and 7, or
\item with one number per line, where the number is between one and
49.  Here's how to generate that number: the type is between 1 and 7,
using the ordering just presented, and the size is between 1 and 8
(though it must be 1 for the single-pixel glyph).  The number is then
$8 \times (type-1) + size$.
\end{itemize}

Filled circles may be the most visually appealing glyph, but take
much longer to draw under Microsoft Windows.  Consequently, our
sample data files usually use a small filled rectangle as the default
glyph.

Colors must be integers from 0 to the number of colors in
the colormap - 1.  (Currently, only one colormap is used, and it
contains 10 colors.)

\subsection {XML}
\label{slbl:XML}

The XML format is described in {\em Using XML Input Formats},
available from the web site.  The XML format allows more detailed
specification, such as

\begin{itemize}
\item multiple datasets within a single XML file, all available within
      a single GGobi process,
\item rules for linking between datasets, and
\item edges: line segments connecting pairs of points.
% Let's leave this out for now ...
% \item the colormap to be used in brushing.
\end{itemize}

\subsection {Database access}
\label{slbl:MySQL}

It is possible to interface ggobi with either a Postgres or MySQL
database. Details of this are described in the document {\em GGobi and
  Database Management Systems}, available through the web site. 

\section{Integration of ggobi with other software systems}
\label{slbl:Integration}

The application programming interface (API) makes it possible to
access ggobi from other software. The main example that currently
exists is the integration of ggobi in R. This is described in the
document {\em Using the R-GGobi Link} available through the web site.
Other examples include the GGobi plugin for \textit{Gnumeric}, and the
Python and Perl modules for GGobi.


\section{View modes: projection and interaction}
\label{slbl:ViewModes}

Selecting any mode on the {\bf ViewMode} menu changes the interactions
available for the display and plot that are current, and pops the
corresponding control panel into the left portion of the console
window.  The modes in the top half of the menu do something more as
well:  they set the projection method for the display, and the meaning
of actions in the variable selection panel always conforms to the current
projection method.

As an example, start GGobi with some data, and watch what happens in
the main GGobi window and a a single scatterplot display.   When
GGobi starts, it's in {\bf XYPlot} mode by default, so the
scatterplot window shows a 2-dimensional projection of the data.  If
you select {\bf Scale} or {\bf Brush} (or any choice in the bottom
half of the menu), the control panel at the left of the main window
changes, reflecting the different interactions available to you in
each mode.  However, the projection in the window doesn't change.
(There may be cues added to the plot to tell you the view mode of the
plot.) If you click on the checkboxes in the variable selection panel
at the right of the GGobi console, you replace one of the plotted
variables.

Now select {\em 2D Tour} in the ViewMode menu.  {\em Everything}
changes at once, because you've effectively selected both a new mode and
a new projection type at the same time.
\begin{itemize}
\itemsep 0em
\item The panel at the left of the console changes, because a
      new set of interactions just became available.
\item The variable selection panel changes, because the variable
      selection behavior for high-dimensional projection types is
      quite different than that for low-dimensional projection types.
\item The plot in the scatterplot display changes, because it's
      now showing a projection of 3 variables instead of 2.  Furthermore,
      it's moving, because a grand tour process is running.
\end{itemize}

If you now select one of the modes in the bottom half of the ViewMode
menu, you'll see again that the variable selection panel doesn't
change, and the plot in the scatterplot doesn't change -- except
that it stops moving.  The only thing that changes with every
selection is the panel at the left of the console.

Now we'll describe each view mode in more detail, starting at the
top of the ViewMode menu. 

\subsection{1D plots}

The 1D plot can be displayed in two ways:  as a textured dot plot
or an average shifted histogram.

The textured dot plot uses a method described in
\cite{TukeyTukey90}.  This method spreads the data laterally by
amounts that are partly constrained and partly random, resulting in a
fairly smooth spreading of the points and minimizing artifacts of the
plotting method, such as stripes, clusters, or gaps.

The average shifted histogram is due to Scott (\cite{Scott92}),
and the code is also his.  In this method, several histograms are
calculated using the same bin width but different origins, and the
averaged results are plotted.  The smoothing parameter essentially
controls the bin width.

The 1D plot will be arranged horizontally if you select a variable
with a left click, and vertically if you use a middle or right click.

To activate this view mode from the keyboard, type {\em d} or {\em D}
with the focus in the plot window, or type Control-{\em d} or
Control-{\em D} with the focus in the console.

\subsection{XY plots}

The XY plots are the rudimentary 2 variable scatterplot (or
draughtsman plot) displays. Two variables are chosen, one to be
plotted horizontally and the other vertically. Controls allow for
variables to be cycled in and out of the plot, generating an animation
through all pairwise plots.

To activate this view mode from the keyboard, type {\em x} or {\em X}
with the focus in the plot window, or type Control-{\em x} or
Control-{\em X} with the focus in the console.

\subsection{1D Tour}
\label{slbl:1DTour}

The 1D tour generates a continuous sequence of 1-D projections of the
active variable space. The projected data are displayed as an average
shifted histogram (ASH), horizontally or vertically. The scrollbar at the
top of the controls allows the speed of rotation to be adjusted. The
pause checkbox stops and starts the tour. {\bf Reinit} initializes the tour
to the projection (the ASH) of the first active variable.  {\bf Scramble} sets
the view to a random projection.

Variables can be toggled into and out of the tour by clicking on the
variable circles. The active variable space is the subset of variables
currently selected, and their variable circles are drawn with
a bold outline. When a variable is toggled out of the tour it fades out
gradually, to maintain continuity of motion.

The variable projection coefficients can be manually manipulated
using manual controls. To select the variable to manipulate, click on
the purple {\bf Manip} button and then click on the variable circle.
Horizontal mouse motions in the plot window then alter the coefficient for
the manipulation variable, constrained by the values of the coefficients
of other active variables (which may also change).

The variable axes can be toggled on or off using the options menu at
the top of the plot window.

To activate this view mode from the keyboard, type {\em t} or {\em T}
with the focus in the plot window, or type Control-{\em t} or
Control-{\em T} with the focus in the console.

\subsubsection{Projection Pursuit}

A rudimentary guided tour is available when the {\bf Projection Pursuit}
button is selected. It is controlled through a separate pop-up projection
pursuit window, which contains a plot of the projection pursuit index.
When {\bf Optimize} is selected, the tour is guided by the index rather
than proceeding randomly.  The numbers displayed to the right of the {\bf
PP index} label are the minimum, current value, and maximum of the index.
(Currently only the principal component index is implemented.)

The {\bf Options} menu on the main menu bar contains controls for
laying out variable circles in the variable selection panel in
different ways, and also for toggling variable fading on or
off. Variable fading means a variable smoothly fades out when it is
de-selected. The alternative is to zero the variable out of view
immediately, which creates a discontinuity in the tour motion, but is
desirable for some situations.

\subsection{2D Tour}
\label{slbl:2DTour}

The 2D tour generates a continuous sequence of 2-D projections of the
active variable space. The projected data are displayed as a
scatterplot. The scrollbar at the top of the controls allows the speed
of rotation to be adjusted. The pause checkbox stops and starts the
tour. {\bf Reinit} initializes the tour to the projection of the first two
active variables. {\bf Scramble} sets the view to a random projection.

Variables can be toggled into and out of the tour by clicking on the
variable circles. The active variable space is the subset of variables
currently entered into a tour, and their variable circles are drawn with
a bold outline. When a variable is toggled out of the tour it fades out
gradually, to maintain continuity of motion.

The variable projection coefficients can be manually manipulated using
manual controls. To select the variable to manipulate, click on the
purple {\bf Manip} button and then click on the variable circle. 
Once a manipulation mode has been selected,
horizontal mouse motions in the plot window alter the coefficient for
the manipulation variable, constrained by the values of the coefficients
of other active variables (which may also change).

There are 5 manipulation modes: {\it oblique} allows unconstrained
manipulation, {\it horizontal} and {\it vertical} constrain manipulation
along the axes, {\it radial} constrains manipulation to the current
direction of the variable keeping angle fixed, and {\it angular}
manipulation allows rotating the variable axis in the plane of the plot
window, keeping the length of the axis fixed.

The variable axes can be toggled on or off using the options menu at
the top of the plot window.

To activate this view mode from the keyboard, type {\em g} or {\em G}
with the focus in the plot window, or type Control-{\em g} or
Control-{\em G} with the focus in the console.

\subsubsection{Projection Pursuit}

A rudimentary guided tour is available from the projection pursuit control
panel. Projection pursuit in the 2D tour requires the variables to be
sphered ahead of time, so the user needs to first enter the {\bf Tools}
menu and use the {\bf Sphering...} tool to clone sphered counterparts of
the currently active variables. Once these variables have been selected
and all others deselected, then the {\bf Projection Pursuit} option can
be chosen. This pops up a new window which provides an interface for
guiding the tour.
(Currently only the holes and central mass indices are implemented. )


The {\bf Options} menu on the main menu bar contains controls for
laying out variable circles in the variable selection panel in
different ways, and also for toggling variable fading on or
off. Variable fading means a variable smoothly fades out when it is
de-selected. The alternative is to zero the variable out of view
immediately, which creates a discontinuity in the tour motion, but is
desirable for some situations.

\subsection{2x1D Tour}
\label{slbl:2x1DTour}

The 1x1D tour generates 2 independent continuous sequences of 1D
projections of 2 active variable spaces, plotting the results
horizontally and vertically generating a scatterplot. The scrollbar at
the top of the controls allows the speed of rotation to be
adjusted. The pause checkbox stops and starts the tour. {\bf Reinit}
initializes the tour to the projection of the first two active
variables. {\bf Scramble} sets the view to a random projection.

Variables can be toggled into the tour by clicking on the variable
circles. A click with the left mouse toggles a variable in the horizontal
direction, and a click with the middle mouse toggles a variable in
the vertical direction. The active variable space is the subset of
variables currently selected, and their variables circles are drawn with
a bold outline. When a variable is toggled out of the tour it fades out
gradually, to maintain continuity of motion.

The variable projection coefficients can be manually manipulated using
manual controls. To select the variable to manipulate, click on the purple
{\bf Manip} button and then click left or right on the variable circle.
Once a manipulation mode has been selected, mouse motions in the plot
window alter the coefficients for the manipulation variable or variables,
constrained by the values of the coefficients of other active variables
(which may also change).

There are 4 manipulation modes: {\it combined} changes both horizontal
and vertical manipulation variable coefficients, {\it equal combined}
constrains the horizontal and vertical changes to be equal, {\it
horizontal} and {\it vertical} constrain manipulation in the corresponding
direction.

The {\bf Options} menu on the main menu bar contains controls for
laying out variable circles in the variable selection panel in
different ways, and also for toggling variable fading on or
off. Variable fading means a variable smoothly fades out when it is
de-selected. The alternative is to zero the variable out of view
immediately, which creates a discontinuity in the tour motion, but is
desirable for some situations.

The variable axes can be toggled on or off using the options menu at
the top of the plot window.

To activate this view mode from the keyboard, type {\em c} or {\em C}
with the focus in the plot window, or type Control-{\em c} or
Control-{\em C} with the focus in the console.

\subsection{Scaling of axes}
\label{slbl:Scaling}

The two styles of interaction, {\bf Drag} and {\bf Click}, are quite
different.  Drag-style scaling is a perfect example of a direct
manipulation interface, in which the points follow cursor motion in a
very simple way.  However, if you're looking at a lot of data, the
points may sometimes lag behind the cursor motion, making the degree
of panning or zooming hard to control.  There's also no way to
hold the aspect ratio fixed with drag-style scaling.

Click-style scaling may take you a few minutes to get used to, but
you'll find that it gives you very precise control and is especially
useful when you have a lot of data.

To activate this view mode from the keyboard, type {\em s} or {\em S}
with the focus in the plot window, or type Control-{\em s} or
Control-{\em S} with the focus in the console.

\subsubsection{Drag}

For the default setting, Drag, the actions of the mouse can be
described in terms of a camera:  you're operating a camera and
looking at a projection of the data in the viewfinder.  When you use
the left button, the camera is panning freely around, following the
mouse exactly.  When you use the middle or right button, you're
zooming the camera in and out, and changing the aspect ratio of
the plot.

\subsubsection{Click}

When you select the Click interaction style, the manipulation is
not so direct, but your control of the panning and zooming becomes
more precise.

With {\bf Pan} selected, the mouse controls the endpoint of a line
segment which is anchored at the center of the plot (just where the
center of the crosshair is in Drag style).  When you press the {\bf
Space} bar, you'll cause the plot to pan so that the endpoint becomes
the new center -- ie, short segments yield small movements.  Repeated
presses repeat the motion in the same direction -- convenient for
browsing time-dependent data, for instance.

With {\bf Zoom} selected, the visual guide changes again: this time,
the mouse controls a rectangle, and two keys are used: {\boldmath $<$}
to zoom in and {\boldmath $>$} to zoom out an amount inversely
proportional to the size of the rectangle -- that is, large rectangles
yield small movements.

\subsubsection{Reset}

To reset the plot, use the menu marked Reset in the main menubar:
it has two entries, allowing pan and zoom to be reset separately.

\subsubsection{Click: pan and zoom options}

By the default, the panning and zooming of the plot is unconstrained,
moving or rescaling vertically and horizontally with each action.
The pan and zoom options allow it to be constrained so that only
one axis is affected, convenient for browsing one variable at a time.

\subsection{Brush: brushing of points and edges}
\label{slbl:Brush}

Brushing is often performed when only a single display is visible,
but it is most interesting and useful to perform brushing with more
than one linked display showing different views of the same data.  

To interactively paint points, drag left to move the ``brush'' within
the plotting window, or drag middle to change the size or shape of
the brush while you paint.  (If you lose the brush by pulling it
outside the plotting window, you can grab it again if you press the
left or middle button while the cursor is inside the display window.)

To activate this view mode from the keyboard, type {\em b} or {\em B}
with the focus in the plot window, or type Control-{\em b} or
Control-{\em B} with the focus in the console.

\subsubsection{Brush on}
%
When the {\bf Brush on} checkbox is checked, moving the brush
over a plotted point causes that point to change its color or
plotting character (called a glyph).  If the brush is turned off, the
brush can be freely moved across the plotting window and it does not
change the points.

This is useful if you are plotting a very large number of points,
and you want to position the brush before painting, because you
can move it much more quickly across the plot.  

\subsubsection{Point and edge brushing}
%
If {\bf Point brushing} is in any state but {\it Off}, the brush has a
rectangular outline.  As the brush is moved across the points, any
points contained by the brush are affected.  You may be changing the
color, glyph shape, glyph size, or the visible state of the point,
depending on the menu setting.

Similarly, if {\bf Edge brushing} is in any state but {\it Off}, the
brush includes a crosshair, and as the brush is moved in the window,
any edges (line segments) intersecting either the vertical or
horizontal ``hair'' are affected.  You may be changing the color,
line type, line thickness, or visible state of the edge.

If both {\bf Point brushing} and {\bf Edge brushing} are on, the
brush is drawn as a crosshair inside a rectangle, and both point and
edge brushing are performed.

\subsubsection{Brushing modes:  persistent, transient}
%
There are two brushing modes.
\medskip
\noindent
\\{\bf Persistent}:  When you brush a point or edge, it retains its new
  characteristics when the brush has moved away.
\\{\bf Transient}:  As the brush moves off a point or edge, it returns
  to the characteristics it had before it was brushed.

\subsubsection{Undo}
%
Clicking on the {\bf Undo} button restores the characteristics
of all points and edges painted between the last mouse-down and
mouse-up.

\subsubsection{Choose color \& glyph ...}

Clicking on this button opens the {\bf Choose color \& glyph} panel,
which can be used to choose the point color and glyph as well as the
edge type for brushing.  At the top of the panel, there is a table of
all possible point glyphs and another table of all possible edge types.
Clicking on a point glyph sets both the glyph and edge type.

Below those tables is a row of rectangles of color which represent the
current color scale.  Clicking on one of these sets the brushing color.
Double-clicking on one of these rectangles, or on the two rectangles
just below them for the background and accent colors, opens a color
selection widget with access to the full color map.  The {\bf Reverse
video} button allows you to swap the background and accent colors.

\subsubsection{Link by ID or by variable}
\label{LinkingRules}

This option menu is used to define which of two linking rules is to
be used.  The default rule, {\it Link by ID}, dictates that points
representing records that have the same {\it id}, as specified in the XML
description (or in the API), will respond identically to brushing events.
Ids are unique within a dataset, so this rule has no effect when only
a single dataset is being studied.

The second rule, {\it Link by variable}, uses the levels of a categorical
variable to link points.  To choose a variable, open the {\bf variable
manipulation} tool, and select a categorical variable.  Now when a
case is brushed in one display, all cases with the same value of the
categorical variable will change accordingly, in this and all other
displays.  For example, a categorical variable may classify cities into
two types, ``coastal'' and ``inland''.  If a coastal city is pointed at
in one display, all coastal cities in this and all other displays will
be highlighted.  In this linking mode, displays of different datasets
can be linked if they have a chosen categorical variable in common.

\subsubsection{Color schemes ...}

See the description in section \ref{slbl:ColorSchemes}.

\subsubsection{Color \& glyph groups ...}

See the description in section \ref{slbl:ColorAndGlyphGroups}.

\subsubsection{Options menu}
%
When the brushing mode is active, the Options menu in the main
menu bar includes this item:
\medskip
\noindent
\\{\bf Update brushing continuously}: Update linked
  brushing with every mouse motion.  The alternative is to update linked
  views only when the mouse is released, which is more efficient when
  there are a great many points in the plot, or a great many plots on
  the screen.

\subsubsection{Reset menu}
%
When the brushing mode is active, the Reset Menu in the main
menu bar contains these items:

\begin{tabbing}
 {\bf Show all points} \hspace{.5in} \= Make all points visible. \\
 {\bf Show all edges} \> Make all edges visible. \\
 {\bf Reset brush size} \> Reset the brush to its default size and position. \\
%{\bf Reset point colors} \> Restore all points to the default color. \\
%{\bf Reset glyphs} \> Restore all points to the default glyph. \\
%{\bf Reset edge colors} \> Restore all edges to the default color. \\
\end{tabbing}
%

\subsection{Identification}
\label{slbl:Identify}

This mode is used to display labels near points in the plotting window.
To see these labels, simply move the cursor inside the plotting window.
The label of the point nearest the cursor is displayed.

By default, the labels are taken from the case (record) labels supplied
by the user either in ascii or in XML; if such a label is not present,
the row number of each case is used.   As an alternative, you can also
display the {\it point coordinates}: the values of the variables in the
current projection.

Identification in one window is instantly reflected in all linked windows.
[Some thought is required before deciding how or whether this view mode
should reflect the linking rules used in brushing.]

To cause a label to become ``sticky,'' click left when the target
label is displayed.  The printing style changes and the label now
remains printed as the cursor moves off, and even remains printed as
you leave the {\bf Identify} mode.  It is possible to rescale or
rotate data, and the sticky labels will continue to be displayed next
to their associated points.

To cause a label to become ``unsticky,'' return to the {\bf Identify}
mode and click left again when the target point is nearest the
cursor.  It is also possible to restore all labels to unsticky status
by clicking on the {\bf Remove labels} button.  You can also see all
the labels at once by clicking on {\bf Make all sticky}.

To activate this view mode from the keyboard, type {\em i} or {\em I}
with the focus in the plot window, or type Control-{\em i} or
Control-{\em I} with the focus in the console.

\subsection{Move Points}
\label{slbl:MovePoints}

In this mode, points or groups of points can be moved.  Move
the cursor in the window until it's nearest to the point you
want to move, then press any mouse button and drag until the
point is where you want.

The {\bf Direction of motion} menu allows the movement to be
constrained.  If the {\bf Move brush group} checkbox is
checked, then all points with the same glyph and color as
the selected point will be moved with it.

The {\bf Undo last} and {\bf Reset all} buttons allow movement
to be reversed.

To activate this view mode from the keyboard, type {\em m} or {\em M}
with the focus in the plot window, or type Control-{\em m} or
Control-{\em M} with the focus in the console.

\newpage
\section{Tools}
\label{Tools}

\subsection{Variable manipulation tool}
\label{slbl:VarManip}

This powerful tool is opened by selecting the first entry on the
{\bf Tools} menu.  It has several important functions:
\begin{itemize}
\itemsep 0em
\item the display of variable statistics,
\item variable subset selection for launching multi-plot displays,
\item setting variable ranges,
\item cloning variables, and
\item adding other new variables.
\end{itemize}

Its first purpose is to report information about each variable:
whether the variable is categorical (and if so, the number of levels
it has), the current variable transformation (if any); the minimum,
maximum, mean, and median of the raw data; the number of missing values
per variable.  (Note: we plan to extend the information supplied about
categorical variables so that we also show the names of the levels
and their corresponding numerical values.)

You can also use it to select variables -- not for plotting, but to
specify subsets when launching a parallel coordinates or scatterplot
matrix display.  Variables are selected by highlighting rows, and the
control and shift keys are modifiers that allow multiple rows to be
highlighted.

These selected variables will also respond to operations contained
within this panel:  you can reset the variable ranges that are used
for projecting the data into the plotting window.  This allows
variables with the same units or potential range (such as percentages)
to use the same range, and facilitates visual comparisons.

The selected variables can also be cloned, and the new variables
you create will be added to the table as well as the main control panel.

\label{NewVariable}
There's another way to add new variables, and that relies on the {\em New
...} button, which brings up a small panel.  Use that panel to specify
the variable's name and to set its values:  either the row numbers or
a set of integers reflecting the assignment of a group identifier to each
combination of point color and glyph.

\subsection{Variable transformation tool}
\label{slbl:VarTransform}

The first step in variable transformation is to specify the variables
you want to transform.

There are three stages in the transformation pipeline, with a
transformation function in each stage operating on the output of the
previous stage.  It's equally acceptable to use any or all of them.

You can think of stage 0 as a domain adjustment stage:  if a variable
has negative values, for instance, many transformation functions
can't be applied to it, so you may need to add an increment to each
value.

Stage 1 transformations include the Box-Cox family of linear
transformations \( T(X)~=~(X ^ \lambda - 1) / \lambda\) \cite{BoxCox64},
and you can either type the Box-Cox parameter into
the text box and hit return, or use the spin button to gradually
increase or decrease the parameter. 

Many of the stage 2
transformations are not linear; they include sorting and ranking.

% di: have a look at this, please? %
\subsection{Sphering}
\label{slbl:Sphering}

To sphere one or more variables, first select the variables in the list
at the top of the window, then click on {\bf Update scree plot}.

It's common to standardize the variables before sphering, that is, use
the correlation matrix instead of the variance-covariance matrix. The
check box {\bf Use correlation matrix} allows for this option.

Now you're ready to sphere the selected variables.  Working your way down
the panel, use your visual interpretation of the scree plot together
with the information in the labeled section ``Prepare to sphere'' to
decide how many principal components you want to create.  By default,
all the selected variables will be sphered, but you decide that the first
few principal components account for a sufficiently high proportion of
the variance.  In that case, you can use the spin button to the right
of the label ``Set number of PCs'' to decrease the number of principal
components you're going to generate.  The variance and condition number
are displayed to help you make that choice.

Once you're satisfied with the selected variables and the number of
principal components, proceed to the last step.  Click on {\em Apply
sphering} to create new variables and add them GGobi's variable selection
panels.  The names of the selected variables will be added to the
``sphered variables'' to help you remember which variables you sphered.

\subsection{Jittering}

Select {\em Variable jittering ...} to open a panel that allows
random noise to be added to selected variables.  This ameliorates
overplotting in scatterplots of data with many ties.

First specify the variables you wish to jitter.  Choose between uniform
and normal random jitter, and then set the degree of jitter using the
slider.  To rejitter without changing the degree of jitter, simply click
on the {\bf Jitter} button.

\subsection{Color schemes}
\label{slbl:ColorSchemes}

The {\bf Color schemes} tool has two purposes: to select a
new color scheme, and to automatically color points
using the current color scheme.

\subsubsection{Choosing a color scheme}

Preview a new color scheme by selecting from the menu.  If you would
like to apply it, click on the button ``Apply color scheme to brushing
colors.''  That replaces the current set of colors visible in the
``Choose color \& glyph'' menu and used in all the displays.

If you are currently using $n$ colors and the color scheme
you have selected
has fewer than $n$ colors, you'll be prohibited from applying
the new scheme.  If you want to use that scheme, you must either
use brushing to reduce of colors in use.

The menu of color schemes is populated using a file specified in your
.ggobirc file.  Here is a minimal .ggobirc file, in which the relevant
line specifies the file {\tt colorschemes.xml}, which is included in the
GGobi distribution.  (You will probably need to modify the path name,
of course.)

\begin{quote}
<?xml version="1.0"?>
<!DOCTYPE ggobirc>
<ggobirc>
<preferences>
  <colorschemes file="/usr/ggobi/data/colorschemes.xml" />
</preferences>
</ggobirc>
\end{quote}

The sample file presently contains 20 or so color schemes, of
different types and sizes, and they're based on the work of
Brewer (\cite{Brewer99}, www.personal.psu.edu/cab38).
The four types represented are

\begin{itemize}
\item diverging: used when the range of the coloring variable has
      a meaningful midpoint;
\item sequential: used to highlight a continuous progression of values;
\item qualitative: used when the coloring variable is categorical;
\item spectral: Brewer's modifications of the popular spectral scale
      to reduce its drawbacks.  She has made the perceptual steps
      more uniform, and made the scales more friendly for people with
      color impairments.
\end{itemize}

\subsubsection{Applying the color scheme by variable}

To brush the data according to the values of a variable, first select
a variable in the list at the top of the tool window.  This adds two
sets of numbers to the display:  along the bottom of the display, at
the center of each stripe of color, is shown the number of points that
will be drawn with this color once {\bf Apply color scheme by variable}
is pressed.  Along the top of the display, at the boundaries between the
stripes of color, appear the values of the chosen variable that define
the boundaries between colors. 

There are two methods available for defining the bin boundaries, and
a menu for choosing between them.  The ``constant bin width'' method
simply partitions the range of the selected variable into equal-sized
sub-ranges, and maps points into those bins.  The ``constant bin
count'' method attempts to map the values into $n$ bins of equal
size.  Since it also tries to assign all equal values into the same
bin, it usually doesn't produce uniform bins if the variable has many
equal values.

To adjust the boundaries between stripes of color, grab one of the
sliders, and notice that both the values and the counts adjust as you
move it.  The displays will respond as the sliders are moved:  either
continuously, or only when you release the mouse.  If you your data set
has a large number of cases, continuous updating will lag behind the
mouse, so it is probably more effective to update only on mouse release.

Try it with the olive oil data used in the tutorial, and select the {\it
Area} variable.  It can be a real time-saver.

\subsection{Color \& glyph groups}
\label{slbl:ColorAndGlyphGroups}

The {\bf Color \& glyph groups} tool displays a table.  Each unique
symbol in the data (combination of color and glyph) occupies a row,
and for each row there are three buttons:  {\bf H} hides all points
drawn in the chosen symbol, {\bf S} shows them all, and {\bf C} 
(complement) hides
what's shown and shows what's hidden.  The three remaining columns
report the number of cases hidden, the number shown, and the sum.

The {\bf Update} button at the bottom of the window will reset the
contents of the table in case it isn't responding to changes in the
displays as you continue to brush.  The {\bf Rescale} button will cause
the views to be redrawn as if any hidden points are no longer in the
data:  that is, the points that are shown will be scaled to fill the
viewing area of the plot.

See \ref{NewVariable} to read how you can add a new variable to the
existing data which serves as an indicator for these ``clusters.''

\subsection{Subsetting}

Select {\bf Case subsetting and sampling ...} to open a panel
that allows subsets to be specified in one of five ways.

{\bf Random sample without replacement}:  Specify the number of
     cases to be in the sample.
\\{\bf Consecutive block}:  Specify the first and last row of the block.
     (The two controls in the second row control the increment used
     in the control in the first row.)
\\{\bf Every nth case}:  Specify the interval and the first row.
\\{\bf Sticky labels}:  All cases with a ``sticky'' label will
  be in the subset.  If no points have a sticky label, this
  will have no effect. (See {\em Identification} for a description
  of sticky labels.)
\\{\bf Row labels}: Type in a row label, and all cases with that
  label will be in the subset.

Select one of those five, then click on {\bf Subset} in the
bottom row of the panel.  If you want to re-include all rows, 
select the {\bf Include all} button in the bottom row.

Click the {\bf Rescale} button to rescale all plots excluding
the points not in the subset.

One purpose of subsetting is to allow the use of GGobi on data matrices
that are so large that dynamic and interactive operations begin to
become painfully slow.  By selecting a smaller subset, a user can
work on that subset at a comfortable speed of tour motion and
interaction.  Another purpose is to do graphical cross-validation:
if the feature you see is still there in repeated subsamples, there's
a good chance it's not just an artifact of visualization.

\subsection{Controls for missing data}
%
By default, missing values are assigned the value $0$, but you
may sometimes find that to be an inconvenient choice.  Use the
{\bf Missing values} panel to assign alternative numbers.

Select the variable or variables whose ``imputed'' value you
wish to change, then select a method and fill in the text window
if necessary, and click {\bf Impute.}

Below the list of variables, a notebook widget allows you to specify
the type of imputation you'd like to use.

{\bf Random imputation}: Sample from the present values for each variable
  to populate the missing values.  If you have done some brushing to
  partition the cases, you can specify that you want the sampling to be
  done using only cases brushed with the same color and glyph.
\\{\bf Fixed value}: Specify any value to use instead of the default.
\\{\bf Percentage below minimum}: Specify a value that is $x$ percent
  below the minimum value.  For example, if the variable ranges from
 $40$ to $80$, specifying 10 will assign the missings the value $40 - (10\%
 * 40) = 36$.
\\{\bf Percentage above minimum}: Specify a value that is $x$ percent
above the maximum.

Click the resale button when you want to update the view to
respond to the new values.

%
% End of tools section, so end of working through the main menubar
%

\section{Multiple datasets}

Several datasets can be open in GGobi simultaneously. The {\bf File}
menu interfaces reading in additional data sets. Data tabs corresponding
to each open data set appear above the variable selection window on
the main controls window, and in various tool windows. The rules for
linking these data sets is described in Section \ref{LinkingRules}.

\section{Edges}

A special case of the use of multiple datasets is use of edges (line
segments).  There are many reasons one would want to display line segments
in a scatterplot. 

There are several datasets with edges distributed with GGobi: {\em
algal-bloom.xml}, in which edges are used to structure the display of an
analysis of variance; {\em buckyball.xml}, data describing a graph -- a
geometrical object; {\em eies.xml}, social network data; {\em pigs.xml},
a dataset which includes several time variables; {\em prim7.xml},
in which line segments are used to illustrate structure in the
data which was found during extensive exploratory data analysis.

We specify specify line segments (edges) for GGobi by the addition of a
second dataset in an XML file (or through the API) in which each record
has tags for {\it source} and {\it destination}.  These edge datasets
can still have variables, of course, which might represent variables
measured on transactions or interactions.

A scatterplot which has corresponding edge datasets (which we might call
edge sets) will have an {\bf Edges} menu in its menubar.  If there's a
choice of edge sets, you'll see a cascading menu showing their names.
Edges can have ``arrowheads'' added, to indicate the edge's direction;
it's also possible to see the arrowheads alone.

Edges can be brushed directly, as described in \ref{slbl:Brush}.

If the records that define each edge also have variables, then
displays of the variables in those edge datasets are also possible.
A point in the scatterplot of an edge dataset corresponds to the
same record as an edge in another scatterplot.  So brushing an
edge in one is the virtually the same action as brushing a point
in another.

% dfs here %
\newpage
\section{Differences from XGobi}
\label{slbl:xgobi}

In this section, we summarize the key differences between GGobi
and XGobi for those readers who are already familiar with XGobi.

\subsection {Multiple displays}

The first thing you'll notice when you look at a GGobi display is
that the plotting window has become separated from the control
panels.  The main reason for that change is so that a GGobi process
can have multiple display windows, of the same type or of different
types.  In addition to the basic scatterplot, GGobi currently has
scatterplot matrices, parallel coordinates plots, and time series
plots.  (See \ref{slbl:GraphicalDisplays} for more detail.)

This design change has had far-reaching effects.

First, user interactions available for the simple scatterplot display
can now be made available for other display types.  In XGobi, for
instance, there is a parallel coordinates display, but it's not
possible to brush it -- in GGobi, it is.

Unfortunately, having multiple displays introduces a new source of
ambiguity: you now have to tell GGobi which display, and which plot
within a display, you want to address.  Do that by simply clicking
inside the target plot.  GGobi will draw a thick white outline around
that plot so that you can check which plot your actions will be
addressing.  The ViewMode panel in the main GGobi window should
always correspond to the state of the current display, too.  We need
more user experience before we can tell whether this approach is
satisfactory.

The basis for linking has changed, too:  since all displays are
linked by default, it's no longer necessary to run multiple processes
in order to achieve linked displays.

One of the most interesting implications of using multiple displays
is that a GGobi process is no longer restricted to a single data
set.  The XML file format makes it easy to specify two or more data
matrices in a single file, as described in section \ref{slbl:XML}.
In addition, it's possible to add data matrices using the Read button
on the File menu or using R.  (The R~-~GGobi interface will be
introduced below, and it's more fully described in section
\ref{slbl:Integration}.)

The rules for linking in XGobi had evolved into a rather complicated
hodge-podge with special handling of ``row groups,'' the
``nlinkable'' notion to exclude points from linking, and linking
points to line segments (edges).  This has been replaced with a single set
of rules that can be specified in the XML file.  See section 
\ref{LinkingRules} for details.

With multiple displays, too, we no longer have to launch a new
process to open missing value plots -- the plots of 1's and 0's
which represent the presence and absence of data in each cell.

A convenient side effect of multiple displays is that, since each
display now sits in a window of its own, it's now very simple to
adjust the aspect ratio of a plot; this simple operation is very
awkward in XGobi.

\subsection {Data format}

Before we describe other key changes in the visible design, we'll
introduce the changes in data format.  The XGobi format, in which a
set of files with a common base name is used, is still supported,
though some file formats have changed ({\em .colors}), and some are no
longer supported ({\em .bin}, {\em .vgroups}, {\em .lines}).  (See
section \ref{slbl:DataFormat} for more detail.) The functions served
by the discarded file types are usually served now in a different
way:  for example, XGobi uses the {\em .vgroups} file to force
a group of variables to use the same axis ranges.  In GGobi, that's
accomplished by setting limits in the variable manipulation tool.

The new format, in which all the data lives in one file, is written
in XML.  XML (Extensible Markup Language) is a widely used language
for specifying structured documents and data to be viewed and
exchanged.  It was initially intended to be read by browsers, but it
is also used to define documents that are read by other software.
XML files can be validated automatically, and XML specifications can
be easily extended, too, by adding to the set of tags in use.

As GGobi grows and develops, we will favor the XML format:  it
will become increasingly difficult to keep the old format up to
date.  For instance, it's no longer possible to specify edges
in the ascii data format, but they can be specified in XML -- and
they can have associated data values.

A few XML data files are included in the sample GGobi data, and
the details of their format is described in {\em XML Input Format}
(XML.pdf) which is included as part of the GGobi distribution.

Another innovation in data access is the ability to read data
directly from a MySQL database, as described in section \ref{slbl:MySQL}.

\subsection{Integration}

Many XGobi users are also users of the SPlus or R statistics software, and
have used the S function which launches an XGobi process viewing S data.
Once that launch occurred, the resulting process was utterly independent
of its parent.  The XGobi authors did some experiments in the early 90s
to achieve a more intimate connection, but made little headway with S,
though interprocess communication was used successfully with ArcView
(\cite{SwayneBujaHubbell91,SMCM97}).

With GGobi, that problem has been solved, and it's now possible
to have real-time integration between GGobi and a variety of
other software environments.  An example of this integration is
the embedding of GGobi into R (or S), with the addition of a set of
R (or S) functions that manipulate GGobi data and displays,
resetting data values, projection, and the appearance of the plot.

For more details, see section \ref{slbl:Integration}, or read
{\em Using the R-GGobi Link} (in RGGobi.pdf), another piece of
documentation included in the GGobi distribution.

\subsection {Variable selection}

There have been a few changes in variable selection.  The familiar
variable circles are still used for high-dimensional view modes, but
we've switched to a simple checkbox interface for plots where only one
or two variables can be selected simultaneously.  In these plots, the
rich feedback provided by the variable circles is not needed, and may
just be confusing to novices.

The basics of the user interface haven't changed, though:
click with the left button to select a variable to be plotted
horizontally and the middle (or right) button to plot vertically.

\subsection {The variable manipulation tool}

The table in the {\bf Variable manipulation tool} can also be used to
select variables -- not for plotting, but for specifying variables
subsets when launching a parallel coordinates or scatterplot matrix display.

This table can also be used to specify limits for variables or groups
of variables, and so we have dispensed with the {\em .vgroups}
functionality in XGobi.

Variable cloning is another new feature: it appears as a button
on the variable selection and statistics table.

For more detail on this table, see section \ref{slbl:VarManip}.

\subsection{Changes in ViewModes and Tools}

{\bf Brush} may be the view mode that has changed the most,
because GGobi has a much richer notion of color selection than
XGobi.  Open the {\bf Choose color \& glyph} panel, and then
double-click on any element of the color palette to bring up a color
selection widget with access to the full color map.  Notice that you
can change the background color as well as any of the foreground
colors, and notice that changing the glyph also changes the line
type to be used in edge brushing.

The {\bf Color schemes} tool has extended brushing considerably,
by allowing a choice of color schemes and enabling a form of
automatic brushing.  See \ref{slbl:ColorSchemes}.

{\bf Scale} has also changed.  The direct manipulation shifting and
scaling methods work as they did in XGobi, but we've added what we
call ``Click-style interaction'' for more precise control.  See section
\ref{slbl:Scaling}.

The tour methods in GGobi are still evolving: many things are not yet
implemented, but new things are beginning to appear.  In the {\bf
1DTour}, a projection of several variables is viewed as an average
shifted histogram, as described in section \ref{slbl:1DTour}.

The redesign of the tour methods is reflected in the {\bf Sphering}
tool, which also appears in the most recent versions of XGobi.
Instead of automatically sphering variables in projection pursuit,
GGobi requires you to specify which variables to sphere.  Since this
method makes use of variable cloning, it creates new variables,
allowing you to look at plots of principal components against the
original data.

See section \ref{slbl:Sphering} for details.

\subsection{On-line help}

The on-line help system used in XGobi has been replaced with
``tooltips,'' so leaving the mouse over a widget for a couple of
seconds brings up a phrase describing the function of that widget.
If the tooltips annoy you, you can turn then off using a checkbox on
the {\bf Options} menu.

\section{Known problems}

\begin{itemize}
\item The combination of linking by variable and edge brushing
      doesn't work yet.  [Or does it?]
\end{itemize}

\section{Future work}

Soon:

\begin{itemize}
\item A two-button toggle widget for variable selection
\item Categorical data displays
\item Graph layout
\end{itemize}

Longer term:

\begin{itemize}
\item Edge editing (maybe)
\item Maps
\end{itemize}

\section*{Web Links}

The web site for GGobi:

\centerline{{\tt ggobi}: {\tt www.ggobi.org}}

contains details on downloading and installing software, related
documentation and a picture gallery.

\newpage
\bibliography{manual}
%\addcontentsline{toc}{section}{References}


\end {document}

