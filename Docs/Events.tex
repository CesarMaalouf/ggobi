\documentclass{article}
\input{WebMacros}
\input{CMacros}
\usepackage{times}
\usepackage{fullpage}

\bibliographystyle{plain}

\def\SignalName#1{\textbf{"\Escape{#1}"}}
\def\GGobiPlugin#1{\textsl{\Escape{#1}}}
\title{Events in GGobi}
\author{
Duncan Temple Lang \\
Debby Swayne
}

\begin{document}
\maketitle
\begin{abstract}
  When embedding GGobi within other applications or writing extensions
  for GGobi using the plugin mechanism, it is convenient to be
  informed of certain changes to a ggobi instace.  Specifically, it is
  useful to know when a new dataset or a variable within a dataset is
  added; when a new plot is created; when a point is moved, or
  identified; or when the brushing region is moved or resized.
  We describe how to use the basic event mechanism in GGobi
  to have callbacks for these types of events.
  Examples can be found in the GGobi plugin for Gnumeric.
\end{abstract}

\section{Motivation}
When we embed GGobi in other applications, those applications may need
to know when a new plot in GGobi is created so that they can draw on
it, add it to a list of windows under its control, or simply register
event handlers for it.  Similarly, we might want to link the brushing
or point identification in GGobi to displays in the host application.

Plugins within GGobi may also want this dynamic information.  For
example, using the data grid provided by the \GGobiPlugin{DataViewer}
plugin, we might want to highlight the row in the grid corresponding
to the point that has just been identified.  Also, when a new dataset
is added to a ggobi instance, the plugin might want to menu items for
that new dataset.

Some examples can be found in the code for the GGobi plugin for
Gnumeric, specifically in \file{Gnumeric/ggobiPlugin.c}.

We do not want developers to have to add their code to GGobi to be
informed of these events.  Rather we want them to treat it as a
library and for them to be able to ``anonymously'' be informed of the
events. We do this using signals and callbacks (or listeners).  GGobi
simply broadcasts when an event occurs.  Other code will have
registered to be notified of such broadcasts and that code will be
called. These callback routines can then take any action based on the
event to update themselves.

\section{The Events}
GGobi provides a growing list of internal events that it broadcasts to
interested parties. It uses the Gtk signal mechanism to manage the
notification to ``listeners''.  The events currently supported are
described in the following list.  Currently, all callbacks have no
return value.  This means that the callbacks cannot influence what
ggobi will do. In other words, we tell people of an event but don't
ask for their opinion.

We tend to use \Croutine{gtk_signal_connect_object} to register a
callback or handler for these events.  This arranges for the user data
given in this call to be passed as the first argument to the callback.
The routine \Croutine{gtk_signal_connect} puts it as the last
argument. See \cite{pennington99} or \cite{gtkTutorial}.
The 

\begin{description}
\item[new data] The \SignalName{datad_added} signal is generated each
  time a dataset is created and added to a GGobi instance.  Callbacks
for this event are given two arguments
in addition to the user-defined data provided when
one registers the callback.
\begin{verbatim}
 void (*callback)(void *userData, datad *data, ggobid *gg);
\end{verbatim}
\item[brush movement] 
The signal name is \SignalName{brush_motion}.
This event is generated when the user moves (or
  resizes?)  the brushing region within a plot.  The callback is
  passed the ggobi instance, the splot in which the event occurred and
  a reference to the low-level event so that one can have access to
  all the information about where the mouse is, etc.
\begin{verbatim}
 void (*callback)(void *userData, ggobid *, splotd *sp, GdkEventMotion *);
\end{verbatim}

\item[point identification] The \SignalName{identify_point} event is
generated when the user is in Identify mode in GGobi, and specifically
each time a different point is identified. By this we meant that the
user is moving the mouse around and identifies a different point than
the previously identified record.  The callback is given three
arguments: the plot in which the identification took place, a
reference to the \CStruct{GdkEventMotion} event, and the GGobi
instance.
\begin{verbatim}
 void (*callback)(void *userData, splotd *sp, GdkMotionEvent *ev, ggobid *gg);
\end{verbatim}

\item[new plot] The signal name is \SignalName{splot_new}.  This is
called when a new `plot' is created.  This occurs for each plot within
a \CStruct{displayd}.  This callback has two system arguments: the
plot and the ggobi instance.  From the plot, one can access the
\CStruct{displayd}.
\begin{verbatim}
 void (*callback)(void *userData, splotd *sp, ggobid *gg);
\end{verbatim}

\item[variable added] The signal name is \SignalName{variable_added}.
This is generated when a new variable is added to a dataset in a ggobi
instance.
\begin{verbatim}
void (*callback)(void *userData, vartabled *var, ggobid *gg);
\end{verbatim}
\end{description}

\bibliography{gtk}

\end{document}
