\documentclass{article}
\usepackage{graphics}
\usepackage{fullpage}
\begin{document}

\title{Using GGobi in Classification Problems}

\author{Di Cook, Iowa State University}

\date{September 2003}

\maketitle

\section{Background}

Supervised classification, often called discriminant analysis in
Statistics, requires the knowledge of the classes, and seeks to find
the rule for best separating the groups based on the measured
variables. Unsupervised classification, commonly called cluster
analysis, involves detecting previously unknown clusters in the
data. It is often imagined that clusters are neatly separated subsets
of observations, but quite often cluster analysis results in a
partitioning of the data space.  Sometimes a data problem involves
both supervised and unsupervised classification: there are some known
groups in the data, but there may be more unknown groups that the
analyst should discover.

Much of today's multivariate data analysis involves classification
tasks. Gene expression analysis involves determining which genes are
responsible for metabolic processes. Unsupervised classification is
commonly used to partition the genes into similar expression
patterns. Supervised classification is sometimes used when different
tissue types or gene function are known. Market segmentation uses
unsupervised classification to stratify the consumers into similar
demographic or behavioral groups, simplifying the marketing of
products.  Mapping land usage across the globe commonly involves
supervised classification using ground truth information in
conjunction with remotely sensed data. Identifying species or gender
of organisms is often achieved using supervised classification to
build a classifier on specimens. Medical diagnoses may use a rule
based on supervised classification to determine the prognosis of a
patient. New equipment scanning faces at airports in the search of
terrorists uses classifcation techniques.

Whether the class information is known or not the visual tools for
classification are primarily the same. When class identity is known,
colors and/or symbols can be used to code this information into the
plots. It is also relatively easy to approach the analysis of data
which contains a mix of known and unknown class identities. This is
not typically true in numerical algorithms, where the knowledge of
group information dramatically changes the algorithm for finding the
separations between groups. There are numerous numerical technques in
common use, for example, neural networks, support vector machines,
random forests, linear/quadratic discriminant analysis, logistic
regression, self-organizing maps, fuzzy c-means clustering,
model-based cluster analysis. Good data analysis involving
classification requires both visual and numerical methods.

Graphics is very important in classification to build an understanding
of the class structure in the data, and to determine the importance of
variables for the classification. While the benchmark for new
classification methods in recent years has been purely predictive
accuracy, it is now recognized that understanding the classification
is just as important.

This chapter describes visual tools that are useful for classification
problems.  The methods described include random, manual and guided
tours, parallel coordinate plots and scatterplot matrices. There are
three data sets used: olive oils was collected for a quality control
study, prim7 is an old data set from a particle physics experiment,
and the rats data arose from a gene expression study of the central
nervous system tissue of rats.

% Interpoint distances
% do we need a notation?

\section{Purely Graphics: Getting a Picture of the Class Structure}

\subsection{Supervised Classification}

The objective here is to find differences in the measured variables
amongst the classes. Differences here might be actual separations
between classes on a variable or linear combination of variables. The
approach is relatively simple:

\begin{enumerate}
\item
Use color and or symbol to code the categorical class information into
plot.

\item  
Begin with low-dimensional plots (histogram, density plot, dot plot,
scatterplot) of the measured variables and work up to high-dimensional
plots (parallel coordinate plot, tours), exploring class structure in
relation to data space.
\end{enumerate}

% Beware usng too many colors. 
The methods are demonstrated using the olive oil data.

\noindent{\em Regions - Univariate Plots}

Using 1D Plot mode sequentially work through the variables, either by
manually sepecting variables or cycling through automatically, to
examine separations between regions. Its possible to neatly separate
the oils from southern Italy from the other two regions using just one
variable, eicosenoic acid. Figure \ref{olive-1d} displays a textured
dotplot and an ASH plot of this variable.

The oils from southern Italy are removed, and we concentrate on
separating the oils from northern Italy and Sardinia. Although a clear
separation between these two regions cannot be found using one
variable, two of the variables, oleic and linoleic acid, appear to be
important for the separation (Figure \ref{olive-1d}).

\begin{figure*}[htbp]
\centerline{{\pdfimage width 3in {Figures/olive-1da.jpg}}{\pdfimage width 3in {Figures/olive-1db.jpg}}}
\centerline{{\pdfimage width 3in {Figures/olive-1dc.jpg}}{\pdfimage width 3in {Figures/olive-1dd.jpg}}}
\caption{Separation between the 3 regions of Italian olive oil data
in univariate plots.}
\label{olive-1d}
\end{figure*}

\noindent{\em Regions - Bivariate Plots}

Starting from the two variables identified by the univariate plots as
important for separating northern Italian oils from Sardinian oils,
the remaining variables are explored in relation two these two using
scatterplots. Oleic acid and linoleic acid show some, but not cleanly,
separated regions. Arachidic acid and linoleic acid display a clear
separation between the regions, but it is a very non-linear boundary.

\begin{figure*}[htbp]
\centerline{{\pdfimage width 1.5in {Figures/olive-2da.jpg}}{\pdfimage width 1.5in {Figures/olive-2db.jpg}}{\pdfimage width 1.5in {Figures/olive-t1da.jpg}}}
\caption{Separation between the northern Italian and Sardinian oils in 
bivariate plots and multivariate plots.}
\label{olive-2d}
\end{figure*}

\noindent{\em Regions - Multivariate Plots}

Starting from the three variables found from bivariate plots to be
important for separating northern Italian and Sardinian oils we use a
higher-dimensional technique to explore them. Using either Rotation,
Tour1D or Tour2D examine the separation between the two regions in the
3-dimensional space. Figure \ref{olive-2d} shows the results of using
Tour1D on the three variables. The two regions can be separated
cleanly by a linear combination of linoleic and arachidic acid,
roughly corresponding to $0.957\times \mbox{linoleic} + 0.289\times
\mbox{arachidic}$.

\noindent{\em Areas - Northern Italy}

There are three areas in the region, Umbria, East and West
Liguria. From univariate plots there are no clear separations between
areas, although several variables, for example, linoleic acid (Figure
\ref{olive-nth}, show some differences. In bivariate plots, two variables, 
stearic and linoleic show some differences between the
areas. Examining combinations of variables in Tour1D shows that West
Liguria is almost separable from the other two areas using
palmitoleic, stearic, linoleic and arachidic acids. Umbria and East
Liguria are separable, except for one sample, in a combination of most
of the variables. To get this result, the projection pursuit controls
were used followed by manual manipulation to assess the importance of
each variable in the separation of the areas.

\begin{figure*}[htbp]
\centerline{{\pdfimage width 2in {Figures/olive-nth-1d.jpg}}{\pdfimage width 2in {Figures/olive-nth-2d.jpg}}}
\centerline{{\pdfimage width 2in {Figures/olive-nth-t1da.jpg}}{\pdfimage width 2in {Figures/olive-nth-t1db.jpg}}}
\caption{Separation in the oils from areas of northern Italy, univariate, bivariate and multivariate plots.}
\label{olive-nth}
\end{figure*}

\noindent{\em Areas - Sardinia} 

Separating the oils from the coastal and inland areas of Sardinia can
be simply done by using two variables, oleic and linoleic acid.

\noindent{\em Areas - Southern Italy}

There are four areas in the south Italy region, North and South
Apulia, Calabria, Sicily. From univariate plots and bivariate plots
there are no clear separations between all four areas. Simplify the
work by working with fewer groups, take two of the areas at a time, or
three at a time, and plot the data in univariate and bivariate
plots. The best results are obtained by removing Sicily. If the oils
from this area are removed then the remaining 3 areas are almost
separable using the variables palmitic, palmitoleic, stearic and oleic
acids. Figure \ref{olive-sth} shows plots taken from the Tour2D of
these 4 variables revealing the separations between the areas, North
and South Apulia, and Calabria. However, the fatty acid content of
oils from Sicily is confounded with those of the other areas.

\begin{figure*}[htbp]
\centerline{{\pdfimage width 1.5in {Figures/olive-stha.jpg}}{\pdfimage width 1.5in {Figures/olive-sthb.jpg}}{\pdfimage width 1.5in {Figures/olive-sthc.jpg}}}
\caption{The areas of southern Italy are mostly separable, except for Sicily. }
\label{olive-sth}
\end{figure*}

\noindent{\em Taking stock}

What we have learned from this data is that the olive oils from
different geographic regions have dramatically different fatty acid
content. The three larger geographic regions, North, South, Sardinia,
are well-separated based on eicosenoic, linoleic and arachidic acids.
The oils from areas in northern Italy are mostly separable from each
other using all the variables. The oils from the inland and coastal
areas of Sardinia have different amounts of oleic and linoleic
acids. The oils from three of the areas in southern Italy are almost
separable. And one is left with the curious content of the oils from
Sicily. Why are these oils indistinguishable from the oils of all the
other areas in the south? Is there a problem with the quality of these
samples?

\subsection{Unsupervised Classification}

With unsupervised classification the first task is to construct the
class information, that is, using some measure of distance determine
which cases should be grouped together. This section describes visual
methods to approach classification when the class identities are
unknown. We describe the brush-and-spin approach to interactive
cluster analysis. The approach will work reasonably when there are
good separations between groups even when there are marked differences
in variance structures between groups, and non-linear boundaries. It
doesn't seem to work very well when there are classes which overlap in
space, or when there are no distinct classes but rather we simply wish
to partition the data. In these situations it may be better to begin
with a numerical solution, and attempt to refine it with visual tools.

The rats gene expression data will be used to demonstrate graphics for
unsupervised classification. A parallel coordinate plot is used to
display the expression measurements from embryonic to adult rats. A
second window is used to examine the cluster algorithm results.

Figure \ref{rats1} (left plot) shows the results of Wards linkage
using the two different distance measures. The methods agree for the
most part on the major clusters, as can be seen from the strong
diagonal. Clusters 1, 2 and 3 using correlation distance are the same
as clusters 1, 2, 3 using fluoresence. There is some difference in the
group called cluster 4 by correlation distance - for fluoresence these
cases are split between clusters 2 and 4. Similarly cluster 5 using
correlation is split over two clusters by fluoresence. We examine
these results in the gene expression patterns over time.

The genes which are considered to be cluster 2 by both methods have
low expression values for the first embryonic time and gradually
increase, peaking at the last embryonic stage and birth. Most of the
genes have high expression values for the remaining developmental
stages with two exceptions. 

\begin{figure*}[htbp]
\centerline{{\pdfimage width 1in {Figures/rats1a.jpg}}{\pdfimage width 5in {Figures/rats1b.jpg}}}
\centerline{{\pdfimage width 1in {Figures/rats5a.jpg}}{\pdfimage width 5in {Figures/rats5b.jpg}}}
\caption{Examining the results of Wards linkage hierarchical cluster analysis
using two different distance metrics on the rat central nervous system.}
\label{rats1}
\end{figure*}

The genes considered to be cluster 4 by correlation but cluster 2 by
fluorescence have a similar expression pattern as the group considered
to be cluster 2 by both methods. Thet differ in that they have
consistently smaller values at the earliest stages and consistently
higher values at the adult stage.

%\section{With Numerical Methods}

%\section{Relationship with Statistical Distributions - MANOVA, 
%Hotellings T$^2$}


\section{Exercises}

\begin{enumerate}
\item
\begin{enumerate} This question uses the flea beetle data. 
\item
Generate a scatterplot matrix of the flea beetle data. Which variables
would contribute to separating the 3 species?
\item
Generate a parallel coordinate plot of the flea beetle data. Characterize 
the 3 species by the pattern of their traces.
\item
Watch the flea beetle data in a grand tour. Stop the tour when you see a 
separation and describe the variables that contribute to the separation.
\item
Using the projection pursuit guided tour, with the holes index, find a
projection which neatly separates all 3 species. Put the axes onto the
plot and explain the variables that are contributing to the
separation. Using univariate plots confirm that these variables are
important to separate species. 
\end{enumerate}
\item This question is about the Australian crabs data.
\begin{enumerate}
\item
From univariate plots assess if any individual variables are good
classifiers of species or sex.
\item From either a scatterplot matrix or pairwise plots, determine 
which pairs of variables best distinguish the species, and sexes
within species.
\item Examine the parallel coordinate plot of the 5 measured variables. 
Why isn't a parallel coordinate plot helpful to determine the 
importance of variables for this data?
\item Using Tour1D (perhaps with projection pursuit with LDA index) 
find a 1-dimensional projection which mostly separates the 
species. Report the projection coefficients.
\item Now transform the 5 measured variables into principal components 
and run Tour1D on these new variables. Is a better separation between
the species to be found?
\end{enumerate}
\item This question uses the rat gene expression data.
\begin{enumerate}
\item Explore the patterns in expression level for the functional
classes. Can you characterize the expression patterns for each class?
\item How well do the cluster analysis results match the functional
classes? Where do they differ? 
\item Could you use the cluster analysis results to refine the
classification of genes into functional classes? How would you do
this?
\end{enumerate}
\end{enumerate}

\end{document}
