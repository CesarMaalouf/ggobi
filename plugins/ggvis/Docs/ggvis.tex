\documentclass[11pt]{article}
\usepackage{fullpage}

%% Choose a different style.
\bibliographystyle{asa}

\input{WebMacros}
\input{SMacros}
\input{CMacros}
\input{XMLMacros}

\begin{document}

\title{ggvis: graph layout as a plugin in {\tt ggobi}}
\author{Deborah F. Swayne}
\date{}
\maketitle

\begin{abstract}
ggvis is a graph layout plugin for ggobi, and this manual
describes its use.  It can be expected to change and grow.
\end{abstract}

\section{The data}

First, you need some data.  The ascii format supported by ggobi is not
adequate for the specification of edges, so you'll probably use an xml
file with a minimum of two datasets:  a set of cases or nodes, and a set
of edges.  The records in the node data must have
\XMLAttribute{id}s:

\begin{verbatim}
<record id="0"> ... </record>
<record id="1"> ... </record>
\end{verbatim}

The edge records use those \XMLAttribute{id}s to specify a
\XMLAttribute{source} and \XMLAttribute{destination}:

\begin{verbatim}
<record source="0" destination="1"> ... </record>
\end{verbatim}

No two records in the same xml file may have the same record id.

The sample data that is discussed here is part of the ggobi distribution:
\file{snetwork.xml}, an artificial social network of 140 people who are
connected by 205 edges, representing some (unspecified) form of social
contact.  Notice that there are two variables recorded for each node,
and two variables for each edge.

In a ggobi scatterplot display, use the {\bf Edges} menu to
display the edges, and maybe the ``arrowheads'' which indicate
edge direction.  Note: Until the edges are displayed, neither layout
algorithm will work.

\section{Layout}

There are two layout algorithms presently implemented in ggvis,
classical multidimensional scaling (with an optional force
adjustment algorithm), and a radial layout.

Find the ``Graph layout ...'' item on ggobi's Tools menu
to open the ggvis control panel.  It contains a ``notebook''
widget with one page for each layout method.

\subsection {Network layout}

This layout uses classical multidimensional scaling (CMDS).  CMDS yields a
layout in which the sum of squares of the physical distances between nodes
is as close as possible to the sum of squares of the ``link distances,''
that is, the number of edges in the shortest path between $node_i$
to $node_j$.

Click on ``cmds.''  It adds two variables to the node data, so 
just plot ``Pos1'' against ``Pos0'' to see the resulting layout.

Layouts produced by MDS are dominated by large distances, so it
is sometimes useful to apply a few iterations of a force adjustment
algorithm to spread out the leaf nodes.  Force adjustment algorithms
are sometime called spring algorithms because they model a graph as
a physical system, a collection of weights and springs.

These algorithms can generate layouts in spaces of any dimension,
but as of this writing, our implementation only works in 2-D.

\subsection {Radial layout}

The radial layout (\cite{Wills99}) sets the first node in the dataset
as the center node, and arranges the rest of the nodes in concentric
circles around it.  The resulting layout is a tree arranged radially.
Even if the underlying graph is not very tree-like, the layout can result
in a great many edge crossings, and the layout doesn't do anything to
minimize these crossings.

Click on ``apply'' under the ``Radial'' tab.  It adds a number of
variables to the layout:

\begin{itemize}
\item x, y: The node positions.  Plot y against x to view the layout.
\item depth: For each node, the number of steps required to get from it
  to the center node. (Plots of y'' or x against depth unwrap the
  radial layout and make it look like a tree, if an unbeautiful one.)
\item in degree, out degree:  The number of edges in and out of each
  node. (This is the only place edge direction shows up.)
\item nparents: For $node_i$, the number of nodes with $depth = depth_i~-~1$
  with which it shares an edge. 
\item nchildren: For $node_i$, the number of nodes with $depth = depth_i~+~1$
  with which it shares an edge.
\item nsiblings: For $node_i$, the number of nodes with $depth = depth_i$
  with which it shares an edge.
\end{itemize}

\section{Manipulations}

All the standard ggobi direct manipulations are available.  Plots of
the graph can obviously be linked node-wise to plots of node covariates.
What might be less obvious is that they can also be linked to plots of
edge covariates, so that an edge in the graph corresponds to a node in the
plot of edge data.  Nodes can be interactively brushed and ``identified;''
edges can be brushed -- to set the color of the line type and width.
The ``color schemes'' tool can be used to automatically color the nodes
or the edges.

Nodes can also be moved, which can help untangle the layout a bit.
Groups of nodes can be moved together: first brush them with
the same symbol and color, and then select {\tt Move brush group}
in the {\tt Move points} ViewMode.

There is only one added manipulation, and that puts it an appearance
during the use of identification.  When a node is made ``sticky,'' a
set of nodes and edges are are highlighted using the current brushing
color:  the sticky node, its edges, and all nodes to which those edges
are connected.  To remove the highlighting, click again on the node to
make it unsticky.

\subsection{Thinning the graph}

Brushing can be used to thin the plot, by hiding nodes with especially
high in or out degree, for example, or nodes with large values of depth.
Once those nodes are hidden, a new radial layout can be produced.
Any nodes or edges still visible that no longer have a path to the
center node will then be hidden as well, so it isn't necessary to erase
an entire subtree: erasing its root is enough.

If you erase the center node, ggvis will refuse to proceed with
a new layout until you have restored it.

\section{Plot control from R}

If you have launched ggobi from within R, you can use the API to
drive the plot.  Here are some fragments of code in the S language
that I have used.

In the first example, I have two xml files representing two
related graphs, and I'm interested in comparing them.  This is
part of the code used to highlight the edges the two graphs
have in common.

\begin{verbatim}
g1 <- ggobi("f1.xml")
setDisplayEdges.ggobi(.gobi=g1)
e1 <- getEdges.ggobi(.data=2, .gobi=g1)
g2 <- ggobi("f2.xml")
setDisplayEdges.ggobi(.gobi=g2)
e2 <- getEdges.ggobi(.data=2, .gobi=g2)
...
esame1 <- enames1 %in% enames2
edgecolors1 <- rep(1, nedges1)
edgecolors1[esame1==T] <- 6
setColors.ggobi (edgecolors1, .data=2, .gobi=g1)
...
\end{verbatim}


In the second example, there is one graph, and its edge covariates are
the values of a variable recorded for each edge at $t_i$.  I use R to
animate the graph, using color to encode the edge weight; I first chose
a sequential colorscheme.  Similarly, one could use setGlyphs.ggobi() to
set node type or size.  The same command sets edge type or thickness
when applied to the edge data.  To hide some of the edges, use
setHiddenCases.ggobi().

\begin{verbatim}
edges <- getData.ggobi(2)
ntimesteps <- dim(edges)[2]

for (i in 1:ntimesteps) {
  tgraph (i, edges)
  ...
  colors <- integer(dim(edges)[1])
  colors[lnw>(3*mx/4)] <- 5
  colors[lnw>(mx/2) & lnw<=(3*mx/4)] <- 4
  colors[lnw>(mx/4) & lnw<=(mx/2)] <- 3
  colors[lnw>0 & lnw<=(mx/4)] <- 2
  setColors.ggobi (colors, 1:length(colors), 2)
}
\end{verbatim}

In an extension of the second example, the xml file includes three
datasets.  The third is of dimension $ntimesteps$ by $2$, and its
single time series plot represents the sum of all measurements
for all edges at each time step.  As the animation runs,
we highlight the corresponding point in a scatterplot of this time series.

\begin{verbatim}
tcolors <- integer(ntimesteps)
tcolors[1:length(tcolors)] <- 3
tcolors[i] <- 7
setColors.ggobi (tcolors, 1:length(tcolors), 3)
\end{verbatim}

\section{Future work}

We want very much to expand the set of available layouts methods.
We intend to port {\tt xgvis}, an extension to {\tt ggobi's} precedessor,
{\tt xgobi,} a rich environment for interactive multidimensional scaling.

% Bibliography: MDS reference, Graham Wills layout paper, xgvis papers

\bibliography{references}

\end{document}
