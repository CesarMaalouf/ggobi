\documentclass{article}
\usepackage{times}
\usepackage{fullpage}

\title{Implementing Plugins for GGobi with R}
\begin{document}
\maketitle

\begin{abstract}

  We describe a mechanism by which one can create plugins for GGobi, a
  direct manipulation data visualization system, using the S
  programming language, and specifically the R environment.  
  The ease of programming using S and the large collection of 
  data analysis tools available in R make this a natural
  environment for providing extensions for GGobi. This is
  complimentary to the Java plugin mechanism also supported by GGobi
  and the R-GGobi interface which allows GGobi to be embedded and 
  programmatically controlled by R users.

\end{abstract}

 Plugins for GGobi provide a powerful mechanism for providing
 functionality that is not directly in ggobi.
 It allows one to add new ways to read data, save output from GGobi
 and provide new run-time features and tools.
 While one can develop plugins in C, the programming language 
 used to develop GGobi,  it is often more convenient to use
 a higher-level interpreted language such as R or Python.
 This partly because one can typically develope code more
 rapidly using these languages than with C.
 More importantly however it is because of the range of existing
 functionality related to data analysis in a language like R, Matlab
 or Octave.


\section{R Plugins for GGobi}
There are two styles of plugins.  The most commonly used one involves
supplying the R engine within GGobi a list of function objects that
are invoked by GGobi when the plugin needs to react or provide
information to GGobi.  The functions in the collection typically share
a common state that is mutable, i.e. can change across calls to these
functions.  This is achieved using R's closure mechanism.

The second style of plugin involves providing an S object to the
R-GGobi engine and then GGobi calls specific generic functions with
that object as the primary argument.  The idea is that the developer
of the plugin will provide specific methods for the class of the
plugin object and, in this way, provides different functionality for
different plugin types. The methods can use either the S3 or newer
S4-style class mechanism.

Both input plugins and regular plugins can be implemented in this
setup. In this way, we can use R to create run-time, analysis
functionality and also new facilities for reading data to populate a
GGobi instance.

One specifies a plugin using the usual XML \XMLTag{plugin} syntax in
an initialization file for GGobi.  For an R plugin, one specifies a
\XMLAttr{language} attribute with a value \verb+"R"+.  (One should
declare the R (meta-) plugin itself before this.)  In addition to the
usual \XMLTag{description} and \XMLTag{author} information, one
provides values for the \XMLAttr{init} and \XMLAttr{create} attributes
of the \XMLTag{plugin} tag.

The \XMLAttr{init} attribute gives the name of an R script that is
\SFunction{source}'d by the R interpreter and is equivalent to the
\Croutine{onLoad} routine of other plugins.  This provides an
opportunity to perform the necessary initializations for using this
plugin class (not instance) such as loading libraries, defining
functions and methods to create and use the plugin, etc.

The \XMLAttr{create} attribute should be an S-language command that is
(parsed and) evaluated and is expected to return the plugin object
consisting of a collection of functions.

A simple example is given in the following XML stanza.
\begin{verbatim}
<plugin name="Rtest" source="plugins/R/R/testPlugin.S" 
                     command="ggobiTestPlugin()"
                     language="R">
 <description>Test of the R plugin mechanism.</description>
  <author>Duncan Temple Lang</author>
  <dependencies>
     <dependency name="R"/>
  </dependencies>
</plugin>
\end{verbatim}
When this plugin is loaded by GGobi, we source the file
\file{plugins/R/R/testPlugin.S}.  This defines a function named
\SFuncion{ggobiTestPlugin} and this is the function that is called to
create the plugin instance.

The \SFunction{ggobiTestPlugin} function defines two functions within
its body and returns them in a list.  The first element is the
\SFunction{onClose} function which is called when the plugin is no
longer needed (i.e. when it is deactivated by the user or the GGobi
instance disappears).  The second function is
\SFunction{onUpdateDisplay}, and is called by GGobi each time the
display menu is recreated. Again, these functions are direct parallels
of the C routines that are called for plugins implemented directly in
C.

We do not have an \SFunction{onCreate} as we do in C-level plugins.
The reason for this is that the command in the \SFunction{create}
attribute performs the same role, and it is in this command that one
does any initialization of the S object.


We will change this format and make the meta-plugin understand it.
Done now. onCreate and onLoad.

Need to figure out whether they are expressions, file names or
function names.


\subsection{Plugins by Function Lists}
This is not worth supporting.  Almost every plugin will need to be
able to store state somehow.  We could pass the arguments for each
call, but that is not as simple just using a closure.

\subsection{Plugins by Explicit Methods}


\section{Input Plugins}

\section{R Plugins using R GGobi}
One can also control GGobi directly from the S language using the
\SPackage{Rggobi} package.  In this case, R is in control and GGobi is
treated as a library of functions.  However, running GGobi in this
manner still allows for GGobi's plugins and even R plugins.
Specifically, we can run R, start GGobi and GGobi can create plugins
that are implemented in R.  This nesting does not cause any conceptual
problems.


\end{document}
