\documentclass{article}

\input{pstricks}

% Taken from omegahat/Docs/WebMacros.tex
\def\strip#1>{}
\def\Escape#1{\def\next{#1}%
{\frenchspacing\expandafter\strip\meaning\next}}
\def\Env#1{{\gray\texttt{\Escape{#1}}}}
\def\file#1{\HREF{#1}{\textbf{\Escape{#1}}}}
\def\dir#1{\textbf{\Escape{#1/}}}
%%%%%%%%%%%%%%%%%%%%%%%%%%%%%%%%%%%%%%%%%

% Taken from omegahat/Docs/XMLMacros.tex
\def\XMLTag#1{\textit{#1}}
\def\XMLAttribute#1{\textsl{#1}}
%%%%%%%%%%%%%%%%%%%%%%%%%%%%%%%%%%%%%%%%%

\def\directory#1{\dir{#1}}
\def\XMLElement#1{\XMLTag{#1}}

\usepackage{comment}

\usepackage{fullpage}
\usepackage[pdftex,bookmarks,colorlinks,pdfpagemode=UseOutlines]{hyperref} 
\title{The GGobi XML Input Format}
\author{\href{http://cm.bell-labs.com/stat/duncan}{Duncan Temple Lang}\\
\href{http://www.research.att.com/~dfs}{Deborah F. Swayne}}
\begin{document}
\maketitle
\section{The Advantages of XML}

GGobi's XML format allows a rich variety of data attributes and
relationships to be specified in a single file, including

\begin{itemize} \itemsep 0em
\item missing values,
\item how missing values
      are encoded (for the entire dataset or per variable),
\item how many records there are, 
\item levels of a variable that are not observed
      but known to exist (e.g. ethnicities not encountered in a survey),
\item the source of the data,
\item the type of each variable (e.g. factor or numeric), 
\item graph topology,
\item symbol types, sizes and colors (for the entire dataset or
      per observation).
\end{itemize}

% Can be validated

There is no doubt that the XML format is verbose.  However, its
copious markup and rigid structure offer many compensating benefits to
authors of the input files as well as to application programmers.  For
example, XML files can be validated externally: in other words, a
well-formed file can be tested outside of the application for which it
serves as input, which greatly helps in preparing and maintaining
correct input files.  XML parsers check whether the document is
well-formed, that all obligatory sections are present, and that all
sections are correctly placed.  Identifiers can be specified for each
row and validating parsers can check that they are unique.

To validate a ggobi data file, execute
\begin{verbatim}
  xmllint -noout -dtdvalid ggobi.dtd flea.xml
\end{verbatim}

% Familiar, supported by various tools

The growing use of XML means that its structure is now familiar to
many people, and there are editors and browsers to create and view XML
files.  Given the ability of R, S and Omegahat (and an increasing
number of other statistical applications) to read XML, the dataset can
be used in other applications with little or no additional code.

% Flexible

Additionally, it is easy to define new DTDs to represent different
inputs such as property or resource files, descriptions of plots,
layout specifications for multiple plots, graph descriptions, etc.
This can leverage much of the same parsing setup and importantly
provides a uniform and increasingly familiar interface for the user
for specifying files.

% Reads compressed files

XML offers support for reading compressed files.  The XML parser we
employ (Daniel Veillard's libxml) can parse XML directly from
compressed files with very little speed penalty.  You can try this
feature by using GNU zip (gzip) to compress the file flea.xml in the
\directory{data} directory and starting the ggobi application
\begin{verbatim}
  ggobi data/flea.xml.gz
\end{verbatim}
The parser automatically determines whether the file is compressed or
not.

% Reads via http
% No data there currently
% Note also that the XML parsing library has support for reading files
% via ftp and http.
% % No longer here; but should be somewhere
% \begin{verbatim}
%   ggobi http://www.ggobi.org/data/flea.xml
% \end{verbatim}

% File-specific defaults + record and cell-specific overrides

GGobi xml files allow one to specify default attributes (e.g., symbol
type, size and color) for all records in the file, and to override
those defaults for a single record at a time.  This greatly simplifies
experimenting with different parameter values.

% One-pass reading

The fact that the number of records and variables are specified in the
file format means that only one pass of the file is needed to read the
data.  Additionally, it is easier to handle non-rectangular data,
which may occur when data are sparse or when there is a variable number
of values per observation (e.g in medical studies).

\section{The File Format}

The format of the file is described by the DTD (Document Type
Definition) \texttt{ggobi.dtd}, though it may be easier to learn about
the file format by looking at the examples in the data directory.
Each file starts with the usual XML declarations that identify it as
XML (and its version) and the particular document type and associated
DTD.


\begin{verbatim}
<?xml version="1.0"?>
<!DOCTYPE ggobidata SYSTEM "ggobi.dtd">

<ggobidata>
\end{verbatim}
%
The string \XMLElement{ggobidata} indicates that this is the top-level
tag for the document, and this is what appears next.  To specify that
more than one dataset is included, use the \XMLAttribute{count}
attribute:
%
\begin{verbatim}
<ggobidata count="2">
\end{verbatim}
%
This tag must be terminated at the end of the file:
\begin{verbatim}
</ggobidata>
\end{verbatim}

\subsection{Data}

This is followed by the \XMLAttribute{data} tag, which begins the
entries for a dataset:

\begin{verbatim}
<data name="Flea beetles">
\end{verbatim}
%
Here you can also specify the name which will appear in the titlebars
of ggobi windows.

There can be multiple datasets within a file, and there can be two
types of relationships among their elements: 
\begin{itemize}
\item records in multiple datasets can represent different variables
  recorded for the same subject, as described in section
  \ref{LinkingRecords}, or
\item one dataset can contain a description of edges which connect
  points in another, as described in section \ref{Edges}.
\end{itemize}
Both of these schemes depend on the \XMLElement{record} \XMLAttribute{id} to
uniquely identify a record.

The remainder of the dataset is specified as sub-elements or sub-tags
within this \XMLElement{data} element.

\subsection{Color schemes}

Included in the source code is a file called colorschemes.xml,
which contains (as of this writing) 265 distinct color schemes.

To specify one that one of these schemes should be the default
for a particular set of data, specify it inside the ggobidata
element.
\begin{verbatim}
<ggobidata>
<activeColorScheme name="YlOrRd 9"/>
...
</ggobidata>
\end{verbatim}

In case you have devised your own scheme you'd like to use, 
specify it as follows:

\begin{verbatim}
<activeColorScheme file="/full/path/name/mycolorschemes.xml"
                   name="MyColorScheme 7"/>
\end{verbatim}

\begin{comment}
\subsection{Colormap}

The user can specify rows of a color table in RGB format.  (Perhaps
different formats such HSV, etc. can be supported directly by GTK
also.)  The idea is that colors for records, etc.  are specified as
row identifiers for this table.  The colormap section allows the user
to specify the values for these rows.  In this way, the data and color
references can remain fixed and one need only change the contents of
the colormap to which the correspond.

It is often convenient to use the same colormap for several
datasets. Rather than encoding the data in each input file, the XML
input file can be told to read color data from an external file.  This
is specified via the \XMLAttribute{file} and \XMLAttribute{type} The
value of the \XMLAttribute{file} attribute identifies a URL or
file. If not an absolute file name or URI, then this is located
relative to the location of the document being currently read.
That is, suppose we are running in the top-level distribution
directory of ggobi and run with the input file \file{data/flea.xml}.
Then, a reference to 
\begin{verbatim}
 <colormap file="stdColorMap.xml">
\end{verbatim}
is found as \file{data/stdColorMap.xml}.  Similarly, if a colormap
file named \file{map.xml} is located in the directory \dir{color}
parallel to \dir{data}, then
\begin{verbatim}
 <colormap file="../color/map.xml">
\end{verbatim}
would find it for the same input file \file{data/flea.xml}.

%  I don't want this to happen.  - dfs
%When an external colormap file is used in conjunction with local
%values, there are two size specifications.  One is in the colormap of
%the external file and the other is local. Both refer to the number of
%local entries, i.e. entries under their control.  By default, these
%are accumulated so that the size of the table is the sum of the two
%sizes.  This can lead to oversized tables.


External color map files can be formatted using XML (see
\file{data/stdColorMap.xml}) or as a simple rectangular array.  In the
latter case, each row should contain 3 values separated by white
space.


To differentiate between the two formats, the \XMLAttribute{type} is
used when specifying an external color map file to read.  This can be
either \texttt{xml} or \texttt{ascii}.  (See \file{ggobi.dtd})


Each entry can have an identifier attribute.  This is usually an
integer identifying the index by which one can reference this color in
records and other files.  The alternative values for the identifier
attribute are ``fg'' or ``bg''.

The identifier is really only used to override other values read from
a file.  This is due to the fact that subsequent entries without
identifiers occupy the next entry  in the color table.
In other words, the input
\begin{verbatim}
<color id="4" r=".5" g=".5" b="0" />
<color>0 0 1</color>
\end{verbatim}
sets the $5$ entry (in position $4$) to blue ($0 0 1$) since the
previous set entry was indexed explicitly as $4$.


The attribute \XMLAttribute{range} value can be specified for the entire
colormap or on a per-entry basis.  If this is present, it is
interpreted as a numeric value and each value in the entry (or all
entries if specified for the entire colormap) is divided by that
amount.  This allows one to easily use different scales such as 0 to 1
or 0 to 100, etc. This is designed to assist when the software
creating the file does not facilitate such rescaling.
\end{comment}

\subsection{Description}

The second of the sub-elements within the \XMLTag{data} tag is
a description of the dataset.

\begin{verbatim}
<description>
Physical measurements on flea beetles.
</description>
\end{verbatim}

This includes the source, any references, etc.  This is currently
free-format.  A convenient attribute is \XMLAttribute{source} which
indicates where it can be found.

\subsection{Variables}

The next section of the file contains the descriptions of
the variables.  It begins with the \XMLElement{variables} tag,
which must include the number of variables:

\begin{verbatim}
<variables count="3">
</variables
\end{verbatim}
%
Between the \XMLElement{variables} tags, the file lists the variables,
which can be continuous (real or integer) or categorical.  Continuous
real variables are specified simply:
\begin{verbatim}
<realvariable>
 <name> tars1 </name>
</realvariable>
\end{verbatim}
Integer variables use the tag \texttt{integervariable}.
% 
For categorical variables, some correspondance must be established
between level values and data values, since GGobi stores and
manipulates numbers. (These levels can also serve as a basis for
linking records during brushing as described in \ref{LinkingRecords}.)
The simplest way to specify the variable is:
%
\begin{verbatim}
  <categoricalvariable name="SEX" levels="auto" />
  <categoricalvariable name="SMOKER" levels="auto" />
\end{verbatim}
%
In that case the record values must be individually tagged as strings,
and all the rest of the values in the record should also be individually
tagged.  This is a sample record from tips.xml:
%
\begin{verbatim}
<record label="1">
 <real>1</real> <real>16.99</real> <real>1.01</real> <string>F</string>
 <string>no</string> <real>6</real> <real>1</real> <real>2</real>
</record>
\end{verbatim}

If your data values are numbers instead of strings, and all levels of
interest are present in the data, a convenient alternative is to let
ggobi assign the levels:
%
\begin{verbatim}
<categoricalvariable name="fraudp">
  <levels count="3" />
</categoricalvariable>
\end{verbatim}
%
By default, the level values in this case will be 1, 2, and 3, and
the level names will be L1, L2 and L3.  If you want level values
that begin at some number other than 1, specify the variable range: 
%
\begin{verbatim}
<categoricalvariable name="fraudp" min="0" max="2">
  <levels count="3" />
</categoricalvariable>
\end{verbatim}

These methods are appropriate when the levels of the categorical
variable have no natural ordering, and when all levels of interest are
known to be present in the data.  When you want all gaps to be filled
in, and you want complete control over the correspondance between
value and level name, a fully explicit specification can be used to
establish the correspondance between levels and values:
%
\begin{verbatim}
<categoricalvariable name="fraudp">
  <levels count="3">
    <level value="0">low</level>
    <level value="1">medium</level>
    <level value="2">high</level>
  </levels>
</categoricalvariable>
\end{verbatim}
%
For another example, see Shipman.xml.  The levels of month are
fully specified, because we want the correspondance between month
name and number to be properly established; the levels of place
and cause can be automatically assigned because no such restriction
exists.

Two other variable types exist: \texttt{countervariable}
\texttt{randomuniformvariable}.  For these two variables, no data
should be specified; they will be automatically populated when
ggobi starts, the first with integers from 1 to N, the second
with random real numbers on [0, 1].

The name of the variable can be specified as the text within the
variable tag rather than as an attribute, and an optional ``nickname''
can be added, to be used for labelling variables in tight spaces:
\begin{verbatim}
<realvariable name="tars1" nickname="t1" />
\end{verbatim}
%
(By default, the nickname is the first two letters of a variable,
and that isn't always enough to make distinctions.)

Variable ranges can be specified for real variables as well:

\begin{verbatim}
<realvariable name="Proportion"  min="0.0"  max="1.0" />
\end{verbatim}

% I suppose variable ranges can be specified for categorical
% variables as well?

%The name of the transformed variable can be specified via the
%attribute \texttt{transformedName}.

\begin{comment}
Additionally, instructions as to how to create the variable can be
specified as a programming command via the Programming Instruction (PI)
\begin{verbatim}
<realvariable>
<?R rnorm(10)>
</realvariable>
\end{verbatim}
\end{comment}

\subsection{Records}

The next section of the file is the data itself.  The individual
\texttt{record} tags are contained within the \texttt{records}
element, which must include the \texttt{count} attribute, specifying
the number of records in the data.

\begin{verbatim}
<records count="74" color="2" glyphType="3" glyphSize="3" missingValue=".">
</records>
\end{verbatim}
%
It may optionally include tags specifying the default color (the
index in the color table), glyph type and size, or the character to
interpret as a missing value.

Glyph type and size can also be specified in a single string:
\begin{verbatim}
glyph="fc 3"
\end{verbatim}
GlyphSize can range from 0 to 7; glyphType from 0 to 6, with 0
representing a single-pixel point and 6 a filled circle, and
the rest as shown in the table below.  Legal values
for the first string in the glyph tag are described in this table:

\vspace{1em}

\begin{tabular}{l|l|l}
glyphType & glyph specifier & resulting glyph \\
\hline
0 & .    & point \\
1 & plus & glyph resembling a ``+'' \\
2 & x    & glyph resembling an ``x'' \\
3 & or   & open rectangle \\
4 & fr   & filled rectangle \\
5 & oc   & open circle \\
6 & fc   & filled circle
\end{tabular}

\vspace{1em}

The body or content of each \texttt{record} may take two forms:
\begin{itemize} \itemsep 0em
\item An ASCII listing of the values, with each value
separated by white space (space character, tabs or new lines).
(See flea.xml.)
\item A fully specified listing of the values, with each value tagged
as real, int, or string.  (See tips.xml.)  For the file to be
valid, all values must be tagged if any one is.  GGobi will read the
file without complaint, but Rggobi will balk and the file will not
validate.
\end{itemize}

A record can be considered ``hidden''.  This is set via a logical
value for the attribute \texttt{hidden}.

\subsubsection{Linking Records}
\label{LinkingRecords}

Each record can be given an \XMLAttribute{id} attribute value, which
can be an arbitrary string.  This
is different from a label in that it is not used by ggobi instance
when displaying plots.  Instead, it is used only to uniquely identify
a record within a dataset, and it has two purposes.

The \XMLAttribute{id} can be used in the case where different
datasets contain different variables for the same subjects, or
for some of the same subjects.  For instance, dataset A may contain
usage data for a set of customers, while dataset B contains
demographic data for a subset of those customers.  Those datasets
will be linked for brushing and identification if they have the
same value for \XMLAttribute{id}.

The levels of a categorical variable can also be used as the basis for
linking records during brushing.  If records are linked by a
categorical variable (see the Brush control panel), then brushing one
observation causes all observations in all displays with the same
level for that variable to be painted as well.

\subsubsection{Edges}
\label{Edges}

The \XMLAttribute{id} is also a critical part of the specification
of edges.  In order to specify a set of edges, or line segments,
between pairs of points in a dataset, it's necessary to define
a second dataset whose records have \XMLAttribute{source}
and \XMLAttribute{destination} tags, like this:

\begin{verbatim}
<record source="a" destination="b"> </record>
\end{verbatim}
% 
The values used for \XMLAttribute{source} and \XMLAttribute{destination}
must correspond to the \XMLAttribute{id}s specified elsewhere.

A record which includes this edge specification can also include
any other attribute or property of a record:
\begin{verbatim}
<record source="0" destination="2" color="3"> 4.2 6 9.6 </record>
\end{verbatim}
%
If other data values are present, this dataset is like any other
dataset, and the values can be displayed in scatterplots, 
parallel coordinate plots, and so on.

% This may be deleted ...
\section{Conversion of Old Files}
The distribution contains an application named xmlConvert that can be
used to read datasets provided in the old file format to XML.  This
can be used by specifying the name of the file containing the
old-style data in the same manner as ggobi expects.
The output is written to standard output
and can be redirected to a file using basic shell commands.
For example,
\begin{verbatim}
  xmlConvert data/flea > flea.xml
\end{verbatim}
In the future, we will support writing the output to a file. (We need
to process the command line arguments and look for a -o flag).

Note that this dynamically loads the libraries libGGobi.so and
libxml.so.  Thus the directories that contain these libraries must be
referenced in the environment variable 
\Env{LD_LIBRARY_PATH}.
Alternatively, the makefile can be edited to statically link these
libraries.

\begin{comment}
\section{Compilation}
To activate the XML mechanism, define the variable
\Env{USE_XML} in  local.config.

When we use autoconf, this can be done by
\begin{verbatim}
  ./configure --with-xml
\end{verbatim}

This requires the XML parsing library libxml (also know as gnome-xml)
by Daniel Veillard.  (\texttt{Daniel.Veillard@w3.org}).  See
\Escape{http://xmlsoft.org}
\end{comment}


\section{References}
The XML Handbook, Charles F Goldfarb and Paul Prescod,
 Prentice Hall.

\vspace{1em} \parindent 0em
\url{http://www.w3.org/XML}

\end{document}
