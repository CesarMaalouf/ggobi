\documentclass{article}

\usepackage{fullpage}
\usepackage{times}

\usepackage{}

\input{WebMacros}
\def\MakeVariable#1{\Escape{#1}}
\title{Integrating \Cplus{} in GGobi}
\begin{document}
\maketitle
To read default parameters for specifying the SQL connection and query
for a dataset (see DBMS in this directory), I needed to read a
properties file similar to Java's (and more the more ubiquitous)
format
\begin{verbatim}
  name: value
\end{verbatim}
To do this, I developed some \Cplus{} code in a separate library which
will be part of Omegahat, released under the BSD or GPL license.
Now, two issues arise: a) How do we integrate
a separate library, and b) how do we integrate an
additional language for the compilation?

The difficulties associated with a)  include
\begin{itemize}
\item we may need a particular version of the library
and we cannot use any arbitrary version
because it may be incompatible.
\item if we copy the code into this source tree,
fixes made to the source tree of that library will not
be reflected here.
\item by keeping things separate, we complicate 
the distribution.
\end{itemize}
Basically, the issue is having two independent development branches.
Given that we ``know'' the owners of these libraries, we should keep
them separate and distribute them within the same package or bundle.
Changes made to a ggobi version of another library should be reported
back to that development team.  It is much the same as the libxml from
Daniel Veillard.


The reasons for wanting to introduce \Cplus{} include
\begin{enumerate}
\item It is a superset of C and more strict.  So it catches oddities
in C and at least makes one aware of them.  In a sense it is a very
strong form of gcc's -Wall argument.
\item The OOP approach simplifies certain
idioms and additionally helps to group
\item Other libraries become available (e.g. this graph library
I mentioned before and am looking for).
\item It is a useful thing to be compatible with.  Since we are
providing an API, we should allow people to access it in other
languages that can communicate with C.  \Cplus{} is the obvious and
easy one.
\item Symbol conflicts are reduced because of class names being
`mangled' into the symbol name.  Additionally, polymorphism allows us
to use the same name for routines that take different arguments. This
makes the API more natural.
\end{enumerate}

\begin{enumerate}
\item It requires an additional component for compiling from
source. On Windows this might be an issue, but supplying binaries may
be more effective anyway. Additionally, the cygwin tools should be
available for compilation (if we assume gtk) and hence egcs will be
present on the machine.

\item It complicates linking.

\item There are issues with the gtk header files.

\end{enumerate}


\section{What is Involved}
Basically, the steps involved are as follows.
\begin{itemize}
\item Declare all routines as being external
of type/mode C.
The header files simply need 

\begin{verbatim}
#ifdef __cplusplus
 extern "C" {
#endif

    ...

#ifdef __cplusplus
 }
#endif
\end{verbatim}
\item Change the makefile to link with the \Cplus{} compiler (g++, CC,
etc.)  I have introduced an LD argument which takes on the value of
the variable \MakeVariable{CC} if we are not using \Cplus{} code.  If
we are (currently if we are \MakeVariable{USE_MySQL} is defined), then
it takes the value of the \MakeVariable{CXX}
\item The purely \C{} source files need not be compiled with g++ or
CC. The linker will handle the symbol resolution correctly.
\end{itemize}
\end{document}
