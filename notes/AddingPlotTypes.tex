\documentclass{article}
\usepackage{times}
\usepackage{fullpage}
\input{pstricks}
\input{WebMacros}
\input{CMacros}
\input{GtkMacros}

\title{Adding GGobi display and plot type}
\author{Duncan Temple Lang}

\begin{document}
\maketitle

This document describes some of the issues related to introducing a
new plot type for ggobi.  Since we have designed ggobi to be
extensible in some sense (not in an object-oriented manner) and to be
embedded in to other systems such as S, Python, Perl, etc.  it is
important that the plots provide an easy and consistent API so that
they can share common facilities and simplify the task of developers
integrating into these other languages.

We are now using the (horrible) C-based class system via Gtk. This is
extremely error-prone, verbose and tedious,
but we are using it neverthless.
This is motivated by two reasons:
\begin{itemize}
\item our reliance on Gtk and the willingness to use it to its fullest
given our original use of it;
\item the desire to make the objects accessible 
 to R, Python, Perl, etc. via simple machine-generated bindings.
\end{itemize}


Extensibility is the goal here. We want to allow developers to add new
plots without needing to alter the existing GGobi source code.  This
makes maintaining such additions across different releases easier for
the developer and also simplifies testing, etc. within the central
GGobi development.

Also, when changes are made to the GGobi source, modifications to the
language bindings (e.g. Rggobi, etc.) are also needed.  This
abstraction so that there are generic methods for the new classes
should greatly simplify this and remove the need to manually
synchronize the code.

There is a difference but close connection between the concept of a
display and a plot in GGobi.  A display can be thought of as a
container for multiple related plots.  An example is the scatterplot
matrix display.  Another is the single scatterplot window.  In the
first case, the display contains a grid of plot elements.  In the
latter, there is a single plot object.  Due to the way the parallel
coordinates ``plot'' was implemented, it is a display that houses a
plot for each variable.

To complicate things further, it is possible in the
current Rggobi package to create embedded plots
that are not initially associated with any display.
They can then be located within a Gtk container
in different ways to get user-defined layouts.
This is one of the motivations for creating a
class system for plots and displays. This will 
allow us to create the objects as regular Gtk elements
and place them directly within Gtk GUIs.
And all of this can be done trivially using 
bindings for ones favorite programming language,
e.g. RGtk for R, PyGtk for Python, etc.


Let's start with the simplest case first, namely creating a new
top-level display class.  By this, we mean that it has a top-level
window.  In this case, we will want to derive our class from the
\GtkClass{GtkGGobiWindowDisplayClass}.  For safety and simplicity in
developing the extended display mechanism, we have created an
intermediate class -- \GtkClass{GtkGGobiExtendedDisplayClass} -- which
we can use to form the basis of the extended classes.  This extends
\GtkClass{GtkGGobiWindowDisplayClass} and we will extend this class.

The barchart and time series plots in \headerFile{barchartDisplay.h}
\headerFile{tsdisplay.h}
illustrate how this is done.

In the \Croutine{get_type} routine for the new Gtk class/type, we
specify a routine that is called to initialize the class structure for
this Gtk class.  In the initialization of the extended class, we set
the methods and any class-specific fields.  For example, we can
specify the string to use in the GGobi display tree to identify this
type of top-level display.  We do this by specifying a value for the
\Cfield{treeLabel}. In the unlikely case that the computation is more
dynamic and complicated than a single constant value, then we can
compute the string at run-time by calling the \Croutine{tree_label}
method in the class. Again, one specifies a value for this field in
the class in the class initialization method.

The fact that we have to access the fields from the inherited class
indirectly via the \Cfield{parent_class} field in the class structure
is one of the several frustrating aspects of the C-based class
mechanism.


\section{Registering a Display type}

The built-in but extended display types are registered at the
beggining of the process in \Croutine{GGOBI(main)}.  This is done by
passing routines that load/initialize/register the \Ctype{GtkType} and
return that value. From that, we store the class in a list.

We can handle people giving us the names of these routines,
i.e. via an R command, or via the initialization file
mechanism. Plugins should compute the \Ctype{GtkType}
directly and register this.


We iterate over these extended types when constructing the Display
menus for a GGobi instance. For these extended display types, we
construct the menus by using the value of the title label in the
extended class as the name of the display type.  We use a different
callback to create the display type, specifically
\Croutine{extended_display_open_cb}.  This uses the \Cmethod{create}
routine in the extended display class to create the new top-level
display.  That \Cmethod{create} method can do whatever it likes
to get its job done, including selecting the variables to display, etc.





Implement the \Cmethod{cpanel_make} method to create the control panel
for that display type in a \CStruct{ggobid}.  Put it in a hash table
indexed by the class or type. Or simply store it in the class itself.
The class initialization should create this object.


\section{Creating the Display}


\section{Selecting variables}

Is this a property of the display or the splot? The code seems to pass
the splot. But this might be a convenience rather than a formal
intent.  Indeed, it is a feature of the display.  The barchart is
rather special in that it (currently) only has one splot.

Variable selection can be customized by setting different signal
handlers in the control panel.  We of course allow such different
behavior to by-pass \Croutine{varsel_cb}. However, if one wants to use
the built-in mechanisms, then one need only provide a method for that
extended display to switch the variables.


\section{Examples of New Displays/Plots}
\subsection{Conditional Plot}

\subsection{Margin Plot}

\subsection{Hexagonal Bin Plot}

\subsection{Multi-Dataset Plot}
Putting multiple variables  a single, e.g., scatterplot.


\section{Control Panel}

The control panel is set for a display via a call to
\Croutine{cpanel_set} called form \Croutine{display_set_current}.  An
extended display can look at its value for the \Cstruct{cpaneld} in
its own scope (i.e. inside its \Cstruct{extendedDisplayd} structure).
If that is \Cnull, it then builds it and stores it for future use
(presumably, although it can elect to discard it when the display is
no longer the active one). (Events could be used here!)

There is the issue that display type may have more than one control
panel, each one associated with a different view mode for that
display.  For example, the time series plot supports 4 different types
of views: default, brush, scale \& identify.  For this, it maintains 3
different control panels.  So we maintain an array of these panels.
This array can be initialized in the class initialization if that
simplifies things. Alternatively it can be done on demand.

We can simplify this allocation if we want by having a field in the
class that specifies the number of modes to use.  If this is set to $>
0$, we could allocate that many!  If the number of panels is not known
until run time, the class should set this to 0 or lower, thus allowing
it to determine the value dynamically. 

 This is partially implemented but needs to get the order of the
initialization of the class, and inherited types for an instance
correct. The copy of the class info for the, e.g., time series has not
been put into the object and so the values are set to $0$.



Who sets the \literal{display->cpanel} field before
\Croutine{cpanel_set} is called?  the \Croutine{*_cpanel_init}
routines do!


Actually, we only need the widget.  And for time series, all we really
need is the option menu.  So we could actually store that directly
when we create it.

In \Croutine{viewmode_set}, we set the appropriate control panel box
as the active one for the \Cfield{viewmode_frame}.  At this point, if
we have an extended view mode, then we can ask the current display


Extended displays should handle key events themselves by registering
signal handlers when the plots are created.  See the Ctrl-v for time
series displays.

Supportin multiple new modes within an extended display requires that
display setting the viewmode in the control panel
(\literal{cpanel->viewmode}) to any value that it understands and then
decode this in the \Croutine{viewomode_control_box} method.

Added the \Cfield{current_control_panel}.



Have to allow the extended display types to store information from
the preferences/initialization file.


The time series plots seem to rely heavily on knowing where they are
in the display.  As a result, we can use methods for the display.
Alternatively, we can create a sub-class for the extended splot type
and access the display from the particular splot via the
\Cfield{displayptr} field.






\section{Serialization}

We can call methods to serialize a display.  When deserializing them,
they should have written the type of the object so that we can reload
the Gtk type and get the class and then call the deserialize
mechanism.


\section{Options}
To set the title for a display, the \Croutine{computeTitle}
looks at the display and tries to figure out
what the correct name is.
There are several ways in the current code to specify this.
\begin{description}
\item[display class]
one can create a new class that extends (or is derived from)
\GtkClass{GtkGGobiExtendedDisplay}.
Then in the class initialization routine, set the value of
the \Cfield{titleLabel} field in the
inherited \CStruct{extendedDisplayd} field.
To use this, of course, one must create an instance of
the class when creating the display.

Let's look at barchart as an example.  We can specify a value of
\CVariable{GTK_TYPE_GGOBI_BARCHART_DISPLAY} in the call to
\Croutine{gtk_type_new}. This creates an instance of
\CStruct{barchartDisplay} and the initialization in \file{ggobiClass}
sets \Cfield{titleLabel} to \literal{barchart}.


\item[Explicitly set the \Cfield{titleLabel} field in an extended
  display]
Rather than defining a new class solely for this purpose, 
one can arrange to create a regular
\GtkClass{GtkGGobiExtendedDisplay} object
and then fill in its \Cfield{titleLabel}
explicitly after it is created.
\begin{verbatim}
    display = gtk_type_new(GTK_TYPE_GGOBI_EXTENDED_DISPLAY);
    GTK_GGOBI_EXTENDED_DISPLAY(display)->titleLabel = "BarChart";
\end{verbatim}

\item[Enumerations]
We can use enumerations for the built-in display types.
\end{description}

Which approach we use depends on how many other methods, 
if any, we need for that class. If there are many, then
defining the value within a new class is appropriate.
Otherwise, setting the \Cfield{titleLabel} works well. 


Note that the \Croutine{computeTitle} now automatically handles adding
identifiers to mark the display as being current and is no longer the
responsibility of the display type to do this.






\section{Functionality}
\begin{description}
\item[name] 
\item[description]
\item[setVariable]
\item[multiple variables]
\end{description}


\section{Things to keep in mind}


Please keep the user interface material in a separate file from the
internal operations of the plot.  The GUI aspects are essentially the
view and the control elements of the Model-View-Control
pattern\cite{DesignPatterns}.

It is important that the internals be exposed via additions to the
ggobi API. This allows the developers of interfaces to other languages
easily provide connections for this plot type.

Use enumerations for a list of possible types,
e.g. the \CVariable{tsplot_arrangement}.


The redraw field in the splot class allows the plot type to indicate
if it needs to do a full redraw when the color id is reset.
This avoids conditionals in the code.

\end{document}
