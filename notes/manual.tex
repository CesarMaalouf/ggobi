\documentclass[11pt]{article}
\usepackage{fullpage}

% Issues:
%  I haven't worked out a consistent notation:  when to display
%  a button's contents in bf, when to use em, etc.

%  We need a better terminology to distinguish between the two
%  types of variable selection.  Can I call one widget the
%  variable selection panel, and the other the variable chooser?
%  Can you think of something better?

%%%%%%%%%%%%%%%%%%%%%%%%%%%%%%%%%%%%%%%%%%%%%%%%%%%%%%%%%%%%%%%%%%%%%%

% If it turns out that I can use pdflatex, I can use jpeg graphics
% files and go straight to pdf.

\newif\ifpdf
\ifx\pdfoutput\undefined
  \pdffalse
\else
  \pdftrue
\fi
  
\ifpdf
  \pdfoutput=1
  \usepackage[pdftex]{graphicx}  % uncomment if using graphicx
  %  \usepackage[pdftex]{hyperref}  % uncomment if using hyperref
\else
  \usepackage{graphicx}  % uncomment if using graphicx
  %  \usepackage{hyperref}  % uncomment if using hyperref
\fi
%%%%%%%%%%%%%%%%%%%%%%%%%%%%%%%%%%%%%%%%%%%%%%%%%%%%%%%%%%%%%%%%%%%%%%

\hyphenation{XGobi}
\hyphenation{ggobi}
\hyphenation{di-men-sion-al two--di-men-sion-al}
\input epsf
\def\Rcirc{\mbox{\small{\ooalign{\hfil\raise.07ex\hbox{\tiny R}\hfil\crcr\mathhexbox20D}}}}

\parindent 0in
\parskip 6pt

% I'm taking text from the 1997 xgobi paper wherever possible.
% (On fry, that's in /usr/dfs/papers/xgobi.1997/paper.tex)
%
%  slbl stands for 'section label'

\begin {document}

\title {ggobi manual}
\author{
Deborah F. Swayne, AT\&T Labs -- Research \\
Dianne Cook, Iowa State University \\
Andreas Buja, AT\&T Labs -- Research \\
Duncan Temple Lang, Lucent Bell Labs
}
\date{January 2001}

\maketitle


\begin{abstract}

The ggobi software is a data visualization system with state-of-the-art
interactive and dynamic methods for the manipulation of views of
data.  It represents a big step in the evolution of XGobi, with
multiple plotting windows, more flexible color management, XML file
handling, and better portability to Windows.

The most significant change is that ggobi can be embedded in other
software and controlled using an API (application programming
interface).  This design has been developed and tested in partnership
with R.  When ggobi is used with R, the result is a full marriage
between ggobi's direct manipulation graphical environment and R's
familiar extensible environment for statistical data analysis.

It has the same graphical functionality whether it is running
standalone or embedded in other software.  That functionality
includes 2-D displays of projections of points and lines in
high-dimensional spaces, as well as scatterplot matrices, parallel
coordinate and time series plots.  Projection tools include average
shifted histograms of single variables, plots of pairs of variables,
and grand tours of multiple variables.  Views of the data can be
reshaped.  Points can be labeled and brushed with glyphs and colors.
Several displays can be open simultaneously and linked for labeling
and brushing.  Missing data are accommodated and their patterns can
be examined.
\end{abstract}

\section{Introduction}

This paper gives an overview of the layout and functionality of ggobi,
interactive graphical software for exploratory data analysis.  Readers
who are familiar with xgobi will find much that is familiar in ggobi's
design, and might want to skip directly to section \ref{slbl:xgobi}
where key differences between the two tools are described.

You will find that you can use ggobi for simple tasks with virtually
no instruction.  All that a user needs is some cursory knowledge of
the developments in interactive statistical graphics of the last 15
years, as well as a willingness to experiment with the sample data
provided guided by the tooltips.  In parallel with the hands-on learning
process, it is probably useful to acquire a basic understanding of
the overall layout and functionality of the system.  The greatest
success is obtained by users who have gained experience with the
system and combine it with creativity and data analytic
sophistication.

We first describe the specifics of GGobi's layout and functionalities.
Figure~1 shows a GGobi window, made using a version that was
available in early 2001.

\vbox{
\vspace{.4in}
%\epsfxsize 6in \centerline{\epsfbox{figures/manual1.ps}}
\centerline{[Figure 1 here]}

\vspace{.15in} \centerline{ \parbox{6in}
{Figure~1: Layout of a ggobi session.  The display window contains a
simple scatterplot.  Brushing has been used, and three different
symbols are in use.
}}
\vspace{.4in}
}

\section{Tutorial}

Several sample data files are included with the ggobi distribution,
in a directory called {\bf data,} and there you will find the file
{\bf pigs.xml}, on a dataset on pig production in the United Kingdom,
taken from Data by Andrews and Herzberg.  Start ggobi for the 
pigs data by typing:

\begin{quote}
ggobi pigs
\end{quote}

Two windows will appear, the ggobi control panel and a scatterplot,
as shown in Figure \ref{}.

The control panel has a panel of ViewMode controls on the left and a
variable selection region on the right.  You can see that the
scatterplot contains a 2-dimensional projection of the data, a plot
of YEAR vs TIME.  Move the mouse to one of the variable labels in the
variable selection region, and leave it in place until the tooltip
appears, explaining how to select new variables for the plot.   Begin
to get a feeling for the data by looking at several of the 2d plots:
profit vs herdsz, herdsz vs time.

Some of the variables are not self-explanatory, and here's
what they mean:

\begin {itemize}
\item GILTS: number of sows ``in pig'' (reproducing) for the first time
\item PROFIT: ratio of price to an index of feed price
\item S/HERDSZ: ratio of the number of breeding pigs slaughtered to the total
      breeding herd size
\item PRODUCTION: number of pigs slaughtered that were reared for meat
\end {itemize}

Next, get acquainted with the main menubar for the control panel by
exploring each of its menus.  Pay particular attention to the
Display, ViewMode and Tools menus.

\begin {itemize}
\item The Display menu is the interface for opening new plotting
  windows.
\item The ViewMode menu is the interface for specifying both the
  projection (1d, 2d, 3d or higher) and mouse interactions (for scaling
  the plot, highlighting points, and so on) for the current plot.
\item The Tools menu lets you open other windows to manipulate
  characteristics of the data and the view.
\end {itemize}

Using the Display menu, open another scatterplot display. Notice that
the new window has a narrow white band drawn around the outside of
the plotting area:  that means that the new window is now the
``current display'' and its plot is the ``current plot.''  Click in
plotting region of the other scatterplot to make it the current plot,
and now notice what happens in the variable selection region of the
control panel when you do that:  it should always show which
variables are plotted in the current plot.

Now set up a plot of GILTS vs HERDSZ in the first scatterplot, and
PROFIT vs HERDSZ in the second, and make that the current plot.
Using the ViewMode menu, choose {\bf Brush.}   The pink rectangle
that appears in the window is the ``paintbrush,'' and dragging it
over points changes their color.  Use the left button to drag it and
the middle or right button to resize it.  Make it almost as wide as
the window and about a quarter as tall, and paint first the
profitable herds, and then the unprofitable ones.  While you do that,
keep an eye on the of GILTS vs HERDSZ, and notice where the painted
points fall in that scatterplot.  Using two linked 2d plots is one
way to explore the relationship among three variables.

Look at the buttons and menus inside the leftmost portion of the
ggobi control panel.  Notice that they're contained by a frame
labelled ``Brush,'' which is the ViewMode of the current plot.  This
frame contains most of the brushing controls, which are described in
the section \ref{slbl:Brush}.  A few of the brushing controls,
though, are in the main menubar: in the Reset menu and at the bottom
of the Options menu.

Next use the Tools menu to open the {\bf Variable manipulation tool},
a very frequently-used part of the ggobi interface.  It contains a
table with a few basic statistics for each variable, and the buttons
below the table allow you to set variable limits, add new variables,
and a few other things.  Try clicking the mouse in the table -- 
you can select row of the table at a time, or use the control and
shift keys as modifiers to select more than one variable.  Hold the
control key down while clicking to select variables that are not
contiguous in the table; use the shift key if you want to select
a contiguous range of variables. 

You select variables in this fashion when you want to use one of
the tools to operate on them.  For example, select just the GILTS
variable, and then click on 'Clone' at the bottom of the tool.
You have just added a duplicate of the GILTS variable to the dataset,
and you'll see it at the bottom of the table in this tool, and in
the variable selection region of the ggobi control panel, too.
Select that new GILTS variable in the {\bf Variable manipulation tool},
then return to the Tools menu and open the {\bf Variable transformation
tool.}

[etc, do a transformation]


\section{Layout and functionality}

\subsection{The major functions}

Across the top of the control panel in Figure 1 stretch a row of buttons for
selecting major GGobi functions from pull-down menus.  Buttons
such as ``File'', ``Display,'' ``ViewMode'', ``Tools'', ``Options'' and
``DisplayTree'' organize functions in menus.  

As expected, the ``File'' button opens a menu for selecting
input/output functions as well as exiting.

The ``Display'' button opens a manu for opening a new plotting
window.  The display types are
\begin{itemize}
\itemsep 0em
\item Scatterplot,
\item Scatterplot matrix, 
\item Parallel coordinates plot,
\item Time series plots.
\end{itemize}

If the data has missing values, a scatterplot or scatterplot matrix
of the ``missingness'' information can be opened.

The ``ViewMode'' button opens a menu for selecting from ggobi's
inventory of interactive graphics operations: 
\begin{itemize}
\itemsep 0em
\item 1-D dotplots and average shifted histograms,
\item 2-D XY-plots, 
\item 1-D and 2-D tours, random or guided,
\item axis scaling,
\item brushing, 
\item identification, 
\item point motion.  
\end{itemize}

The unifying feature of these operations is that they take over the
display area and dictate the mouse operations for directly
manipulating the view.  Examples are mouse sweeps to control 3-D
rotations, or moves of the brush when brushing.  These operations
also determine the content of the control panel on the left of the
window (see below).  Exactly one mode is selected at all times.
``ViewModes'' will be discussed in greater detail in
section~\ref{slbl:ViewModes}.

The ``Tools'' button gives access to 
\begin{itemize}
\itemsep 0em
\item a variable manipulation table,
\item a variable transformation pipeline,
\item an variable sphering panel,
\item jittering controls,
\item a panel for hiding groups of cases,
\item subsetting functions for systematic and random subsampling, 
\item assignment of a value to missings.

\end{itemize}
The division between ``ViewMode'' functions and ``Tools'' is somewhat
arbitrary.  The ``ViewMode'' functions determine the mouse
interactions in the display windows, while none of the ``Tools''
does.  Furthermore, ``ViewMode'' functions populate the left hand
control panel in the main ggobi window, while ``Tools'' create their
own permanent control panels or views of the data in separate
windows, as shown in figure~2.  For example, the variable
manipulation ``Tool'' opens a table of data each variable, and the
transformation ``Tool'' generates a transformation control window.

The ``Options'' button allows users to set some options for the
main control window:  whether tooltips are displayed, for instance,
and whether the control panel is shown.

The ``DisplayTree'' menu allows users to open a tree listing
all the currently open display windows, each of which may contain
several plots.

In some ViewModes, an ``I/O'' menu appears, with options for
reading and writing data pertaining to that mode.

\subsection{Graphical displays}
\label{slbl:GraphicalDisplays}

\subsubsection{Current display, current plot}

Since there are multiple displays, some of which contain multiple
plots, the question arises:  Which plot in which display window
corresponds to the control panel?  If you select the {\bf Brush} {\bf
ViewMode}, how can you tell which plot is going to respond to
brushing?

There is in ggobi a notion of the ``current display'' and the
``current plot.''  (We need both because some displays, like the
scatterplot matrix, contain multiple plots.) The current plot is the
one which is outlined with a thick white border; the current display
is the one which contains the current plot.

To reset the current plot and display, just click once (left, right
or middle) in the plot you wish to address.  To understand the
effects of this selection, open a few displays and set them in
different ViewModes, then click on different plots and see what
happens.  The white border should follow your actions, and the
control panel should update so that its panels correspond to the
current display type and ViewMode.

\subsubsection{Display types}

Each display type is briefly described here.  A full description of the
view modes and variable selection behavior for each display will be
given in section \ref{slbl:ViewModes}.

As mentioned earlier, there are presently four main display types:

The {\bf scatterplot} display is a window containing a single scatterplot.
By default, it includes axes; they can be turned off using the Options
menu in the main control window.  It has the largest number of view
modes of any display.

The {\bf scatterplot matrix} is a window containing a symmetric matrix
of scatterplots for the chosen variables.  The plots along the diagonal
are ASHes (Average Shifted Histograms).

The {\bf parallel coordinates} display contains a single parallel
coordinates plot, which can be arranged horizontally or vertically.
(To understand this plot if you are encountering it for the first time, 
imagine deconstructing a high-dimensional scatterplot and arranging
its axes in parallel instead of orthogonally.  To represent case $i$,
think of drawing a dot on axis $j$ at that point's value for variable
$j$, and then connecting the dots into one set of connected line segments.
For a more detailed explanation, see [].)

The {\bf time series plots:}

[I can't quite figure this out.]

\subsubsection{Missing values displays}
% \ref{slbl:MissingData}

When your data includes missing values, two additional display types
are available.  The missing values scatterplot and scatterplot matrix
display not the data itself but the ``missingness'' for each point, namely
a jittered scatterplot of indicator variables encoding the presence or
absence of values for each variable.  Examining a number of views in one
of these displays allows us to explore the joint distribution of missing
values across variables.  Since these displays are linked to all others,
brushing the missings in a missing values display allows us to explore
the association between missing values and the variables.

\section{Data format}
\label{slbl:DataFormat}
\subsection {XML}
\label{slbl:XML}
\subsection {Database access}
\label{slbl:MySQL}

\section{Integration of ggobi with other software systems}
\label{slbl:Integration}

\section{View modes: projection and interaction}
\label{slbl:ViewModes}

Selecting any mode on the {\bf ViewMode} menu changes the interactions
available for the display and plot that are current, and pops the
corresponding control panel into the left portion of the main control
window.  The modes in the top half of the menu do something more as
well:  they set the projection method for the display, and the meaning
of actions in the variable selection panel always conforms to the current
projection method.

As an example, start ggobi with some data, and watch what happens
in the main ggobi window and a a single scatterplot display.   When ggobi
starts, it's in XYPlot mode by default, so the scatterplot window shows
a 2-dimensional projection of the data.  If you select {\em Scale} or
{\em Brush} (or any choice in the bottom half of the menu), the control
panel at the left of the main window changes, reflecting the different
interactions available to you in each mode.  However, the projection in
the window doesn't change.  If you click on the checkboxes in the
variable selection panel at the right of the main ggobi window, you
replace one of the plotted variables.

Now select {\em 2D Tour} in the ViewMode menu.  {\em Everything}
changes at once, because you've effective selected both a new mode and
a new projection type at the same time.
\begin{itemize}
\itemsep 0em
\item The control panel changes, because a new set of interactions
      just became available.
\item The variable selection panel changes, because the variable
      selection behavior for high-dimensional projection types is
      quite different than that for low-dimensional projection types.
\item The plot in the scatterplot display changes, because it's
      now showing a projection of 3 variables instead of 2.  Furthermore,
      it's moving, because a grand tour process is running.
\end{itemize}

If you now select one of the modes in the bottom half of the ViewMode
menu, you'll see again that the variable selection panel doesn't
change, and the plot in the scatterplot doesn't change -- except
that it stops moving.  The only thing that changes with every
selection is the control panel at the left of the main window.

Now we'll describe each view mode in more detail, starting at the
top of the ViewMode menu. 

\subsection{1D plots}

\subsection{XY plots}

\subsection{1D Tour}
\label{slbl:1DTour}
\subsection{2D Tour}
\label{slbl:2DTour}
\subsection{2x1D Tour}
\label{slbl:2x1DTour}

\subsection{Scaling of axes}
\label{slbl:Scaling}

\subsubsection{Drag}

For the default setting, Drag, the actions of the mouse can be
described in terms of a camera:  you're operating a camera and
looking at a projection of the data in the viewfinder.  When you use
the left button, the camera is panning freely around, following the
mouse exactly.  When you use the middle or right button, you're
zooming the camera in and out.

\subsubsection{Click: Pan or Zoom}

When you select the ``Click'' interaction style, the camera metaphor
becomes weaker, but your control of the panning and zooming becomes
more precise.

With ``Pan'' selected, the mouse controls the endpoint of a line
segment which is anchored at the center of the plot (just where the
center of the crosshair is in Drag style).  When you press the space
bar, you'll cause the plot to pan so that the endpoint becomes the
new center.  Repeated presses repeat the mostion in the same
direction -- convenient for browsing time-dependent data, for
instance.

With ``Zoom'' selected, the visual guide changes again:  this time,
the mouse controls a rectangle, and two keys are used:  ``i'' to
zoom in and ``o'' to zoom out an amount proportional to the size
of the rectangle.

To reset the plot, use the menu marked Reset in the main menubar:
it has two entries, allowing pan and zoom to be reset separately.

\subsubsection{Pan and Zoom Options}

By the default, the panning and zooming of the plot is unconstrained,
moving or rescaling vertically and horizontally with each action.
The pan and zoom options allow it to be constrained so that only
one axis is affected, convenient for browsing one variable at a time.

\subsection{Brush: Linked brushing of points and lines}
\label{slbl:Brush}

Brushing is often performed when only a single display is visible,
but it is most interesting and useful to perform brushing with more
than one linked display showing different views of the same data.  

To interactively paint points, drag left to move the ``brush'' within
the plotting window, or drag middle to change the size or shape of
the brush while you paint.  (If you lose the brush by pulling it
outside the plotting window, you can grab it again if you press the
left or middle button while the cursor is inside the display window.)

\subsubsection{Brush On}
%
When the {\bf Brush on} toggle is checked, moving the brush
over a plotted point causes that point to change its color or
plotting character (called a glyph).  If the brush is turned off, the
brush can be freely moved across the plotting window and it does not
change the points.  

This is useful if you are plotting a very large number of points,
and you want to position the brush before painting, because you
can move it much more quickly across the plot.  [This isn't true
in the default mode, where the brush jumps to the cursor.  Shall
I re-add the other mode?]

%
\subsubsection{Points and lines}
If {\bf Points} is selected, the brush is drawn as a rectangle.
As the brush is moved across the points, any points contained by
the brush are redrawn using the currently selected glyph and color.
If {\bf Lines} is selected, the brush is drawn as a crosshair, and
as the brush is moved in the window, any lines intersecting either
the vertical or horizontal ``hair'' are redrawn in the chosen color.
(There is no line brushing implementation for line types yet.)
If both {\bf Points} and {\bf Lines} are
selected, the brush is drawn as a crosshair inside a rectangle,
and both point and line brushing are performed.

\subsubsection{Color and glyph}
%
use the {\bf Color and glyph} menu to choose whether to brush
with color, glyph, or both, or whether to hide the points you
paint.

%
\subsubsection{Brushing modes:  persistent, transient}
%
These are two brushing modes.
\medskip
\noindent
\\{\bf Persistent}:  When you brush a point, it retains its new color or glyph
when the brush has moved away.
\\{\bf Transient}:  As the brush moves off a point, it returns to the color and
glyph it had before the brush covered it.

\subsubsection{Undo}
%
Clicking on the {\bf Undo} button restores the glyph and color and
visibility of all points painted between the last mouse-down and
mouse-up.

\subsubsection{Reset Menu}
%
When the brushing mode is active, the Reset Menu in the main
menu bar contains these items:

\begin{tabbing}
 {\bf Show all points} \hspace{.5in} \= Make all points visible. \\
%{\bf Reset point colors} \> Restore all points to the default color. \\
%{\bf Reset glyphs} \> Restore all points to the default glyph. \\
 {\bf Show all lines} \> Make all lines visible. \\
%{\bf Reset line colors} \> Restore all lines to the default color. \\
 {\bf Reset brush size} \> Reset the brush to its default size and position. \\
\end{tabbing}
%

\subsection{Linked identification of points}

This mode is used to display labels near points in the plotting
window.  The labels are taken from the file of row or case labels
supplied by the user; if the file is not present, the row number of
each case is used.  To see these labels, simply move the cursor
inside the plotting window.  The label of the point nearest the
cursor is displayed.

Identification in one window is instantly reflected in all linked windows.

To cause a label to become ``sticky,'' click left when the target label
is printed.  The printing style changes and the label now remains
printed as the cursor moves off, and even remains printed as you leave
the {\bf Identify} mode.  It is possible to rescale or rotate data, and
the sticky labels will continue to be printed next to their associated
points.

To cause a label to become ``unsticky,'' return to the {\bf Identify}
mode and click left again when the target point is nearest the cursor.
It is also possible to restore all labels to unsticky status by
clicking on the {\bf Remove Labels} button.  You can also see all the
labels at once by clicking on {\bf Make all sticky}. 

\subsection{Line editing}

\section {Display types}
\subsection{Scatterplot}
\subsection{Scatterplot matrix}
\subsection{Parallel coordinate plot}
\subsection{Time series plots}

\newpage
\section{Tools}
\subsection{Variable manipulation tool}
\label{slbl:VarManip}

This powerful tool is opened by selecting the first entry on the
{\bf Tools} menu.  It has several important functions:
\begin{itemize}
\itemsep 0em
\item the display of variable statistics,
\item variable selection for other tools,
\item setting variable ranges,
\item cloning variables, and
\item adding other new variables.
\end{itemize}

Its first purpose is to display a few statistics for each variable:
the current variable transformation (if any); the minimum, maximum,
mean, and median of the raw data; the number of missing values per variable.

You can also use it to select variables -- not for plotting, but to
be operated on by other tools.  Variables are selected by highlighting
rows, and the control and shift keys are modifiers that allow multiple
rows to be highlighted.  The variable transformation tool, among
others, will operate on the variables selected in this way.

These selected variables will also respond to operations contained
within this panel:  you can reset the variable ranges that are used
for projecting the data into the plotting window.  This is especially
useful for a data set where several of the variables have the same units.

The selected variables can also be cloned, and the new variables
you create will be added to the table as well as the main control panel.

There's another way to add new variables, and that relies on the {\em New
...} button, which brings up a small panel.  Use that panel to specify
the variable's name and to set its values:  either the row numbers or
a set of integers reflecting the groups currently defined by brushing.

\subsection{Jittering}

Select {\em Variable jittering ...} to open a panel that allows
random noise to be added to selected variables.

First specify the variables you wish to jitter.  By default, the
variables plotted in the current display will be affected; if you
wish to select others, use the {\bf Variable manipulation tool.}

Choose between uniform and random jitter, and then set the degree of
jitter using the slider.  To rejitter without changing the degree of
jitter, simply click on the {\bf Jitter} button.

\subsection{Controls for missing data}

By default, missing values are assigned the value $0$, but you
may sometimes find that to be an inconvenient choice.  Use the
{\em Missing values} panel to assign alternative numbers.

By default, all missings in all variables will be affected
together, but you can choose to assign values only for selected
variables using the menu at the top of the panel.

Below the menu, a notebook widget allows you to specify the
type of imputation you'd like to use.

{\bf Random imputation}: Sample from the present values for each variable
  to populate the missing values.  If you have done some brushing to
  partition the cases, you can specify that you want the sampling to be
  done using only cases brushed with the same color and glyph.
\\{\bf Fixed value}: Specify any value to use instead of the default.
\\{\bf Percentage below minimum}: Specify a value that is $x$ percent
  above the minimum value.  For example, if the variable ranges from
 $40$ to $80$, specifying 10 will assign the missings the value $40 - (10\%
 * 40) = 36$.
\\{\bf Percentage above minimum}: Specify a value that is $x$ percent
above the maximum.

If you want all plots to rescale immediately when you assign new
values, turn on {\em Rescale} toggle.  Finally, click on the {\em Impute}
button to perform the imputation or assignment of values.

\subsection{Sphering}
\label{slbl:Sphering}

The {\em Sphering} panel starts by displaying a scree plot for the
currently selected variables (see section \ref{slbl:VarManip}).  If the
scree plot doesn't reflect the variables you want to sphere, then open
the Variable statistics panel, select the variables of interest, and
click on {\em Update scree plot}.

It's advisable to standardize the variables first, so a label at the top
of the panel reminds you if they aren't.  In that case, open the {\em
Variable transformation} tool and standardize the selected variables,
then click {\em Update scree plot.}

Now you're ready to sphere the selected variables.  Working your way down
the panel, use your visual interpretation of the scree plot together
with the information in the labelled section ``Prepare to sphere'' to
decide how many principal components you want to create.  By default,
all the selected variables will be sphered, but you decide that the first
few principal components account for a sufficiently high proportion of
the variance.  In that case, you can use the spin button to the right
of the label ``Set number of PCs'' to decrease the number of principal
components you're going to generate.  The variance and condition number
are displayed to help you make that choice.

Once you're satisfied with the selected variables and the number of
principal components, proceed to the last step.  Click on {\em Apply
sphering} to create new variables and add them ggobi's variable selection
panels.  The names of the selected variables will be added to the
``sphered variables'' to help you remember which variables you sphered.

[And I forget what 'Restore scree plot' is for -- dfs]

\subsection{Subsetting}

Select {\em Case subsetting and sampling ...} to open a panel
that allows subsets to be specified in one of five ways.

{\bf Random sample without replacement}:  Specify the number of
cases to be in the sample.
\\{\bf Consecutive block}:  Specify the first and last row of the block.
\\{\bf Every nth case}:  Specify the interval and the first row.
\\{\bf Sticky labels}:  All cases with a ``sticky'' label will
  be in the subset.  (See {\em Identification} for a description
  of sticky labels.)
\\{\bf Row labels}: Type in a row label, and all cases with that
  label will be in the subset.

Select one of those five, then click on {\bf Subset} in the
bottom row of the panel.  If you want to re-include all rows, 
select the {\bf Include all} button in the bottom row.

If the {\bf Rescale} button is checked, then the plots will be
rescaled to exclude all points not in the subset.  If it is
not checked, the points will be hidden but not excluded from
plot scaling operations.

One purpose of subsetting is to allow the use of ggobi on data matrices
that are so large that dynamic and interactive operations begin to
become painfully slow.  By selecting a smaller subset, a user can
work on that subset at a comfortable speed of rotation and
interaction.  Another purpose is to do graphical cross-validation:
if the feature you see is still there in repeated subsamples, there's
a good chance it's not just an artifact of visualization.

\section{Implementation}

\section{Differences from xgobi}
\label{slbl:xgobi}

In this section, we summarize the key differences between ggobi
and xgobi for those readers who are already familiar with xgobi.

\subsection {Multiple displays}

The first thing you'll notice when you look at a ggobi display is
that the plotting window has become separated from the control
panels.  The main reason for that change is so that a ggobi process
can have multiple display windows, of the same type or of different
types.  In addition to the basic scatterplot, ggobi currently has
scatterplot matrices, parallel coordinates plots, and time series
plots.  (See \ref{slbl:GraphicalDisplays} for more detail.)

This design change has had far-reaching effects.

First, user interactions available for the simple scatterplot display
can now be made available for other display types.  In xgobi, for
instance, there is a parallel coordinates display, but it's not
possible to brush it -- in ggobi, it is. 

Unfortunately, having multiple displays introduces a new source of
ambiguity: you now have to tell ggobi which display, and which plot
within a display, you want to address.  Do that by simply clicking
inside the target plot.  ggobi will draw a thick white outline around
that plot so that you can check which plot your actions will be
addressing.  The ViewMode panel in the main ggobi window should
always correspond to the state of the current display, too.  We need
more user experience before we can tell whether this approach is
satisfactory.

The basis for linking has changed, too:  since all displays are
linked by default, it's no longer necessary to run multiple processes
in order to achieve linked displays.

One of the most interesting implications of using multiple displays
is that a ggobi process is no longer restricted to a single data
set.  The XML file format makes it easy to specify two or more data
matrices in a single file, as described in section \ref{slbl:XML}.
In addition, it's possible to add data matrices using the Read button
on the File menu or using R.  (The R~-~ggobi interface will be
introduced below, and it's more fully described in section
\ref{slbl:Integration}.)

The rules and the interface for linking across data sets have not yet
been defined.

With multiple displays, too, we no longer have to launch a new
process to open missing value plots -- the plots of 1's and 0's
which represent the presence and absence of data in each cell.

A convenient side effect of multiple displays is that, since each
display now sits in a window of its own, it's now very simple to
adjust the aspect ratio of a plot; this simple operation is very
awkward in xgobi.

\subsection {Data format}

Before we describe other key changes in the visible design, we'll
introduce the changes in data format.  The xgobi format, in which a
set of files with a common base name is used, is still supported,
though some file formats have changed ({\em .colors}), and some are
no longer supported ({\em .bin} and {\em .vgroups}).  (See section
\ref{slbl:DataFormat} for details.)

The new format, in which all the data lives in one file, is written
in XML.  XML, the Extensible Markup Language, is a widely used
language for specifying structured documents and data to be viewed
and exchanged.  It was initially intended to be read by browsers, but
it is also used to define documents that are read by other software.
XML files can be validated automatically, and XML specifications can
be easily extended, too, by adding to the set of tags in use.

As ggobi grows and develops, we expect to favor the XML format:  it
will become increasingly difficult to keep the old format up to
date.  For instance, we will no longer support the ``.bin'' file used
by xgobi, which can be used to store the data in a binary format.
Instead, the XML file will have a tag that can be used to give the
name of a binary file to be read -- and the binary format we use will
be portable.

A few XML data files are included in the sample ggobi data, and
the details of their format is described in ??.

Another innovation in data access is the ability to read data
directly from a MySQL database, as described in section \ref{slbl:MySQL}.

\subsection{Integration}

Many xgobi users are also users of the SPlus or R statistics
software, and have used the S function which launches an xgobi
process viewing S data.  Once that launch occurred, the resulting
process was utterly independent of its parent.  The xgobi authors
did some experiments in the early 90s to achieve a more intimate
connection, but made little headway.  (reference)

With ggobi, that problem has been solved, and it's now possible
to have real-time integration between ggobi and a variety of
other software environments.  An example of this integration is
the embedding of ggobi into R (or S), with the addition of a set of
R (or S) functions that manipulate ggobi data and displays,
resetting data values, projection, and the appearance of the plot.

For more details, see section \ref{slbl:Integration}, or read
(reference).

\subsection {Variable selection}

There have been a few changes in variable selection.  The
familiar variable circles are still used for high-dimensional
view modes, but we've switched to a simple checkbox interface
for plots where only one or two variables can be selected
simultaneously.  In these plots, the rich feedback provided
by the variable circles is not needed, and may just be confusing
to novices.

The basics of the user interface haven't changed, though:
click with the left button to select a variable to be plotted
horizontally and the middle (or right) button to plot vertically.

\subsection {The variable manipulation tool}

The table in the {\bf Variable manipulation tool} can also be used to
select variables -- not for plotting, but to be operated on by other
tools.  Several of the tools, such as variable transformation and
jittering, will operate on the plotted variables in the current
display by default.  However, if any rows in this table are
highlighted, then all selected variables will be affected.

This table can also be used to specify limits for variables or groups
of variables, and so we have dispensed with the {\em .vgroups}
functionality in xgobi.

Variable cloning is another new feature: it appears as a button
on the variable selection and statistics table.

For more detail on this table, see section \ref{slbl:VarManip}.

\subsection{Changes in ViewModes and Tools}

{\bf Brush} may be the {\bf ViewMode} that has changed the most,
because ggobi has a much richer notion of color selection than
xgobi.  Open the {\bf Choose Symbol} panel, and then double-click on
any element of the color palette to bring up a color selection widget
with access to the full color map.  Notice that you can change the
background color as well as any of the foreground colors.
%See section \ref{slbl:Brush}.

{\bf Scale} has also changed -- the direct manipulation shifting and
scaling methods work as they did in xgobi, though you now specify
whether you want to {\bf Pan} or {\bf Zoom}, and then use the same
mouse button for both.  New methods (the {\bf Click} interaction
style) that allow greater precision have replaced the xgobi controls
that were represented by buttons on the control panel.  See section
\ref{slbl:Scaling}.

The tour methods in ggobi are still evolving:  many things are not
yet implemented, such as projection pursuit, but new things are
beginning to appear.  In the {\bf 1DTour}, a projection of several
variables is viewed an an average shifted histogram, as described
in section \ref{slbl:1DTour}.

The redesign of the tour methods is reflected in the {\bf Sphering}
tool, which also appears in the most recent versions of xgobi.
Instead of automatically sphering variables in projection pursuit,
ggobi allows you the choice:  you can use the {\bf Sphering} tool to
decide which variables to sphere.  Since this method makes use of
variable cloning, it creates new variables, allowing you to look at
plots of principal components against the original data.

See section \ref{slbl:Sphering} for details and pictures.

\subsection{On-line help}

The on-line help system used in xgobi has been replaced with
``tooltips,'' so leaving the mouse over a widget for a couple of
seconds brings up a phrase describing the function of that widget.
If the tooltips annoy you, you can turn then off using a
checkbox on the {\bf Options} menu.

\end {document}

