\documentclass{article}
\input{WebMacros}
\input{CMacros}
\usepackage{fullpage}
\usepackage{times}

\title{Using Gtk types and ``classes'' for GGobi objects.}
\author{Duncan Temple Lang}
\begin{document}

\maketitle

\begin{abstract}

  The extensible type system provided by Gtk can be used to give GGobi
  a reasonable object-oriented framework on which we can build
  extensions to the basic plot types, etc.  Additionally, by complying
  with the Gtk type system, automatically generated bindings such as
  pygtk, java-gnome, perlgtk and RGtk can more readily interface to
  GGobi.  Also, dynamically reflected information about these types is
  available which can be used by the end-user and development
  environments.

\end{abstract}






\section{Older}

We oultine the necessary modifications and the benefits introduced.

By having a clean interface for a class of splots
(and displays), we can have a well-defined interface
for such things as 
\begin{description}
\item[description]
\item[serialization]
\item[construction]
\end{description}

It is relatively easy to introduce new plot types
without having to add any extra code to the core code.

For the constructors, how do we get the ggobid instance to the
display, and the display to the new splot, etc?

What we want to be able to do is introduce a new plot type, perhaps
simply as a GtkType or perhaps we require slightly more in addition.
Then, when a user asks for a sequence of these plots, it is easy to
create them.

Suppose we register a display type.  This puts an element into a
table.  Then when we create the Display menu
(\Croutine{display_menu_build}), we loop over those elements and add
each to the menu.  We ask each element for a menu name.  At this
point, we are good to go and the user can take control.

Now, when the user selects this plot type, we are in
\Croutine{display_open_cb}.  This looks for a property on the menu
widget named \texttt{displayMetods} and that, if non-\CNull, gives a
reference to the \CStruct{DisplayInterface} to use.  This defines the
different methods for creating the display.  This is passed to
\Croutine{display_create} which actually creates the display and its
plots.  It invokes the method \Croutine{display_create_with_vars} and
passes it the selected variables.  If this routine is not specified,
but \Croutine{display_create} is, then the latter is called.  This
allows the display to determine by itself which variables to plot.

Both forms of the \Croutine{display_create} routine are expected to
create the entire display window.  How they negotiate to create the
appropriate plot types is their own business. They should attempt to
use the existing infrastructure, when possible.


Also we ask what mode the plot type for that display supports
(i.e. Scale, Brush, Identify, Move Points) We can use a fixed
collection of flags against which we OR the result.  Alternatively, we
can ask the display type for a collection of mode names.
Alternatively, we can ask the display to create the ViewMode menu.

splot and display.  There is a mix between these that needs to be
clearly defined and we would like to be able to avoid
redundancy/duplication in display and plot type specification.


\section{displayd}
Basic idea is to remove the switch statements
an call a single routine via a function pointer
that is specific to the class of display.

\Croutine{cpanel_set} (\file{cpanel.c}).
Rather than having a switch statement, we would have
a single line.
\begin{verbatim}
   display->cpanel_set(display, gg)
\end{verbatim}

Way to determine which variables should be included in a display
for a given type. (\file{vartable.c})

\Croutine{varsel} calls \Croutine{tsplot_varsel}, etc.  in
\file{varpanel_ui.c}
Also \Croutine{varpanel_refresh}.
And \Croutine{varpanel_tooltips_set}.


The routine \Croutine{plotted} (which seems to determine
if a variable is currently being plotted in any of
the displays associated with a given dataset).
\file{varchange.c}
Ok 

display.c

\Croutine{computeTitle}

\Croutine{display_type_handles_action} determines whether the display
type handles a particular view mode.

\Croutine{splot_add_plot_labels} in \file{sp_plot.c}.


\Croutine{splot_world_to_plane}

\Croutine{splot_plane_to_screen}


How to get the methods set for an \CStruct{splotd}.  Well, we can do
this with a Gtk class.  When we create the splotd, we use an
appropriate GtkType that sub-classes Gtk


Computation of \CVariable{loop_over_points} in \Croutine{splot_draw_to_pixmap0_unbinned}
in \file{sp_plot.c} for extension types.
Ok.

Remove the drawing of the line segments for tsplot in \Croutine{splot_draw_to_pixmap0_unbinned}
in \file{sp_plot.c}.
Ok.


\CppMacro{DOPT_MISSINGS} in \Croutine{display_options_cb} in \file{display.c}
Ok.

Remove the menu creation in \Croutine{display_menu_build}
Ok

\Croutine{display_type_handles_action} in display.c
 Ok.

\Croutine{plotted} in varchange.c
 Ok.

tsplot case in varsel in \file{varpanel_ui.c}
Ok

\Croutine{varpanel_refresh}
 Ok.

\Croutine{varpanel_tooltips_set}
 Ok

vartable.c
\Croutine{plotted_col_gets}
 Ok;


Facilities for reading about a plot type in other applications
using reflectance. E.g. R

In \file{read_init.c} we restore previously created displays.  Need a way to
find the type. See createDisplayFromDescription and getDisplayType.



\end{document}



