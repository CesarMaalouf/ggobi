\documentclass{article}

\usepackage{fullpage}
\title{XML Input Format}
\begin{document}
\section{Using XML Input Formats}
There is now support for reading the data from an XML
file.
This is an alternative to having the inputs
for a dataset be contained in a collection of multiple but associated files
that provide the different characteristics such as 
\begin{itemize}
\item data
\item observation labels
\item glyph information
\item color
\item connected segments.
\end{itemize}
Instead, the XML format allows all the information to be specified in
a single file.  Additionally, different sections of information can be
omitted.


One of the advantages of XML is that it can be validated externally.
In other words, a well-formed file can be tested outside of the
application for which it serves as input.  This helps to prepare and
maintain correct input files.  Given the ability of R, S and Omegahat
(and an increasing number of other statistical applications) to read
XML, the dataset can be used in other applications with little or no
additional code.  (Of course, read.table() and associated functions in
R/S makes reading the old file formats easy.)

XML parsers can check whether the document is well-formed.  This means
that all obligatory sections are present, that sections are in the
correct place.  Additionally, identifiers can be specified for each
row and validating parsers can check that they are unique.

Additionally, more useful information can be added
to the data source.
This includes items such as 
\begin{itemize}
\item how missing values
are encoded (for the entire dataset or per variable),
\item how many records there are, 
\item levels of a variable that are not observed
but known to exist (e.g. ethnicities not encountered in a survey)
\item the source of the data
\item tooltips for use in describing the data
\item the type of each variable (e.g. factor or numeric)
\end{itemize}
The fact that the number of records and variables are specified in the
file format means that only one pass of the file is needed to read the
data and no reallocation is needed as more observations than expected
are located.  Additionally, it is easier to handle non-rectangular
data.  For example, sparse data and variable number of values per
observational unit (e.g in medical studies).


Additionally, the growing usage of XML means that there are editors
and browser to create and view XML files.

There is no doubt that the XML format appears more verbose and indeed
it is. However, its more rigid structure benefits the authors of the
input files as well as the application programmers.  It is more
convenient and significantly less error-prone to have related
information be in the same location.  For example, specifying the
color of a line segment connecting two points in one file and the
actual connection in another means that one has to have a mechanism to
link the two specifications. Frequently this is a simple order in
which they appear in the different files.  Removing individual lines
of information commonly leads to lengthy searches for why the input
does not produce the desired result.  Often this is because of an
``off-by-one'' connection.


Note that the XML approach will also be used to generate and
potentially read plot descriptions for persistence and exchange with
other software systems.


A final benefit to using XML is that we have support for reading
compressed files.  The XML parser we employ (Daniel Veillard's libxml)
can parse XML directly from compressed files.  For large datasets,
this is convenient as we don't have to uncompress the files before
using them.

Being able to specify file-specific defaults allows one to easily
change the characterstics of the plots without excessive editing of
the contents of the file.  For example, to use a different glyph type,
we need only specify a different value for the \texttt{glyphType}
attribute in the \texttt{ggobidata} tag.  This greatly simplifies
experimenting with different parameter values.



\section{The File Format}
The format of the file is described by the 
DTD (Document Type Definition) 
\texttt{ggobi.dtd}

The file starts with the usual XML
declarations that identify it as XML
(and its version)
and the particular document type
and associated DTD.
\begin{verbatim}
<?xml version="1.0"?>
<!DOCTYPE ggobidata SYSTEM "ggobi.dtd">
\end{verbatim}
The string ggobidata indicates that this is the top-level tag for the
document, and this is what appears next.  The attributes of this tag
specify the number of records in the dataset.  The identifier for a
missing value can also be specified as an attribute
(\texttt{missingValue}). (We might allow this to be overridden for
each variable.)

Also, default values for different characteristics of the observations
can be specified here.  For example, the glyph type and size, the
color of an observation, etc. can all be specified here.  These are
applied to the individual observations for which they are not
specified.  So, such an entry might look like
\begin{verbatim}
<ggobidata numRecords="" color="2" glyphType="fc" glyphSize="3">
\end{verbatim}
The remainder of the dataset is specified as sub-elements or sub-tags
within this \texttt{ggobidata} element.  Thus, there must be a
termination of this \texttt{ggobidata} at the end of the file.

The first of the sub-elements is a description of the datasets.  This
includes the source, any references, etc.  This is currently
free-format.  A convenient attribute is \texttt{source} which
indicates where it can be found.

The next element of the file 
lists the variables.
\begin{verbatim}
<variables>
<variable name="A" group="1" />
<variable name="B" group="1" />
</variables
\end{verbatim}
The name of the variable
can be specified as the text within the
variable tag rather than as an attribute.

The name of the transformed variable can be specified via the
attribute \texttt{transformedName}.

The group attribute allows variables to be joined for the same purpose
as ifentified by the .vgroups file in the old format.

Additionally, instructions as to how to create the variable can be
specified as a programming command via the Programming Instruction
(PI)
\begin{verbatim}
<variable>
<?R rnorm(10)>
</variable>
\end{verbatim}

The next section of the file is the data itself.  The individual
\texttt{record} tags are contained within the \texttt{records}
element.  The body or content of each \texttt{record} is a simply
ASCII listing of the values. Each value is separated
by white space (space character, tabs or new lines).


\section{Using XML Files}
To use data that is contained in an XML file, 
invoke ggobi with the command line flag
\texttt{-x}.

In the future, ggobi could be smart enough to include the detection of
the XML format.



\section{Conversion of Old Files}
The distribution contains an application named xmlConvert that can be
used to read datasets provided in the old file format to XML.  This
can be used by specifying the name of the file containing the
old-style data in the same manner as ggobi expects.
The output is written to standard output
and can be redirected to a file using basic shell commands.
For example,
\begin{verbatim}
  xmlConvert data/flea > flea.xml
\end{verbatim}
In the future, we will support writing the output to a file. (We need
to process the command line arguments and look for a -o flag).


Note that this dynamically loads the libraries libGGobi.so and
libxml.so.  Thus the directories that contain these libraries must be
referenced in the environment variable 
\texttt{LD_LIBRARY_PATH}.
Alternatively, the makefile can be edited to statically link these
libraries.


\section{Compilation}
To activate the XML mechanism, define the variable
\texttt{USE_XML} in  local.config.

When we use autoconf, this can be done by
\begin{verbatim}
  ./configure --with-xml
\end{verbatim}

This requires the XML parsing libray libxml (also know as gnome-xml) by Daniel Veillard.
(\texttt{Daniel.Veillard@w3.org}).
See \texttt{http://xmlsoft.org/#Downloads}
\section{References}
The XML Handbook, Charles F Goldfarb and Paul Prescod
 Prentice Hall.

http://www.w3.org/XML

\end{document}
