\documentclass{article}

\input{pstricks}
\input{WebMacros}
\def\directory#1{\dir{#1}}

\def\XMLAttribute#1{\Escape{#1}}
\def\XMLTag#1{\Escape{#1}}
\def\XMLElement#1{\XMLTag{#1}}

\usepackage{fullpage}
\title{XML Input Format}
\begin{document}
\section{Using XML Input Formats}
There is now support for reading the data from an XML
file.
This is an alternative to having the inputs
for a dataset be contained in a collection of multiple but associated files
that provide the different characteristics such as 
\begin{itemize}
\item data
\item observation labels
\item glyph information
\item color
\item connected segments.
\end{itemize}
Instead, the XML format allows all the information to be specified in
a single file.  Additionally, different sections of information can be
omitted.


One of the advantages of XML is that it can be validated externally.
In other words, a well-formed file can be tested outside of the
application for which it serves as input.  This helps to prepare and
maintain correct input files.  Given the ability of R, S and Omegahat
(and an increasing number of other statistical applications) to read
XML, the dataset can be used in other applications with little or no
additional code.  (Of course, read.table() and associated functions in
R/S makes reading the old file formats easy.)

XML parsers can check whether the document is well-formed.  This means
that all obligatory sections are present, that sections are in the
correct place.  Additionally, identifiers can be specified for each
row and validating parsers can check that they are unique.

Additionally, more useful information can be added
to the data source.
This includes items such as 
\begin{itemize}
\item how missing values
are encoded (for the entire dataset or per variable),
\item how many records there are, 
\item levels of a variable that are not observed
but known to exist (e.g. ethnicities not encountered in a survey)
\item the source of the data
\item tooltips for use in describing the data
\item the type of each variable (e.g. factor or numeric)
\end{itemize}
The fact that the number of records and variables are specified in the
file format means that only one pass of the file is needed to read the
data and no reallocation is needed as more observations than expected
are located.  Additionally, it is easier to handle non-rectangular
data.  For example, sparse data and variable number of values per
observational unit (e.g in medical studies).


Additionally, the growing usage of XML means that there are editors
and browser to create and view XML files.

There is no doubt that the XML format appears more verbose and indeed
it is. However, its more rigid structure benefits the authors of the
input files as well as the application programmers.  It is more
convenient and significantly less error-prone to have related
information be in the same location.  For example, specifying the
color of a line segment connecting two points in one file and the
actual connection in another means that one has to have a mechanism to
link the two specifications. Frequently this is a simple order in
which they appear in the different files.  Removing individual lines
of information commonly leads to lengthy searches for why the input
does not produce the desired result.  Often this is because of an
``off-by-one'' connection.


Note that the XML approach will also be used to generate and
potentially read plot descriptions for persistence and exchange with
other software systems.


A final benefit to using XML is that we have support for reading
compressed files.  The XML parser we employ (Daniel Veillard's libxml)
can parse XML directly from compressed files.  For large datasets,
this is convenient as we don't have to uncompress the files before
using them.  You can try this feature by using GNU zip (gzip)
to compress the file flea.xml in the \directory{data}
directory and starting the ggobi application
\begin{verbatim}
  ggobi -x data/flea
\end{verbatim}
This searches for a file named data/flea.xml and if this is not found,
\file{data/flea.xml.gz} and then \file{data/flea.xmlz}.  The parser
automatically determines whether it is compressed or not.  This
support can be turned off.  Some simple tests illustrate that the XML
representation can be about about 50\% larger than their simpler ASCII
equivalents (i.e. without the markup) but that the compression brings
them to about $1/3$ the size. The speed at which the compressed file
is read is almost the same as the uncompressed XML, and is about 33\%
longer than the ASCII equivalents. (These were done on a $100000
\times 5$ matrix.  Note also that all the color and glyph information
was included in the XML and not provided in the ASCII, so the XML
looks slightly better than the above numbers suggest.)

Note also that the XML parsing library has support for reading files
via ftp and http.
\begin{verbatim}
  ggobi -x http://www.ggobi.org/data/flea.xml
\end{verbatim}



Being able to specify file-specific defaults allows one to easily
change the characterstics of the plots without excessive editing of
the contents of the file.  For example, to use a different glyph type,
we need only specify a different value for the \texttt{glyphType}
attribute in the \texttt{ggobidata} tag.  This greatly simplifies
experimenting with different parameter values.


In contrast with the multiple input files for each dataset used by
xgobi, the XML approach reduces the number of files to $1$. This means
that it is easier to distribute inputs to others. There is no need to
send multiple files individually, or to combine them into an archive,
etc.

An advantage of XML over a simpler but omre specialized format is that
people are somewhat familiar with the basic rules of HTML, and hence
XML. Additionally, it is easy to define new DTDs to represent
different inputs such as property or resource files, descriptions of
plots (see \dir{SVG/}), layout specifications for multiple plots,
graph descriptions, etc.  This can leverage much of the same parsing
setup and importantly provides a uniform and increasingly familiar
interface for the user for specifying files.


\section{The File Format}

The format of the file is described by the DTD (Document Type
Definition) \texttt{ggobi.dtd}.  [Not at the moment, I bet.]
The file starts with the usual XML declarations that identify it as
XML (and its version) and the particular document type and associated
DTD.

% This doesn't work now; I get a core dump.
%<!DOCTYPE ggobidata SYSTEM "ggobi.dtd">

\begin{verbatim}
<?xml version="1.0"?>
<ggobidata>
\end{verbatim}
%
The string ggobidata indicates that this is the top-level tag for the
document, and this is what appears next.  To specify that more than
one dataset is inlcuded, use the \texttt{count} attribute:
%
\begin{verbatim}
<ggobidata count="2">
\end{verbatim}
%
This tag must be terminated at the end of the file:
\begin{verbatim}
</ggobidata>
\end{verbatim}

\subsection{Data}

This is followed by the \texttt{data} tag, which begins the
entries for a dataset:

\begin{verbatim}
<data name="Flea beetles">
\end{verbatim}
%
Here you can also specify the name which will appear in the titlebars
of ggobi windows.

There can be multiple datasets within a file, and there can be two
types of relationships among their elements: 
\begin{itemize}
\item records in multiple datasets can represent different variables
  recorded for the same subject, as described in section
  \ref{LinkingRecords}, or
\item one dataset can contain a description of edges which connect
  points in another, as described in section \ref{Edges}.
\end{itemize}

The remainder of the dataset is specified as sub-elements or sub-tags
within this \texttt{data} element.

\subsection{Colormap}

The user can specify rows of a color table in RGB format.  (Perhaps
different formats such HSV, etc. can be supported directly by GTK
also.)  The idea is that colors for records, etc.  are specified as
row identifiers for this table.  The colormap section allows the user
to specify the values for these rows.  In this way, the data and color
references can remain fixed and one need only change the contents of
the colormap to which the correspond.

It is often convenient to use the same colormap for several
datasets. Rather than encoding the data in each input file, the XML
input file can be told to read color data from an external file.  This
is specified via the \XMLAttribute{file} and \XMLAttribute{type} The
value of the \XMLAttribute{file} attribute identifies a URL or
file. If not an absolute file name or URI, then this is located
relative to the location of the document being currently read.
That is, suppose we are running in the top-level distribution
directory of ggobi and run with the input file \file{data/flea.xml}.
Then, a reference to 
\begin{verbatim}
 <colormap file="stdColorMap.xml">
\end{verbatim}
is found as \file{data/stdColorMap.xml}.
Similalry, if a colormap file named \file{map.xml} is located
in the directory \dir{color} parallel to \dir{data},
then
\begin{verbatim}
 <colormap file="../color/map.xml">
\end{verbatim}
would find it for the same input file \file{data/flea.xml}.

When an external colormap file is used in conjunction with local
values, there are two size specifications.  One is in the colormap of
the external file and the other is local. Both refer to the number of
local entries, i.e. entries under their control.  By default, these
are accumulated so that the size of the table is the sum of the two
sizes.  This can lead to oversized tables.


External color map files can be formatted using XML (see
\file{data/stdColorMap.xml}) or as a simple rectangular array.  In the
latter case, each row should contain 3 values separated by white
space.


To differentiate between the two formats, the \XMLAttribute{type} is
used when specifying an external color map file to read.  This can be
either \texttt{xml} or \texttt{ascii}.  (See \file{ggobi.dtd})


Each entry can have an identifier attribute.  This is usually an
integer identifying the index by which one can reference this color in
records and other files.  The alternative values for the identifier
attribute are ``fg'' or ``bg''.

The identifier is really only used to override other values read from
a file.  This is due to the fact that subsequent entries without
identifiers occupy the next entry  in the color table.
In other words, the input
\begin{verbatim}
<color id="4" r=".5" g=".5" b="0" />
<color>0 0 1</color>
\end{verbatim}
sets the $5$ entry (in position $4$) to blue ($0 0 1$) since the
previous set entry was indexed explicitly as $4$.


The attribute \XMLAttribute{range} value can be specified for the entire
colormap or on a per-entry basis.  If this is present, it is
interpreted as a numeric value and each value in the entry (or all
entries if specified for the entire colormap) is divided by that
amount.  This allows one to easily use different scales such as 0 to 1
or 0 to 100, etc. This is designed to assist when the software
creating the file does not facilitate such rescaling.

\subsection{Description}

The second of the sub-elements within the \texttt{data} tag is
a description of the datasets.

\begin{verbatim}
<description>
Physical measurements on flea beetles.
</description>
\end{verbatim}

This includes the source, any references, etc.  This is currently
free-format.  A convenient attribute is \texttt{source} which
indicates where it can be found.

\subsection{Variables}

The next section of the file contains the descriptions of
the variables.  It begins with the \texttt{variables} tag,
which must include the number of variables:

\begin{verbatim}
<variables count="3">
</variables
\end{verbatim}
%
Between the \texttt{variables} tags, the file lists the variables,
which can be continuous or categorical.  Continuous variables are
specified simply:
\begin{verbatim}
<realvariable>
 <name> tars1 </name>
</realvariable>
\end{verbatim}
% 
For categorical variables, it's also necessary to list their
levels and specify the mapping between data value and level, because
the data values in the \texttt{record} tag are numbers.
\begin{verbatim}
<categoricalvariable>
  <name> fraudp </name>
  <levels count="3">
    <level value="0">low</level>
    <level value="1">medium</level>
    <level value="2">high</level>
  </levels>
</categoricalvariable>
\end{verbatim}

%The name of the variable can be specified as the text within the
%variable tag rather than as an attribute.

%The name of the transformed variable can be specified via the
%attribute \texttt{transformedName}.

% this is obsolete, but can we specify ranges?
%The group attribute allows variables to be joined for the same purpose
%as ifentified by the .vgroups file in the old format.  Variables
%within the same group are scaled ``jointly''.

Additionally, instructions as to how to create the variable can be
specified as a programming command via the Programming Instruction (PI)
\begin{verbatim}
<realvariable>
<?R rnorm(10)>
</realvariable>
\end{verbatim}

\subsection{Records}

The next section of the file is the data itself.  The individual
\texttt{record} tags are contained within the \texttt{records}
element, which must include the \texttt{count} attribute, specifying
the number of records in the data.

\begin{verbatim}
<records count="74" color="2" glyphType="fc" glyphSize="3" missingValue=".">

</records>
\end{verbatim}
%
It may optionally include tags specifying the default color (the
index in the color table), glyph type and size, or the character to
interpret as a missing value.

The body or content of each \texttt{record} is a simply ASCII listing
of the values. Each value is separated by white space (space
character, tabs or new lines).

A record can be considered ``hidden''.  This is set via a logical
value for the attribute \texttt{hidden}.

Each record can be given an \XMLAttribute{id} attribute value.  This
is different from a label in that it is not used by ggobi instance
when displaying plots.  Instead, it is used only to uniquely identify
a record within a dataset, and it has two purposes.

\subsubsection{Linking Records}
\label{LinkingRecords}

The \XMLAttribute{id} can be used in the case where different
datasets contain different variable for the same subjects, or
for some of the same subjects.  For instance, dataset A may contain
usage data for a set of customers, while dataset B contains
demographic data for a subset of those customers.  Those datasets
will be linked for brushing and identification if they have the
same value for \XMLAttribute{id}.

\subsubsection{Edges}
\label{Edges}

The \XMLAttribute{id} is also a critical part of the specification
of edges.  In order to specify a set of edges, or line segments,
between pairs of points in a dataset, it's necessary to define
a second dataset whose records have \XMLAttribute{source}
and \XMLAttribute{destination} tags, like this:

\begin{verbatim}
<record source="0" destination="2"> </record>
\end{verbatim}
% 
The values used for \XMLAttribute{source} and \XMLAttribute{destination}
must correspond to the \XMLAttribute{id}s specified elsewhere.

A record which includes this edge specification can also include
any other attribute or property of a record:
\begin{verbatim}
<record source="0" destination="2" color="3"> 4.2 6 9.6 </record>
\end{verbatim}
%
If other data values are present, this dataset is like any other
dataset, and the values can be displayed in scatterplots, 
parallel coordinate plots, and so on.


\section{Using XML Files}
To use data that is contained in an XML file, invoke ggobi with no
command line flag, or with the command line flag \texttt{-x}.

% Does it do this now?
In the future, ggobi could be smart enough to include the detection of
the XML format.

\section{Conversion of Old Files}
The distribution contains an application named xmlConvert that can be
used to read datasets provided in the old file format to XML.  This
can be used by specifying the name of the file containing the
old-style data in the same manner as ggobi expects.
The output is written to standard output
and can be redirected to a file using basic shell commands.
For example,
\begin{verbatim}
  xmlConvert data/flea > flea.xml
\end{verbatim}
In the future, we will support writing the output to a file. (We need
to process the command line arguments and look for a -o flag).


Note that this dynamically loads the libraries libGGobi.so and
libxml.so.  Thus the directories that contain these libraries must be
referenced in the environment variable 
\Env{LD_LIBRARY_PATH}.
Alternatively, the makefile can be edited to statically link these
libraries.


\section{Compilation}
To activate the XML mechanism, define the variable
\Env{USE_XML} in  local.config.

When we use autoconf, this can be done by
\begin{verbatim}
  ./configure --with-xml
\end{verbatim}

This requires the XML parsing libray libxml (also know as gnome-xml) by Daniel Veillard.
(\texttt{Daniel.Veillard@w3.org}).
See \Escape{http://xmlsoft.org}
\section{References}
The XML Handbook, Charles F Goldfarb and Paul Prescod
 Prentice Hall.

http://www.w3.org/XML

\end{document}
