\documentclass{article}
\input{WebMacros}
\usepackage{fullpage}

\begin{document}


\section{API Routines}
In this section  we describe the different
R/S functions that one can invoke to query
and modify the state of the GGobi session.
There are some common concepts shared between
most of these functions.

Firstly, there are some global variables that are properties of the
entire ggobi system.  These are things such as the color table, the
glyph types and sizes.  These can be considered fixed, but queryable.

Within a ggobi session, one can view different datasets
simultaneously. Each dataset corresponds to its own ggobi data
instance within the single session.  (Note that this is different from
xgobi in which only one dataset could be introduced into the a
process.)

Within each ggobi instance, one can have multiple 
windows, also known as displays.
Each display has one or more plots within it.
The plots can be a
\begin{itemize}
\item $1$ variable plot
such as ASH plots,
\item  scatterplot of two variables
\item scatter matrix of 2 or more variables
\item parallel coordinates plot of $k$ variables.
\end{itemize}

This hierarchical setup is displayed in the following
figure.
\begin{center}
\pstree{\Tr{Session}}{%
 \pstree{\Tr{dataset}}{%
    \pstree{\Tr{display}}{
      \Tr{plot}
      \Tr{$\vdots$}
      \Tr{plot}
    }
 }
 \Tr{$\ldots$}
 \pstree{\Tr{dataset}}{%
    \pstree{\Tr{display}}{
      \Tr{plot}
      \Tr{$\vdots$}
      \Tr{plot}
    }
 }
}  
\end{center}

When operating with multiple datasets in the ggobi session at any one
time, it is important to identify to which instance a command should
be directed.  For example, if we wish to get information on the
variables within a particular instance of a ggobi instance, we can
call \SFunction{ggobi.getVariableNames}. However, we must
specify the ggobi instance in question. All the functions
that operate on a ggobi instance have an optional argument,
\SArg{.ggobi}, which is an integer.
If omitted, this defaults to the currently active
ggobi instance.

This allows us to send repeated commands easily to a ggobi instance.
\begin{verbatim}
> ggobi(args="../data/tes") 
> ggobi(args="../data/flea") #
> ggobi.getVariableNames()
[1] "tars1" "tars2" "head"  "aede1" "aede2" "aede3"
> setDefaultGGobi(1)
> ggobi.getVariableNames(.gobi=2)
\end{verbatim}
Additionally, commands intended for another can be directed without
setting that instance to the default.

By default, this is 
This identifies 


have multiple ggobi
instances in existence at any one time.
Each ggobi has its own


\subsection{getData}
Since the user can load data in a non-programmatic way, there is an
opportunity for XGobi to have data that has not been introduced to the
R/S session.  As a result, we need a mechanism to retrieve the current
values of the active data set.  This is especially important if we
allow the data to be edited in any way.

The return value is a matrix of numeric values containing the data
values.  The names of the rows and columns are the XGobi row labels
and variable names respectively.



\subsection{getColors}


\subsection{set}

\subsection{getGlyphs}


\subsection{getSelectedIndices}
This needs attention to access the correct
variable in the xgobi structure.


\subsection{getNumGGobis}
Returns the number of ggobi instances within this session.  Each ggobi
instance has its own data set. Of course, two instances may have the
same dataset, but they are independent.  We may allow them to be
linked in the future.  Again, the ability to specific which displays
and even plots within them inter- and intra- ggobi instance is most
useful in a programmatic interface.

\subsection{getDefaultGGobi}
Return the index of the currently active
or default ggobi instance.
This is the ggobi instance which 
receives the commands by default.


\subsection{setDefaultGGobi}

\subsection{getRowNames}
\subsection{getRowNames}
\subsection{getSmootherFunction.ggobi}
\subsection{ggobi}
\subsection{ggobi.getActivePlot}
\subsection{ggobi.getCurrentDisplayType}
\subsection{ggobi.getDescription}
\subsection{ggobi.getDisplayOptions}
This returns the current settings of the values that control how new
plots are displayed.  Each ggobi instance has a set of options which
are used to create new plots.  These are logical values that govern
the appearance of these plots.  This function returns a named vector
of those values currently in existence.



\subsection{ggobi.getDisplays}
\subsection{ggobi.getFileName}
\subsection{ggobi.getGlyphs}
\subsection{ggobi.getSegments}
\subsection{ggobi.getVariableIndex}
\subsection{ggobi.getVariableNames}
\subsection{ggobi.getViewTypes}
\subsection{ggobi.parcoords}
\subsection{ggobi.scatmat}
\subsection{ggobi.scatterplot}
\subsection{ggobi.setActivePlot}
\subsection{ggobi.setData}
\subsection{ggobi.setDataFile}
\subsection{ggobi.setDataFrame}
This sets the dataset for the particular ggobi instance to the
contents of the dataframe.  Each column in the data frame corresponds
to an observations. The row names of the data frame are used as the
observation labels in the ggobi displays.

\subsection{ggobi.setDisplayOptions}
This controls the characteristics
of plots that are subsequently
created within the specified ggobi instance.
These control issues such as 
\begin{itemize}
\item whether lines are drawn, 
\item lines are directed or undirected
\item missing values are displayed,
\item grid lines are drawn on the plot(s),
\item axes are shown and/or centered,
\item double buffering is used,
\item the plot(s) are linked to others.
\end{itemize}



\subsection{ggobi.setRowNames}
This sets the observation identifiers for the 
specified rows in the ggobi instance dataset.
These are used in identifying points within 
plots.

The return value is a vector of the previous labels for the
specified rows/observations.

\subsection{setSmootherFunction.ggobi}




\subsection{ggobi.symbol}
This is of little or no interest to the regular user.  It is merely a
function that takes the name of a C routine and maps it to the name of
a routine in the GGobi chapter.  It does so by prefixing the name with
the ``unique'' identifier \Escape{RS_GGOBI} in an effort to avoid
symbol conflicts with other libraries.

\end{document}
