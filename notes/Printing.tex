\documentclass{article}
\usepackage{fullpage}
\usepackage{times}
\input{WebMacros}
\input{CMacros}
\usepackage{hyperref}
\begin{document}
\begin{abstract}

  This describes a simple approach to handling printing in ggobi. This
  scheme allows it to be easily overridden by a host application in
  which ggobi is embedded.  This also provides a way to handle global
  options and instance-specific print settings.
  The default setup outputs only \href{http://www.w3.org/Graphics/SVG}{Scalable Vector Graphics (SVG)}
  format. The R-ggobi interface provides a more direct way of
  rendering ggobi plots in arbitrary graphics devices. This allows us
  to directly generate Postscript, PDF (Portable Document Format),
  PNG (Portable Network Graphics),  JPEG (Joint Photographic Experts Group).
\end{abstract}

\section{Basic Setup}

We define a structure, \CStruct{PrintOptions} that represents print
settings.
This holds things such as 
\begin{itemize}
\item the width and height of the page,
\item the background and foreground colors,
\item the file name to which the output
should be directed,
\item $\ldots$
\end{itemize}

Each \CStruct{ggobid} has its own version of this.  (It may inherit
the values from a/the previously created ggobi.)  The user can set the
values of this ``global'' settings. They are global in the sense that
they will apply to all displays within this ggobi instance.

The user prints a particular display via a routine in the API or via
the Print menu item in the File menu of the display's menu bar.  This
interactive approach typically brings up a dialog that allows the user
to specify instance-specific values of the print options. The default
values presented in the dialog are taken from the associated
\CStruct{ggobid}'s ``global'' value.


\section{Setting the Global Print Options}
The user can click 

\section{Host Application Printing}
In R, we 


\end{document}
