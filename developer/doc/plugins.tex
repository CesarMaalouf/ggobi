\documentclass{article}
\usepackage{hyperref}
\usepackage{comment}
\input{pstricks}
\input{WebMacros}
\input{CMacros}
\input{MakeMacros}
\input{XMLMacros}

\def\XMLAttr#1{\textsl{#1}}

\def\file#1{\href{#1}{\Escape{#1}}}

\usepackage{times}
\usepackage{fullpage}


\title{Extending GGobi's Functionality with Plugins}
\author{\href{http://cm.bell-labs.com/stat/duncan}{Duncan Temple Lang}\\
\href{http://www.research.att.com/~dfs}{Debby Swayne}
}
\begin{document}
\maketitle
\begin{abstract}
  This is a note on the new plugin mechanism added to GGobi. The goal
  is to allow optional facilities to be loaded into the running GGobi
  application.  This mechanism allows us and other users to provide
  extensions to GGobi without directly altering the code. These might
  introduce new plot types, ways to read data, or auxiliary tools to
  view and manipulate data.  One existing plugin provides a data grid
  for viewing data. A different class of plugins -- input plugins --
  provide ways for reading data from different input sources.  The
  Postgres and MySQL input plugins allow users to retrieve data from
  database servers.
\end{abstract}

\section{Motivation}
New software and functionality will always become available at a
faster rate than we can add to GGobi. Data will be made available in
different formats or from different applications. Even if we had
resources to support these features, some would not be of main-stream
interest to warrant adding to GGobi at the risk of complicating the
user interface, configuration, and general development. Instead, we
(as developers of GGobi) and other developers may want to add to
GGobi.  Since GGobi is an Open Source project, we encourage and expect
this to happen.  Ideally, we would not have to learn the details of
the internals of GGobi, and nor would we have to track the development
of the GGobi code and repeatedly add our extensions to that moving
target.  Specifically, we would like to be able to extend GGobi's
functionality without altering the code of GGobi itself.
Additionally, we may want to optionally include certain features,
activating them at run time.

In order to provide a convenient way for developers to add
functionality to GGobi, we have developed a ``plugin'' mechanism.
Developers create a plugin that can be dynamically loaded into GGobi
at run time. The plugin can provide functionality to GGobi, access the
state of the GGobi instance with which it is associated, and even
modify the user interface. 
A second class of plugins allows data to be read from different
input sources. These allow one to have GGobi read data from
data sources other than the XML and ASCII formats 
built into GGobi. One might read from centralized databases, binary 
formats for efficient access, streaming data from a device, and so on.
These are described in section \ref{sec:InputPlugins}.

We expect plugins to be used for adding functionality and also
supporting different input and output formats.  Together with the
ability to treat GGobi as a library and embed it in other
applications, there are many flexible ways to deploy GGobi and
integrate it with other software.  While we support plugins for GGobi,
GGobi also acts as plugin for Gnumeric, the Gnome spreadsheet.


\begin{comment}
\section{Hooks and Events}
Allow the plugin to write to the stored data
in the ggobirc.  
\end{comment}

\section{API}
Because GGobi does not and cannot know much about plugins, the
requirements imposed by GGobi for a plugin developer are few.
First, one should provide a sample XML stanza that can be included in
an initialization file to register the plugin when GGobi starts.  This
should be something like the following in figure \ref{xmlFormat}.

\begin{figure}[htbp]
  \begin{center}
    \leavevmode
\begin{verbatim}
 <plugin name="Data Viewer" load="immediate">
   <description>Displays datasets in data grid.</description>
   <author>Duncan Temple Lang</author>
   <options>
    <named>
     <background>red</background>
     <nrows>20</nrows>
    </named>
    <args>
      <arg>--no-menu</arg>
    </args>
   </options>
   <dll name="plugins/DataViewer/DataViewer">
     <init onCreate="addToMenu"  onClose="closeWindow" />
   </dll>
 </plugin>    
\end{verbatim}
    \caption{Plugin Registration Format}
    \label{fig:xmlRegistration}
  \end{center}
\end{figure}

This stanza should be placed within the \XMLTag{plugins}
section of the initialization file.

GGobi currently communicates with the plugin using $5$ entry points.
Each of these are identified in the \XMLTag{init} element of the
plugin registration.  One associates the name of a C routine in the
plugin's shared library with the corresponding method name in this XML
element.
 
A plugin must provide a value for the \XMLAttribute{onCreate} method.
The remainder are optional.
The 
\begin{table}[htbp]
  \begin{center}
    \leavevmode
    \begin{tabular}{lp{4in}}
onCreate & 
this is called when a GGobi instance is created
and an instance of the plugin is being created
and associated with this GGobi instance.
This can be used to add items to the GGobi interface,
initialize data structures specific to this instance
of the plugin, and so on. \\
onClose & this is called when a GGobi instance
is destroyed/closed. It allows the plugin instance
to take any steps needed to release resources
such as freeing memory, saving state, etc. \\
onUpdateDisplayMenu & 
this is called each time the Display menu
is re-configured which happens when a
dataset is added or removed from the GGobi instance. 
\\
onLoad & this routine is called when a the plugin
is first loaded. This can be used
to perform global initialization 
for the plugin, such as computing values
that are the same across different GGobi
instances, initializing third party libraries, etc.
\\
onUnLoad & this is called when the plugin is de-activated
or unloaded from the GGobi session.
(This may not be called.)
    \end{tabular}
    \caption{Plugin entry points}
    \label{tab:entryPoints}
  \end{center}
\end{table}

\subsection{``Command Line Arguments'' for Plugins}
The user can specify options in the XML stanza to paramterize or
customize the run-time behaviour and appearance of the plugin.  These
are specified within an \XMLTag{options} tag and come in two forms.
The first is a collection of named arguments given as a sequence of
tags each containing some text.  The idea is quite simple. These
arguments are stored in a table where the value within the tag is
stored by name, with the associated name being taken as the name of
the XML tag.  These tags are located within the \XMLTag{named} element
and are analogous to the command line arguments of the form
\begin{verbatim}
--name=value
\end{verbatim}
or to property files of the form
\begin{verbatim}
name: value
\end{verbatim}
In our example above (figure \ref{fix:xmlRegistration}), the table
would have two entries with names \texttt{"background"} and
\texttt{"nrows"}.  The values \texttt{"red"} and \texttt{"20"} are
the associated values stored in the table.

The second form of arguments is a simple list of values, without
names.  These are given as a sequence of \XMLTag{arg} elements within
a \XMLTag{args} tag inside the \XMLTag{options} element.  The
values are collected into a linked list.

Both collections of arguments (named and unnamed) are available to the
plugin when it is initialized. They are stored in the
\CStruct{GGobiPluginDetails} structure associated with the plugin.
Each plugin can process them in any way it pleases.  Additionally, the
plugin is entitled to re-read the initialization file and find its own
stanza should it want to process the user's specifications itself.
This might be useful if the plugin wants additional input in an
extended XML format.

Specifying the values in the initialization file works well when the
values are to be used across different GGobi sessions.  However, when
we want to customize the particular session we have to do so through
GUI controls, etc.  However, the plugins do have access to the command
line arguments used when invoking GGobi.  So, a plugin can process
these to override any values in the XML options.  This is highly
plugin-specific and again, a plugin can do what it wants with these
inputs.  The C routine \Croutine{getCommandLineArgValue} is available
to get the value of a \verb+--var=name+ by giving the key \verb+key+
in much the same way as looking up the table of named arguments.

The command line arguments are stored in the
\CVariable{sessionOptions} variable, having been processed by both
GGobi and Gtk with some values potentially removed.  So name conflicts
for arguments is a problem.  We recommend moving to a convention that
identifies that a collection of arguments are for a particular plugin
by using the flag
\begin{verbatim}
 --myPlugin
\end{verbatim}
where \textit{myPlugin} is the name of the plugin.

For an example of processing XML-supplied arguments and the GGobi
command line arguments, see the Postgres and MySQL input plugins and
the JVM plugin.




\section{Adding Controls \& Menus}
Many plugins are dormant extensions to GGobi's functionality. In other
words, when a plugin instance is created, it doesn't create plots,
manipulating data, or generally doing its intended task.  Instead, it
typically just adds items to the GGobi user interface which the user
can choose to employ.  In order to do this, the plugin may want to add
menus and menu items to the main window of the GGobi instance
associated with the plugin.  To help do this,
GGobi provides two routines in its API:
\Croutine{GGobi_addToolsMenuItem}
and \Croutine{GGobi_addDisplayMenuItem}.

\begin{description}
\item[\Croutine{GGobi_addToolsMenuItem}]
This adds a menu item to the Tools menu in the main
window of the GGobi instance.
This returns the widget for the menu item and
then the plugin can register its own event handlers
with that widget.
The caller need only specify the text to be displayed
on the menu item.
\begin{verbatim}
GtkWidget *GGobi_addToolsMenuItem (const char *label, ggobid *gg)
\end{verbatim}

\item[\Croutine{GGobi_addDisplayMenuItem}] This is typically used to
add a plot type to the Display menu.  This is useful in the
\XMLAttribute{onUpdateDisplayMenu} plugin routine which is called when a
dataset is added or removed from the GGobi instance.  In the future,
this will handle multiple datasets and an item for that plot type with
a cascading menu for each dataset.  The signature of the routine will
change to support this.
\end{description}


\subsection{Dependencies between Plugins}
As we develop more plugins, dependencies of one on the existence of
another will emerge. There are slightly complicated issues of
dynamically loading code that can be seen by another plugin in a
portable fashion.  Leaving this aside, we have a more elementary
problem that we want to ensure that one plugin is started before
another is started.  (Simply link against a common library!) For
example, any of the Java plugins must ensure that the JVM is running
before access to the Java classes is made.  To do this, we support a
list of plugin dependencies in the XML specification of a plugin.  One
provides a list of the plugins which must be loaded and started before
this plugin can be.  Each required plugin is identified by an XML
element of the form
\begin{verbatim}
 <dependency name="name of other plugin"/>
\end{verbatim}
And each of these elements is enclosed within the
\XMLTag{dependencies} element within the \XMLTag{plugin} tag.  The
value of the \XMLAttribute{name} attribute should match the value of
the \XMLAttribute{name} attribute used to identify the other plugin
(i.e. in the \XMLTag{plugin} tag).

When we attempt to load a plugin, we first iterate over the plugins on
which it depends and ensure that they are loaded and initialized (i.e.
call the routine identified by the plugin's \XMLAttribute{onLoad}
attribute).

Note that for Java plugins (identified by the \XMLAttribute{language}
attribute of the \XMLTag{plugin} element being \texttt{"java"}), the
dependency on the JVM is automatically generated.


\section{Examples}

There are currently several examples of plugins in the GGobi source
within the \dir{plugins} directory.  The \dir{DataViewer} plugin
provides a datagrid for viewing the data in a GGobi instance.  This
adds a new menu to the GGobi main window/control panel.  The user
creates the data viewer via that menu.

The \dir{CategoricalPlots} is currently a stub in that it does not yet
create plots. (We are waiting to work out all of the details using
Heike Hoffman's work to find out what needs to be done for plugins to
work this way.) However, it does illustrate a way of adding menu items
to the Display menu for the different datasets.

A test/diagnostic plugin is available in the \dir{TestPlugin}
that illustrates how to use each of the $5$ callbacks.

The JVM plugin provides Java support for GGobi, allowing one to
implement plugins using the Java programming language.

\section{Potential Plugin Projects}

\begin{description}

\item[Data grid] Extend the data-grid to allow color-coding the rows,
identification, editing, etc.  See the GGobi plugin for Gnumeric for a
potentially ``better'' way to do this.

\item[Categorical variable data plots]
Heike Hoffman is working on
adding plots for categorical variables.

\item[Maps]
Drawing maps  as backgrounds for plots,
regions within plots, etc.
connecting to spatial data would be interesting.

\item[Statistical Computations] We might embed R, Octave, etc.  to
provide statistical and mathematical functionality.  As we

\item[Trees]
We can add support for classification tree software
written in Java by embedding the JVM within GGobi.

\item[Interpreters]
Java (done), Python, Perl.

Given Java's support for high-level network access, database
connectivity, XML parsing, etc., coupled with its relative ease of
programming and portability, it may be convenient to develop input
plugins using Java.  This can be done relatively easily given a Java
Virtual Machine plugin and some support classes provided by GGobi
itself. When combined with the Gtk bindings for Java, Java plugins
will be able enhance GGobi without the need for developing in C.


\item[Help System]


\item[Data description] Viewer for descriptions of datasets that are
available from the XML format.

\end{description}

\section{Input Plugins}\label{sec:InputPlugins}
It is simply a fact that data will be stored in a variety of different
formats. Since GGobi reads data only in a couple of specific formats,
the onus is on the user to translate their data into one of these
formats and then run GGobi. This can be, at the least, tedious and at
worst infeasible if the data is available only as dynamic stream.  To
deal with this, we provide an input plugin mechanism that allows one
to extend the data sources from which GGobi can read its input.
Essentially, one can write a small amount of code that reads
data in a particular manner and add have GGobi use this 
by registering it as an input plugin.

The basic idea of the input plugin mechanism is as follows.  One
registers the plugin within the \XMLTag{plugins} tag in the
initialization file.  The format of the XML entry for an input plugin
is defined in the DTD for the initialization file
(\file{GGobiInit.dtd}) and illustrated in figure \ref{fig:input}.

\begin{figure}[htbp]
  \begin{center}
    \leavevmode
\begin{verbatim}
 <inputPlugin name="postgres" interactive="true">
  <description>Reads data from a Postgres DBMS</description>
  <modeName>postgres</modeName>
   <author>Duncan Temple Lang</author>
     <dll name="plugins/Postgres/Postgres"> 
       <init description="postgres_input_description" />
     </dll>
 </inputPlugin>    
\end{verbatim}
    \caption{Input plugin specification}
  \end{center}
\end{figure}
The simplest thing to do is copy the \file{plugin.xml} file from one
of the existing input plugin directories and edit that.

The format is very similar to that of the regular plugins.  We
identify the plugin by name, and provide information (i.e. description
and author) which can be displayed in the plugins display tool.  Then
we provide information about where the shared library/DLL is located
and identify the symbol that is the primary entry point used by GGobi
to use the plugin.

  There are two main differences between this and the regular plugin.
First, one can specify the \XMLAttribute{interactive} attribute for
the \XMLTag{inputPlugin}.  The flag is intended to indicate that the
plugin is capable of asking the user for information about how to get
data. For example, the database plugins provide a GUI in which the
user can input a query to be sent to a database to get data.  The
\XMLAttribute{interactive} attribute is of interest when GGobi is
started with no data input specified (i.e. without a data file being
given on the command line).  In such a case, we look through the list
of input plugins for the first one that is interactive.  When we find
it, we run it and have it figure out how to populate the GGobi
instance with data.

The second addition is the \XMLTag{modename} element.  The string
inside this tag can be used to activate/deploy this plugin from the
command line. One can specify the mode for reading data using the
command line flag \flag{datamode} when starting GGobi.  The value for
this flag should be the value of the \XMLTag{modename} tag for one of
the input plugins.  When GGobi starts and finds that the
\flag{datamode} flag has been specified, it iterates over the input
plugins and, for each plugin, compares the \XMLTag{modename} to the
\flag{datamode} value. When it finds a match, it uses that plugin to
read the data.

\begin{verbatim}
 ggobi -init notes/ggobirc -datamode postgres
\end{verbatim}

\textit{Currently, the \flag{datamode} only applies to input plugins.
In the future, we will support all of the built-in data modes in the
same way.}


\subsection{Available Plugins}
There are currently four input plugins available for GGobi.  One is a
simple test of the mechanism that generates a dataset with $3$
variables and $10$ observations.  The values are generated by the C
routine \Croutine{rand} just for variety.

The more useful plugins are the Postgres and MySQL input plugins.
Each of these allows the user to connect to a relational database
management server, either running locally or remotely on another
machine, and issue an SQL query to get the data for GGobi to display.
The information about the location of the database server, the name of
the database, the user login name and password, the SQL query, etc.
are typically entered interactively via a simple GUI provided by
GGobi.  Alternatively, if the \SharedLibrary{Properties} library is
available, some or all of these values can be specified in a file.
More information is available in \file{DBMS.pdf}.

A fourth plugin -- \dir{FileSelector} -- allows the user to select a
file interactively by browsing files using a graphical directory tree
and an icon browser for files.  This uses the file selector from the
the Gtk+Extra library (\url{http://gtkextra.sourceforge.net}).
Currently this only works for XML files, but this is easily extended
to others.

One motivation for input plugins is the ability to handle large
amounts of data in intelligent and efficient ways.  XML provides us
with a powerful formatting scheme.  However it is verbose and requires
extra resources to process. In contrast, a specialized binary format
can reduce the storage space requirements and be amenable to rapid
processing that exploit knowledge of the specific structure of the
format. It is relatively easy to develop a new plugin that can read
these types of inputs.  (We also need to provide output plugins that
serialize data in a particular format.)


It would be useful to have a plugin for reading 
\begin{itemize}
\item Gnumeric spreadsheets,
\item Excel spreadsheets,
\item HDF5 files,
\item S-Plus and R data frames,
\item Oracle databases,
\item data serialized by other applications,
\item streaming data,
\item Genetic or network configuration and performance data stored in XML.
\end{itemize}

One basic way to get access to R \& Octave data is to embed the entire
engine for that system into GGobi and then to call the appropriate
functions to load the data and then convert it to GGobi.  Of course,
we can run GGobi within R and pass data to GGobi directly from R.


The next round of development for input plugins will possibly allow a
plugin to add itself to the Open dialog so that the user can specify
the format or mechanism to use for reading the file.  We may also add
output plugins that allow serializing the data using a particular
format and mechanism. Gnumeric provides a similar mechanism that we
will probably mimic.


\subsection{Technical Issues about Input Plugins.}
When GGobi starts, it processes the command line arguments and then the
initialization file.  It is at this point that we register the
plugins, and specifically the input plugins.  Next we attempt to load
the data.  The heart of this from the perspective of identifying the
plugin is \Croutine{fileset_generate} in \file{make_ggobi.c}.  This is
responsible for creating an instance of the \CStruct{InputDescription}
with the appropriate data for reading the data.  The user identifies
the plugin by providing a value for the \flag{-datamode} argument.
\Croutine{fileset_generate} reads this value (stored in the session
options) by looking through the known input formats.

\subsection{Java Input Plugins}

The JVM (Java Virtual Machine) plugin allows us to implement plugins,
including input plugins, by writing them in the Java language.  The
basic idea is quite simple and involves implementing a particular Java
interface.  In practice, you might even be able to copy one of the
Java classes in the GGobi distribution that implements an input
plugin.

The simplest Java input plugin works as follows.  One registers a Java
class as the implementation of the plugin. When the plugin is needed,
a new instance of that class is created.  Currently, the constructor
is called with the name of the file and of the data mode for which it
has been requested. (Both of these might be \JNull.)  Then, the
\JMethod{getSourceDescription} method is invoked.  This is called with
the value \JTrue{} indicating that we want a brief description that
can be used as text in menus, etc.

At this point, control returns to GGobi, but not for long. It then
asks the Java object to read the data.  In this simple plugin setup,
the Java object is supposed to read and store the data and then GGobi
asks for the different pieces, i.e.  dimensions, variable names and
records. In the future, more efficient input plugin interfaces
will be made available.

There are four methods that GGobi invokes when it extracts the data
from the Java input plugin object. These are \JMethod{getNumRecords},
\JMethod{getNumVariables} and \JMethod{getVariableNames}.  They are
defined as part of the \JInterface{InputPlugin} interface and are
expected to do what the names suggest.  The Java-GGobi plugin
mechanism takes care of creating a new dataset for the GGobi instance
using the dimensions it retrieves from the Java object.  Then it
copies the variable names.  Finally, it iterates over the records and
calls the \JMethod{getRecord} method that is defined in the
\JInterface{RecordWisePlugin} interface.  GGobi copies the values to
the \CStruct{datad} and finishes by calling \Croutine{start_ggobi}.

We will provide a collection of different interfaces which range from
the minimal methods needed to read data to reading coloschemes, etc.
A plugin implements the relevant interfaces and we check whether the
plugin does in fact implement these. If so, we invoke the relevant
methods.  In the future, the Java code will be able to access the
GGobi instance and the associated C routines directly and can ``push''
data into GGobi rather than cumulating and waiting for GGobi to
``pull'' it out.

One can use the Java bindings to the Gtk libraries to create interface
components that have the same look-and-feel as the rest of GGobi.

\subsection{R}

In the same way that we embed the JVM, we have also embedded the R
interpreter.  This allows one to develop plugins in the R language,
creating regular and input plugins with R functions.

\subsection{Perl \&  Python}

It is also possible to embed the Perl or Python interpreters into
GGobi (as well as GGobi in either of these languages).  Then we can
write plugins by defining classes in one of these languages by
extending a basic plugin class and overriding the relevant methods.
There is now the beginnings of a Perl language plugin.



\section{Creating a Plugin}

\subsection{Developing an Input Plugin}

The \dir{Postgres} and \dir{TestInput} directories in the
\dir{plugins} directory provide examples of two different styles of
input plugin. The Postgres plugin is more complex because it is
event-driven, because it presents a dialog looking for input from the
user before reading the data.  In this way, the GGobi instance is
created and displayed with no data and then populated after the data
is read. This is a little more indirect than the TestInput plugin.
For that plugin, the data is read before the GGobi instance is
displayed and this behaves just like reading data from an XML or ASCII
file.



\subsection{Installation}

First, create the usual files.
\begin{table}[htbp]
  \begin{center}
    \leavevmode
    \begin{tabular}{ll}
\file{configure.in} & the autoconf script that
attempts to determine the parameters for this system.
\\
\file{local.config.in} & 
a simple file that is included by \file{GNUmakefile}
that defines the variables specific for this system. \\
\file{GNUmakefile} &  the top of this should contain
\vbox{
\begin{verbatim}
ifdef ADMIN
 include Install/GNUmakefile.admin
endif
\end{verbatim}
}\\
C source &  List these as the value of the variable \MakeVariable{SRC} in 
\file{GNUmakefile}. \\
\dir{Install}/GNUmakefile.admin & 
 copy this from an existing plugin and change the
value of the \MakeVariable{DIR_NAME}
variable.
\\ 
\file{plugin.xml} & \\
    \end{tabular}
    \caption{Basic files for a plugin}
    \label{tab:pluginFiles}
  \end{center}
\end{table}


To add a plugin to the distribution, one appends the name of the
plugin directory to the definition of the variable
\MakeVariable{PLUGIN_DIRS} in the \file{Install/GNUmakefile.admin}.


\section{Non-\C/\Cplus{} Plugins}

It is often convenient to develop plugins that are implemented not in
C, but a high-level interpreted language.  An obvious example is
implementing an input plugin in Java, Python or Plugin.  In such a
case, we have a well defined \C-level interface and we can invoke the
corresponding methods in the class of the corresponding language.

We get the meta- or language-plugins to process their own sub-plugin
information. This forces the library to be loaded, but the 
language not necessarily started!

\section{Plugin Arguments}
It would be convenient to be able to parameterize a plugin at
run-time, or specifically when it is created, by providing parameters
to it.  We can do this in the initialization file using special XML
tags within the \XMLTag{plugin} element.  We identify two types of
arguments: named and unnamed.  Unnamed arguments are specified within
a \XMLTag{args} tag with each value enclosed within an \XMLTag{arg}
element.  These are given to each plugin instance as
individual string values.
For example,
\begin{verbatim}
 <plugin>
  <options>
   <args>
    <arg>-init</arg>
    <arg>value</arg>
   </args>
  </options>
 </plugin>
\end{verbatim}
is given as two separate elements
\texttt{"-init"} and \texttt{"value"}.

The named arguments are enclosed within a \XMLTag{named} tag, itself
enclosed within the \XMLTag{options} element.
In the example,
\begin{verbatim}
 <options>
  <named>
   <classpath>.:/home/duncan/Projects/org/..</classpath>
  </named>
 </options>
\end{verbatim}
the name-value pair is available as an entry in
a Gnome hashtable.



\section{Building plugins on Windows}
Compiling a plugin on Windows is less well organized than it might be.
However, it is quite simple if you are using the MinGW compiler.  In
the future we will provide more tools to support this. Since we
usually distributed binaries for GGobi and the plugins, this has
been less of an issue.

The steps are quite straightforward when building directly within the
developer's version from CVS (rather than the distributed source).
The ggvis plugin serves as a good example.  First, ensure that the
make variable WIN32 is defined either in the local.config file within
the plugin directory or the top-level local.config in the GGobi root
directory.  We use the CFLAGS from the top-leve GGobi
compilation. These can be dynamically determined using
\begin{verbatim}
 make cflags
\end{verbatim}
in that top-level directory.

We then create the different object (.o) files.  Now we are ready to
create the DLL. To do this we must first create a list of the exported
symbols. Since a plugin typically only makes a small number of its
symbols available to the GGobi engine (e.g. the onLoad, onCreate,
onClose, onUnload entries), it is typically easiest to create this
list by hand. (We could generated it directly from the XML describing
the plugin, but we currently don't support this.)
One creates a text file name \textit{plugin.def}
and edit it to contain something like the following:
\begin{verbatim}
EXPORTS
 addToToolsMenu
 closeWindow
\end{verbatim}
The lines after the \verb+EXPORTS+ line 
list the names of the symbols that are to be made available
to GGobi.

Give this exports file, we are ready to compile the DLL.
The following general rule 
\begin{verbatim}
$(PKG).dll: $(OBJS) $(PKG).def
        $(CC) -mwindows -s -shared -o $@ $(PKG).def $(OBJS) $(GGOBI_LIB) $(LIBS) $(XML_LIBS) 
\end{verbatim}
builds the DLL for the plugin.

Currently, we place the DLLs for the plugins in the 
\dir{plugins} directory of the GGobi installation.
The GGobirc file that lists the plugins should refer
to them within this directory, e.g. as
\begin{verbatim}
<?xml version="1.0"?>
<!DOCTYPE ggobirc>
<ggobirc>
<plugins>
<plugin name="ggvis" load="immediate"><!-- load may be on demand. -->
   <description>multidimensional scaling</description>
   <author>Deborah Swayne and Andreas Buja</author>
   <dll name="plugins/ggvis">
    <init onCreate="addToToolsMenu" onClose="closeWindow" />
  </dll>
</plugin>
</plugins>
</ggobirc>
\end{verbatim}



\section{Identifying Plugins for GGobi}
It is possible to specify plugin specifications on the command line
using the \flag{-plugin} flag.  This allows us to use plugins without
having them specified in the initialization file.  For example,
suppose we want to use a particular plugin that is described in the
file \file{myPlugin.xml} in a directory named \dir{dir} but do not
want to put it permanently in our initialization file.
We can do this by invoking GGobi as
\begin{verbatim}
ggobi --plugin=dir/myPlugin.xml
\end{verbatim}
This will cause GGobi to read the initialization file and then also
the \file{myPlugin.xml} file.  That file can have one or more
\XMLElement{plugin} or \XMLElement{inputPlugin} entries.  If the file
has multiple plugins, you should enclose them within a top-level
\XMLElement{plugins} element.

One can specify any number of plugin files on the command line,
e.g.
\begin{verbatim}
 ggobi --plugin=dir/myPlugin  --plugin=otherDir/otherPlugin.xml
\end{verbatim}
These files are processed in the order specified on the command line.


\section{Plugin Documentation}
One can and should document plugins so that users can understand them.
While one can create separate stand-alone documents, it is often
beneficial to add the documentation to the \file{plugin.xml} file
directly. Using \url{http://www.docbook.org}{Docbook}-like markup, one
can author documentation that is readily converted to HTML, PDF, etc.
We have provided a simple XSL file that converts the documentation to
HTML.  (This uses the Omegahat XSL and Docbook XSL files also.)  See
\file{plugin.xml} in the \dir{ExcelReader} plugin for an example.



\begin{comment}
\section{Issues}
Configuration

Debugging dynamically loaded code.

The plugin mechanism requires XML.
\end{comment}

\end{document}
