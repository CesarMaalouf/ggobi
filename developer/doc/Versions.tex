\documentclass{article}
\usepackage{fullpage}
\input{WebMacros}
\input{CMacros}
\title{Version Information in ggobi}
\begin{document}
\maketitle
The ggobi executable provides information about the version that was
compiled. This is useful for bug reports and also determining
information at run time, such as in the S, Perl and Python interfaces.
Reporting the version in bugs is helpful for the developers.  Finding
the particular version at run time can allow code in other languages
to behave differently for different ggobi releases.

Debby added the version information in \file{version.h}.  I have moved
it to \file{VersionInfo} and this is used in the autoconf
configuration scripts and inserted into a file \file{config.h} that is
created during the installation.  Developers working with a CVS
version of this and not performing the configuration associated with
installation will not have the new functionality. (If people want it,
let me know and I can change things.  It is not hard, I just don't
want to add extra hoops for developers to jump through.)

\section{Stand-alone ggobi}
You can get two pieces of version of information
from the stand-alone ggobi from the command line.
Use 
\begin{verbatim}
ggobi -version
\end{verbatim}
and
\begin{verbatim}
ggobi --version
\end{verbatim}
(with two '-' characters) to get the release date and the major, minor
and patch-level numbers, respectively.

\section{Embedded ggobi}

The ggobi API provides three routines to get the version information.
These are \Croutine{GGOBI(getVersionDate)} and
\Croutine{GGOBI(getVersionNumbers)}
\Croutine{GGOBI(getVersionString)}.  The first returns the date as a
string and this is the value of the \CppMacro{GGOBI_VERSION_STRING}.
The second routine returns an array of three integers corresponding to
the major, minor and patch-level values, and in that order.  The third
routine provides these numbers as a string, separating the major and
minor numbers with a period (`.') and the patch-level with a `-'.  The
return values are not copies of data but fixed constants. Do not free
these returned values.

In the \S{} interface, the function \Sfunction{getVersion.ggobi}
returns a list containing the date string, an integer vector
containing the major, minor and patch numbers
and string version of these numbers.


\section{Release-time Duties}

When releasing a version, please remember to update the VersionInfo
file.  You should increment the \CppMacro{PATCH_LEVEL},
\CppMacro{MINOR_VERSION} or \CppMacro{MAJOR_VERSION}.  Also, set the
\CppMacro{GGOBI_RELEASE_DATE}.

In the future, I may make the \file{VersionInfo} file automatically
generated to update the date information.  In that case, we would edit
the \file{VersionInfo.in} and not the \file{VersionInfo} file.

\end{document}
