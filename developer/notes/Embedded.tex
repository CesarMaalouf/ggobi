\documentclass{article}
\usepackage{fullpage}
\usepackage{pstricks}
\input{pst-beta}
\input{WebMacros}
\input{CMacros}

\def\SharedLibrary#1{\textsl{\textbf{\Escape{#1}}}}
\def\CPPMacro#1{\CPP{#1}}

\title{Embedding GGobi in Applications: \\
GGobi as a library}
\author{Duncan Temple Lang\\
Debby Swayne}
\begin{document}
\maketitle

\textit{This is an imcomplete and early description of embedding GGobi
  in other applications. The API has been changed, the descriptions
  lacking, etc. This will be enhanced in the future.  Please send mail
  to \texttt{ggobi-help@ggobi.org} for information and that will
  encourage us to update this document.}

This describes the basic setup for embedding GGobi/XGobi in another
application.  It relates specifically to R and S, but should be of
some value for other applications.

The document describes the necessary structural support for creating
an embeddable facility.  Next, it describes the API for communicating
with this embedded application.  The API allows the user to query the
current state of the ggobi session, and also programmatically set much
of the state.


\section{Embedding Structure}
The first thing we need is the ability to simply create the interface.
Having GGobi allow creation of the interface without any data (either
a file or actual values) allows us to create the interface and then
specify the data. This allows us to treat the specification
of the data as a second step in the initialization
and implement that stage using a user-level (in the R/S language)
function. Hence we don't have to tie the initialization
to a file name or dataset.

The ggobi session is initialized via the routine \Croutine{init()}.
This takes the name of a file.  Alternatively, we can pass a matrix of
data and variable names to the routine \Croutine{}.


Ideally we would be able to use the different elements of ggobi
independently. For example, we might create an ash display and
separately a scatterplot.  This requires the separation of global
variables.

\section{Event-Loop Integration}
One of the key aspects of embedding a GUI tool into an environment
such as R or S is that we can merge the event models for the two.  R
and S are interactive environments which wait for input from the user
and operate within a read-eval command loop.  The X11 device and all
other sources of events are merged with this via the system
\Croutine{select} routine. To allow this, the event operations must
support the basic operations of 
\begin{itemize}
\item providing a file descriptor on which input can be detected
to indicate the existence of events of interest for that 
source
\item consuming all the available events without blocking
waiting for more.
\end{itemize}
The X11 library provides these facilities via
the \CPPMacro{ConnectionNumber} macro
and \Croutine{XNextEvent},
\Croutine{XtDispatchEvent}

For GTK+ and GDK, the following snippet of code
can be used as an event handler procedure with
the R input source management.
\begin{verbatim}
  while(gdk_events_pending()) {
    gtk_main_iteration_do(false);
    ctr++;
  }
\end{verbatim}
The routine is registered for the file descriptor computed from
\CPPMacro{ConnectionNumber} called with the value of
\CVariable{gdk_display}. This latter variable is available by
including \file{gdk/gdkx.h}. ({\red Something will need to be done for
windows.})


At present, the above code does not quite work to handle all events.
Specifically, when one sets the dataset programmatically,
the variable selection panel is not updated appropriately.
The widgets are created, but not exposed.


Timer functions need to be added to the R, not Glib's, event loop.  (R
needs to handle these timers along with the select call.)


\section{Warnings}
Output that goes to standard error in GGobi should now go through
\Croutine{GGobi_Warning}.  This allows R/S to report it in its
traditional way, namely via the \SFunction{warnings}.

\Croutine{g_printerr} and \Croutine{g_print} need to be changed.


\section{Installation}
The makefile contains a rule for creating a shared library,
\SharedLibrary{libGGobi.so}.


\section{Calling R functions}
While the embedded Ggobi in R allows R users to avail of functionality
provided by Ggobi, the reverse direction is also feasible.  Ggobi can
use R's capabilities to implement certain functionality. Additionally,
we can allow users to specify interpreted R functions to be used for
performing certain computations and to respond to certain events.
We outline how this works in the paragraphs below.

Consider the simple problem of taking two variables, X and Y, and
computing a smoothed version of Y.  The inputs are the values of X and
Y and the output is a vector of the same length as Y (and hence X)
with the smoothed values.
A simple R function to do this is
\begin{verbatim}
 function(x, y) {
  predict(loess(y ~ x, data.frame(x=x, y = y)))
 }
\end{verbatim}

GGobi can then invoke this function with a pair of variable values and
extract the values for the result.
The following C code illustrates how this might work.
\begin{verbatim}
 USER_OBJECT_ x, y, call, tmp;

  call = allocVector(VECSXP, 3);
  PROTECT(args);
  CAR(args) = RS_smootherFunction;
  CAR(CDR(args)) =  RS_GGOBI(variableToRS)(xindex);
  CAR(CDR(CDR(args))) =  RS_GGOBI(variableToRS)(yindex);
     tmp = NEW_NUMERIC(1);
     NUMERIC_DATA(tmp)[0] = width;
     CAR(CDR(CDR(CDR(e)))) =  tmp;
  ans = eval(call, R_GlobalEnv);

  UNPROTECT(1);
\end{verbatim}

The user can now register the function to use for smoothing, allowing
them to control how it is performed. The only constraint is that it
must be an argument that accepts 2 arguments and returns a numeric
vector of the same length as the inputs.



On difficulty we face is how to arrange to have these routines called.
One way is to them to be registered as functions to be called by
GGobi. This would make the GGobi code more complicated and indirect.
An alternative is to use the linking facilities to override or replace
routines from GGobi with those in the R.so.  When we link R.so with
libGgobi.so, we arrange to have the routines we define in this package
be invoked.  For this to happen, we obviously must define methods with
the same name. Also, they must have the same signature - parameter and
return types. Then, we implement the body of the replacement
routine in whatever manner we wish.





Example for smoothing,identify, etc.
Transforming data.



The function that is registered can be a closure
with its own ``local'' state.


Consider identifying points on a plot.  We can arrange to have an R
function be invoked when a new point is encountered or a point
selected by explicitly click on the canvas.
We can make this behave like the locator() function
in R and S in that it gathers up the collection of 
$(x, y)$ pairs for all the clicks
and returns this.

Since this is event driven, rather than a blocking call as with
locator(), we can arrange to to this with an ``object'' with its own
state.
\begin{verbatim}
 function () {
   vals <- list(x=numeric(0), y= numeric(0))

   click <- function(x, y) {
    
     vals$x <<- c(vals$x, x)
     vals$y <<- c(vals$y, y)
   }

   return(click)
 }
\end{verbatim}



Closures for mutable state and local variables.



\section{Multiple GGobi}
Occassionally, it is useful to interact with two or more datasets
simultaneously. Constrasting the same views with different data,
comparing characteristics and working on different projects
simultaneously are obvious contexts. The R-GGobi connection allows
multiple ggobi ``devices'' to be created.  These are indexed by name
and/or number and are similar to the regular graphics devices.


The following command identifies
the symbols that are static variables
\begin{verbatim}
 nm -A *.o | grep ' [a-tA-Z] ' | grep -v ' [UTtb] ' | grep -v '.*\.[0-9]*$'
\end{verbatim}

There are still a lot of statics to be dealt with.

We need to remove the macros from ggobi.h as we go to
make things simpler. Just need to find the time.

(This is just getting it down there on ``paper''....)



One of the important steps in introducing the potential for having
multiple instances of the ggobid structure in existence in an
application is the removal of global variables.  This involves passing
a reference to the particular ggobid object to each routine that needs
it. (This is very similar to threading.)  This is typically a tedious
task, but not very challenging.  (Remove the global variable and
compile to identify which routines need to have it explicitly passed
as an argument.  Then change their signatures and recompile.  Iterate
until messages are gone.)

In an event driven framework, there is an added challenge.  The
callbacks invoked by the underlying widget toolkit have predefined
signatures as defined in the implementation of the toolkit.  Hence we
cannot add an extra argument.  Instead, we must be able to have the
ggobid object passed to it or computed from within that routine.

In many cases, the user-level data that can be registered with a
widget to be passed to the routine is not used.  In such cases, we
specify that the reference to the ggobid object should be supplied as
that argument.

In other cases, the user-level data is already being used.  There are
two obvious possibilities for handling these situations.  Firstly, we
can store the reference to the ggobid object in such a way that we can
uniquely identify it and that this works for different ggobid
instances. A simple strategy is to assign it to user-level data
maintained by a widget.  When the callback is invoked, the source of
the event (a widget) is supplied as an argument. From that, we can
retrieve the particular ggobid reference associated with that
interface. Rather than setting the storing the value in each widget,
it is convenient to store it only for the top-level window that houses
each widget. In this way, it is in a single place for each widget
hierarchy.

An alternative approach is to change the type of the user-level data.
We would allocate a structure that contained the current data and the
reference to ggobid and specify this as the user-level data argument.
For example, suppose we current pass an integer as the user-leve data
to an callback. This is the case for many menu items.  Then we would
have a structure
\begin{verbatim}
typedef struct {
  int value;
  ggobid *gg_reference;
} IntegerCallbackData;
\end{verbatim}
When registering the routine, we would allocate one of these and
fill in its fields, and then specify it as the user-level argument.


This has the potential of being faster than retrieving the value from
a widget's root window. If speed becomes an issue (as in the tour
where there may be a large number of events/callbacks) we can
implement this approach.



\section{Resetting the data}

Change \Escape{vardata_init} to set the group identifier to 0.


Need to change \Escape{dataset_init} to make certain that there is only
one display in the list when this is finished..



\section{Accessing displays and plots}
Rather than relying on user gestures to select windows/displays and
plots within these and make these active, we provide a programmatic
version of this selection mechanism.  The functions
\SFunction{setActivePlot} and \SFunction{getActivePlot} can be used to
set and query which display and plot within that are active.


One can query what plots are currently in existence using the command
\SFunction{ggobi.getDisplays}.  This returns a list with an element
for each of the displays. Each description identifies the type of the
plots the display contains and a list of all the sub-plots contained
in the display. Each of these sub-descriptions identifies the
variables involved.



{\red Need to ensure that the display is updated correctly when
we set the active plot.}


\end{document}
