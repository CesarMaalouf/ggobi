\documentclass{article}

% dfs, just trying to build the pdf file
\def\file#1{\bf{#1}}

\begin{document}
There are 3 orthogonal conditions:
\begin{description}
\item[Fully qualified file \& type] 
The user specifies both the
\begin{itemize}
\item[data type]
via the flags \texttt{-ascii}, \texttt{-xml}
or \texttt{-binary}, and 
\item[file name]
in the form  of an existing file rather than
one without its extension.
\end{itemize}
\begin{verbatim}
run -ascii data/flea.dat
run -xml data/flea.xml
\end{verbatim}
\item[Type \& Partial file name]
In this case, the format/type is specified, 
but the extension is not given for the file name.
\begin{verbatim}
run -ascii data/flea
run -xml data/flea
\end{verbatim}
What can go wrong here? There is no file with the appropriate
extension for the specified type;
or there is such a file, but it is of the wrong format.
\item[Incomplete name and format] In this case, there is no file of
  this name and there is no format specified. What do we do?  We
  iterate over the different data formats we can handle and for each of the
  extensions for that type, we check for a file of that name (by appending
  each suffix/extension for that format). If we find one, we check
  that it is of that type. If not, we continue looking.
\begin{verbatim}
run data/flea
run data/flea
\end{verbatim}
\end{description}

It is, of course, possible to specify the name of a non existent file.
For example, suppose we compress the \file{flea.xml} file.  Then, that
file will be no longer present, but instead will have been transformed
to \file{flea.xml.gz}. In that case, 
\begin{verbatim}
ggobi -xml data/flea.xml
\end{verbatim}
will fail.




The file exists, but is not fully qualified by extension.
\begin{verbatim}
run -ascii data/flea
run  data/flea
\end{verbatim}
\end{document}





\section{Tests}
\begin{verbatim}
run -ascii data/flea
run -ascii data/flea.dat
run  data/flea        # xml_data, data/flea.xml
run  data/flea.dat
run  -xml data/flea.xml
run  data/flea.xml
run -verbose data/buckyball
\end{verbatim}


All of the read routines now take a (reference to a) InputDescription
object which contains information about how to find files.
Additionally, one can call addInputFile() and have it record that the
that file was opened and read.


All of the ascii 
